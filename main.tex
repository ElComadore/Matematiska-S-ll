\documentclass[a4paper]{article}
\usepackage{graphicx}
\usepackage{amsmath, amssymb}
\usepackage{hyperref}
\usepackage{csquotes}
\usepackage{amsthm}
\usepackage{array}
\usepackage{multicol}
\usepackage{tikz}
\usetikzlibrary{patterns,angles,calc,arrows.meta,decorations.markings,arrows,intersections,shapes}
\usepackage{pgf,pgfplots}
\pgfplotsset{compat=1.15}
\usepackage{mathrsfs,subcaption,rotating}
\usepackage[
backend=biber,
style=alphabetic,
sorting=ynt
]{biblatex}
\addbibresource{bibliografi.bib}


%\usepackage{epsfig}
\usepackage{floatflt} %för inkapslade bilder.
\usepackage{epstopdf}
\usepackage{fancyhdr}
\usepackage[top=3cm, bottom=3cm,inner=3cm, outer=3cm]{geometry}	
\setlength{\headheight}{61pt}
%\addtolength{\textwidth}{20mm}
%\addtolength{\textheight}{30mm}
%\addtolength{\textheight}{10mm}
%\addtolength{\headheight}{-10mm}
%\addtolength{\oddsidemargin}{-10mm}
\usepackage{eso-pic}								% Create cover page background
\newcommand{\backgroundpic}[3]{
	\put(#1,#2){
	\parbox[b][\paperheight]{\paperwidth}{
	\centering
	\includegraphics[width=\paperwidth,height=\paperheight,keepaspectratio]{#3}}}}


% Här nedan kommer kommandon där du skall göra val
%


\newcommand{\ARBETE}{ %välj arbetsbenämning här, notera att båda kan användas samtidigt
 \newline \noindent Examensarbete för kandidatexamen i matematik vid Göteborgs universitet % om någon i gruppen läser på GU
% \medskip
% \newline \noindent Kandidatarbete inom civilingenjörsutbildningen vid Chalmers % om någon läser på Chalmers
 \medskip
}


\newcommand{\titel}{Titel} %Skriv in projektets/rapportens titel här
\newcommand{\undertitel}{Eventuell undertitel} %Skriv in ev undertitel här, eller lämna tomt
\newcommand{\engtitel}{Engelsk titel} %Skriv in engelsk översättning av projektets/rapportens titel här


\newcommand{\namn}{ %Skriv in medlemmarnas namn i bokstavsordning.
    Kawthar Abdalla \\
    Nils Alexandersson \\
    Erik Dagobert \\
    Coën Lorcan Olofsson
}

\newcommand{\examina}{ % här skall gruppens medlemmar skrivas in vid önskad examen. Aktivera aktuella examina, inte kurskoder. Vid fyra med samma examen blir det snyggast om man gör en tabell.
%  \newline \noindent \begin{tabular}{ll}
%  förnamn efternamn 1 &
%  förnamn efternamn 2  \\
%  förnamn efternamn 3 &
%  förnamn efternamn 4
% \end{tabular}
% Vid fem eller sex görs tabellen med tre kolumner
% Vid flera examina skall \bigskip aktiveras utom för den sista.
%
%
%%%%%%%%%%%%%%%%%% Kurs MMG900 %%%%%%%%%%%%%%%%
% \newline \noindent {\it Examensarbete för kandidatexamen i matematik vid Göteborgs universitet} \smallskip
% \newline \noindent förnamn efternamn 1
% \quad förnamn efternamn 2
% \quad förnamn efternamn 3
% \quad förnamn efternamn 4
% \bigskip
%%%%%%%%%%%%%%%%%% Kurs MMG910  %%%%%%%%%%%%%%%%
\newline \noindent{ \it  Examensarbete för kandidatexamen i matematik inom Matematikprogrammet vid Göteborgs universitet} \smallskip
% \newline \noindent förnamn efternamn 1
% \quad förnamn efternamn 2
% \quad förnamn efternamn 3
% \quad förnamn efternamn 4
% \bigskip
%
\newline \noindent \begin{tabular}{@{}ll}
    Kawthar Abdalla &
    Nils Alexandersson \\
    Erik Dagobert &
    Coën Lorcan Olofsson
\end{tabular}
\bigskip
%%%%%%%%%%%%%%%%%% Kurs MMG920  %%%%%%%%%%%%%%%%%
%  \newline \noindent {\it Examensarbete för kandidatexamen i matematik inom Matematikprogrammet, inriktning Tillämpad matematik, vid Göteborgs universitet} \smallskip
%  \newline \noindent förnamn efternamn 1
%  \quad förnamn efternamn 2
%  \quad förnamn efternamn 3
%  \quad förnamn efternamn 4
%  \bigskip
%
%%%%%%%%%%%%%%%%%%% Kurs MSG900  %%%%%%%%%%%%%%%%%
% \newline \noindent {\it Examensarbete för kandidatexamen i matematisk statistik vid Göteborgs universitet} \smallskip
% \newline \noindent förnamn efternamn 1
% \quad förnamn efternamn 2
% \quad förnamn efternamn 3
% \quad förnamn efternamn 4
% \bigskip
%
%%%%%%%%%%%%%%%%%%% Kurs MSG910  %%%%%%%%%%%%%%%%%
%\newline \noindent {\it Examensarbete för kandidatexamen i matematisk statistik inom Matematikprogrammet vid Göteborgs universitet} \smallskip
%
%\newline \noindent förnamn efternamn 1
%\quad förnamn efternamn 2
%\quad förnamn efternamn 3
%\quad förnamn efternamn 4
% \bigskip
%
%%%%%%%%%%%%%%%%%%% Kurs MVEX01  %%%%%%%%%%%%%%%%%
% \newline \noindent {\it Kandidatarbete i matematik inom civilingenjörsprogrammet Automation och mekatronik vid Chalmers} \smallskip
% \newline \noindent förnamn efternamn 1
% \quad förnamn efternamn 2
% \quad förnamn efternamn 3
% \quad förnamn efternamn 4
 % \bigskip
%
% \newline \noindent {\it Kandidatarbete i matematik inom civilingenjörsprogrammet Datateknik vid Chalmers} \smallskip
% \newline \noindent förnamn efternamn 1
% \quad förnamn efternamn 2
% \quad förnamn efternamn 3
% \quad förnamn efternamn 4
% \bigskip
%
% \newline \noindent {\it Kandidatarbete i matematik inom civilingenjörsprogrammet Maskinteknik vid Chalmers} \smallskip
%
% \newline \noindent förnamn efternamn 1
% \quad förnamn efternamn 2
% \quad förnamn efternamn 3
% \quad förnamn efternamn 4
% \bigskip
%
%   \newline \noindent {\it Kandidatarbete i matematik inom civilingenjörsprogrammet Teknisk fysik vid Chalmers} \smallskip
%   \newline \noindent förnamn efternamn 1
%   \quad förnamn efternamn 2
%   \quad förnamn efternamn 3
%   \quad förnamn efternamn 4
%   \bigskip
%
%   \newline \noindent {\it Kandidatarbete i matematik inom civilingenjörsprogrammet Teknisk matematik vid Chalmers} \smallskip
%   \newline \noindent förnamn efternamn 1
%   \quad förnamn efternamn 2
%   \quad förnamn efternamn 3
%   \quad förnamn efternamn 4
%   \bigskip
%
 \bigskip

}


\newcommand{\handledare}{% om samtliga handledare är från MV lämnas fältet inst tomt
Lucile Devin \\
& Anders Södergren
%  namn 1& inst \\
%  & namn 2 & inst \\
%  & namn 3 & inst \\
}



% Här nedan kommer kommandon som du inte skall ändra, de ger utformningen av rapporten.

\newcommand{\skribenter} {\begin{tabular}{l} \namn \end{tabular}}

%%%%%%%%%%%%%%%%% Här börjar utformningen av omslaget %%%%%%%%%%%%%%%%
\newcommand{\omslag}{
%\newgeometry{top=3cm, bottom=3cm,left=2.25 cm, right=2.25cm}	% Temporarily change margins	
\newgeometry{top=3cm, bottom=4cm,left=2 cm,right=1cm}	% Temporarily change margins	
%\thispagestyle{empty}
% \addtolength{\textheight}{-20mm}
\pagestyle{fancy}
\pagenumbering{gobble}
\fancyhead[C]{\includegraphics[width=170mm]{Chalmers_GU_svart-eps-converted-to.pdf}\\}
\addtolength{\voffset}{0.3cm}
\renewcommand{\headrulewidth}{1pt}


\parbox{17cm}{
\vspace{60mm}

\noindent{\Huge \titel}
\bigskip

\noindent {\Large \undertitel}
\bigskip

\noindent{\huge \engtitel}

\noindent\hspace*{-1 ex}{\Large \it
\ARBETE
}
\vspace{20mm}

\noindent\hspace*{-1 ex}\parbox{80mm}{\noindent {\huge \skribenter}}}

\renewcommand{\footrulewidth}{1pt}

\fancyfoot[L]{\vspace{0.1mm}\large Institutionen för Matematiska vetenskaper\\
CHALMERS TEKNISKA HÖGSKOLA\\
GÖTEBORGS UNIVERSITET\\
Göteborg, Sverige 2020 }


\newpage
\thispagestyle{empty}
\mbox{}

\newpage}
%%%%%%%%%%%%%% Utformning av omslag slut %%%%%%%%%%%%%%%%%%%%

%%%%%%%%%%%%%%%%%%%% Här börjar utformning av titelsidor %%%%%%%%%%%%%%%%
\newcommand{\titelsidor}{
\newgeometry{top=3cm, bottom=5cm,left=3 cm,right=3cm}

\thispagestyle{fancy}
\fancyhf{}
\renewcommand{\headrulewidth}{0pt}


\mbox{}
\vspace{50mm}

\noindent {\LARGE \titel}\bigskip\bigskip

\noindent {\large \undertitel}
\vspace{30mm}


\noindent {\large \examina}


\vfill
\hspace{-5.8 ex} \begin{tabular}[t]{lll}
Handledare:& \handledare & \end{tabular}


\renewcommand{\footrulewidth}{0pt}

\fancyfoot[L]{\vspace{0.1mm}\large Institutionen för Matematiska vetenskaper\\
CHALMERS TEKNISKA HÖGSKOLA\\
GÖTEBORGS UNIVERSITET\\
Göteborg, Sverige 2020 }




\newpage
\thispagestyle{fancy}
\fancyhf{}
\newgeometry{top=3cm, bottom=3cm,left=3 cm,right=3cm}
\mbox{}
\vfill

\setcounter{page}{0}
\newpage }



\usepackage[T1]{fontenc}                % För svenska bokstäver
\usepackage[swedish,english]{babel}             % För svensk avstavning och svenska rubriker (t ex "Innehållsförteckning")
% Egna kommandon:
% Delar-operator:
\newcommand{\divides}{\mid}
\newcommand{\notdivides}{\nmid}
% Största gemensamma delare
\renewcommand{\gcd}[1]{(#1)}
% Kardinalitet
\newcommand{\card}[1]{\# #1}
% Heltalsmängd:
\newcommand{\A}{\mathcal{A}}
% Primtalsmängd:
\renewcommand{\P}{\mathcal{P}}
% Sållad mängd:
\renewcommand{\S}[3]{S(\mathcal{#1}, \mathcal{#2}, #3)}




% Satser etc. mer info hittas på https://www.overleaf.com/learn/latex/theorems_and_proofs
\newtheorem{theorem}{Sats}[section]
\newtheorem{corollary}{Korollarium}[theorem]
\newtheorem{lemma}[theorem]{Lemma}
\newtheorem{definition}[theorem]{Definition}
\newtheorem{proposition}{Proposition}[section]

% Block-kommentar: använd '\begin{comment} ... \end{comment}'

% Todo
\newcommand{\todo}[1]{\smallskip\noindent\big[To do: #1\big]\smallskip}



\begin{document}
\selectlanguage{swedish}
\omslag

\titelsidor
\thispagestyle{empty}
%\setlength{\textheight}{240mm}
%\addtolength{\topmargin}{-50mm}
\newgeometry{top=3cm, bottom=3cm,left=3 cm,right=3cm}

\section*{Förord}

I förordet skall det anges vilka delar som skall tillskrivas respektive författare. Det skall också anges att en loggbok förts över de enskilda medverkandes prestationer under arbetet. Med loggbok menas här gruppens dagbok och de individuella tidsloggarna.

\newpage

\section*{Populärvetenskaplig presentation}

Här skriver man populärvetenskaplig presentation

\newpage
\begin{abstract}

\end{abstract}

\selectlanguage{english}
\begin{abstract}

\end{abstract}

\newpage
\selectlanguage{swedish}
\pagestyle{plain}
\tableofcontents                % Innehållsförteckning

                % Tabellförteckning

\newpage
\pagenumbering{arabic}
\section{Inledning}
% \documentclass[a4]{article}
% \usepackage{graphicx}
% \usepackage{amsmath, amssymb}
% \usepackage{epsfig}
% \usepackage{floatflt} %för inkapslade bilder.

% \addtolength{\textwidth}{10mm}
% \addtolength{\textheight}{30mm}
% \addtolength{\headheight}{-10mm}

% \usepackage[T1]{fontenc}                % För svenska bokstäver
% \usepackage[swedish]{babel}             % För svensk avstavning och svenska
%                                         % rubriker (t ex "innehållsförteckning)

%Var noggranna med att ange källor till det ni skriver. Vi rekommenderar Vancouver-systemet\footnote{Även annat system accepteras om det används konsekvent.} som är mest använt på MV. Man kan antingen använda siffror [1], [2] etc, eller initialer som associerar till författarnamnet(n) t.ex [BN], [BS] etc. Det senare kan vara lite jobbigt om man har många källor men praktiskt om man har några som huvudreferenser. Läs mer i Fackspråks skrift: Utformning av rapporter och kandidatarbetens skriftliga .... (2008-01-11)\cite{rapp}. I det fall arbetet i huvudsak bygger på en eller ett par källor och det är svårt att identifiera exakt när man använder respektive källa, kan man tala om detta i inledningen. Man kan sedan referera till källan om man återger en definition, en sats eller ett bevis eller på annat sätt ligger nära källan. En direkt översättning kan jämställas med ett citat, återberätta därför som om ursprunget var en skrift på svenska så att ni håller er långt ifrån gränsen för plagiering. Är ni osäkra på något så fråga examinator eller handledare.\footnote{Läs mer om att hantera källor och akademisk hederlighet på Chalmers webbsida \hfill \\ URL: https://writing.chalmers.se/chalmers-skrivguide/att-hantera-kallor  }

\begin{comment}
Förutom att vara klassiska frågor inom talteori angående primtal, vad har primtalstvillingshypotesen, Goldbach hypotesen, och storleken på primtalsgap att göra med varandra? Kort sagt; sållteori. Lite längre sagt; under den senaste århundrade har tekniker inom sållteori utvecklades och tillämpades på dessa, och fler, blandade problem inom talteori, och även vidare ämnen, med relativt stort succé. Med hjälp av sållteori har matematiken lyckats bevisa att det finns oändligt många tal \textit{p} sådan att \textit{p} och \(p+2\) är antingen prima eller semiprima \cite{chen2Prime} och att det det finns alltid två primtal inom 600 heltal av varandra \cite{mayBound}. Dessa tillämpningar väcker frågan, vad är ett matematisk såll för något? 

Ett matematisk såll är en metod med sin ursprung i talteori som försöker att uppskata kardinaliteten av en så kallad siktad mängd där alla element i mängden har någon gemensam egenskap. Ett prototypisk exempel på en sådan mängd som man är intresserad i storleken av är mängden som består av alla primtal mindre än något tal \textit{x}. Sållteori har försökt att hitta och förfina uppskattningar av storleken på precis den mängden sedan Chebychev's berömd uppskattning i 1851. Dock har matematiska såll sin ursprung ännu längre sedan i antikens grekland med arbetet av Erastostenes. Hans idé följer nedanstående mönster;
\begin{enumerate}
    \item Med början vid 2, lista ut alla tal upp till talet du vill använda som övergräns.
    \item Rita en cirkel kring 2 och stryk över alla tal delbar med 2.
    \item Rita en cirkel kring det nästa talet som inte har en linje genom sig och dra en linje igenom alla tal delbar med den.
    \item Upprepa föregående steg tills alla tal på listan har antingen en cirkel kring de eller har en linje ritat över de.
\end{enumerate}
Då man har tillämpat klart Eratosthenes ursprunglig såll, blir alla tal med cirklar kring sig alla primtal som är mindre än övergränsen man valde. Även om det har gått nästan två årtusenden sedan upptäckten av Eratosthenes ursprunglig metod har metoden fortfarande samband med nutidens matematiska såll; samband som en fokus på delbarhet och, följaktligen, specifika modulo klasser av primtal. 

Det är dessa nya matematiska sålltekniker och deras tillämpningar på blandade problem samt numerisk implementering som vår rapport kommer att fokusera på. Vi kommer att hålla oss till att redovisa hur tre olika sållmetoder, nämligen Eratosthenes allmänna såll, Bruns såll, och Selbergs såll, kan tillämpas på XXX och YYY, samt kommer vi redovisa hur man kan implementera i kod ZZZs såll. Vi kommer dessutom att diskutera möjliga förbättringar av både av dessa undersökningar vi har gjort. Men innan vi kan börja med att redovisa vårt arbete kommer vi först att nämna några förkunskaper som en läsare borde ha, sedan kommer vi introducera några grundläggande begrepp och teori inom sållteori, och slutligen kommer vi att introducera mer fullständigt de såll vi har valt att jobba med.
\end{comment}

Den som någon gång har funderat kring primtal, kanske har provat att ta en lista med heltal och börjat markera de tal som är prima. Efter en liten stund kanske man märker att för att hitta alla primtal upp till ett visst tal behöver man bara alla primtal mindre än eller lika med kvadratroten av det talet. Man kanske också börjar lägga märke till mönster som uppträder; såsom att det finns vissa par av primtal som har bara ett tal mellan sig. 

Idén bakom denna process, att hitta primtal i en lista av naturliga tal, har funnits sedan antikens grekland med en algoritm som har tillskrivits den grekiska polyhistorn Eratosthenes (ca. 276 - 194 f.v.t.). Algoritmen har följande struktur;
\begin{enumerate}
    \item Börja med talet 2 och lista alla naturliga tal upp till någon gräns.
    \item Ringa in 2 och stryk över alla andra tal som är delbara med 2.
    \item Ringa in nästa tal som inte är struket och stryk alla andra tal delbara med det nya inringade talet.
    \item Upprepa föregående steg tills varje tal på listan är struket eller inringat. 
\end{enumerate}
När algoritmen är avslutad så har varje primtal i listan blivit inringat och alla andra tal har strukits. Eratosthenes algoritm la grunden för nutidens sållteori; ett område inom matematiken som försöker uppskatta storkleken på så kallade \textit{siktade mängder}. 

En siktad mängd är en mängd där alla element är heltal och har någon gemensam egenskap t.ex. en mängd som består av endast primtal eller mängden av alla heltalslösningar till en ekvation. Den grundläggande sållteorins största fördel är att den är relativt elementär och flexibel, speciellt jämfört med andra metoder inom analytisk talteori. Det krävs inga idéer från komplexanalys som t.ex. Dirichlets L-funktioner eller serier för att ha användning av de enklare sållen och om man kan formulera vikterna på ett korrekt sätt, går det att tillämpa metoderna på nästan vilken mängd som helst. Trots sina enkla konstruktion, kan dessa matematiska såll fortfarande ge starka resultat, även om noggrannhet måste offras lite. En exempel på detta är att asymptotiska beteendet av \(x/\log(x)\) för antalet primtal mindre än \textit{x} som ges i primtalssatsen. Detta kan nästan bevisas utan något arbete med zeta funktioner då man använder Selbergs såll, dock så får man bara en övre gräns av \(x/\log(x)\) istället för asymptotiska beteendet. Mer avancerade sållmetoder har givit svar på frågor närliggande till primtalstvillingshypotesen i \cite{chen2Prime}, och storleken på primtalsgap mellan ett antal primtal i rad i \cite{mayBound}.

Vår rapport kommer att fokusera på små, kombinatoriska såll. Små innebär att de fokuserar på att uppskatta storleken på ett litet antal restklasser modulo primtal, och att ett såll är kombinatorisk innebär att den använder sig av inklusion-exklusionsprincipen för att dela upp mängden på ett lämpligt sätt. Bland denna typ av såll håller vi oss till Eratosthenes allmänna såll, Bruns såll, och Selbergs såll. För varje såll kommer vi att ge en naturlig härledning till dess formulering och en förklaring till hur det används. Efteråt kommer vi att redovisa och analysera en implementering i kod av Eratosthenes algoritm, vilket följer metoden som beskrivs i \cite{HaraldSieve}. Dock, innan vi börjar redovisa någon av sållen vill vi gå igenom några förberedelser och förklara det allmänna sållproblemet.

\section{Sållteorins grunder och notation}
% Skrivet av Erik

I den här texten kommer vi att låna större delen notationen från \cite{cojocarumurty}, exempelvis skriver vi \(p, q, l\) menas primtal, \(n, d, k\) naturliga tal och \(x, y, z \in \mathbb{R}_+\). Den största gemensamma delaren noteras \(\gcd{d, k}\) och funktionen \(\nu(d)\) beskriver antalet distinkta primtalsdelare av \(d\). Nedanstående sektioner syftar till att etablera mer notation och tekniker som är vanligt förekommande i sållteori med utgångspunkt i Cojocarus och Murtys bok.

%\subsection{O-notation}

\subsection{Det allmänna sållproblemet}
Det generella fallet i sållteori är att vi har en mängd heltal \(\A\), en mängd primtal \(\P\) samt delmängder \(\A_p, \forall p \in \P\) och är intresserade av att sålla fram delmängden \(\mathcal{S}(\A, \P) := \A \setminus \cup_{p \in \P} \A_p\) eller, vanligare, kardinaliteten av dito. För \(d\), en kvadratfri produkt (dvs. utan delare \(p^\alpha\) för \(\alpha > 1\)) av element \(p \in \P\) definierar vi \(\A_d = \cap_{p \divides d} \A_p\). Om \(P(z) = \prod_{\substack{p\in \mathcal{P} \\ p < z}} p\) så betecknar vi kardinaliteten av den sållade mängden som
\begin{align*}
    \S{A}{P}{z} = \card{(\A \setminus \cup_{p \divides P(z)} \A_p)} .
\end{align*}



%\subsection{O-notation}
% Intresserade av en övre asymptotisk gräns --> förklara Ordonotation. 

\subsection{Möbiusfunktionen}
En användbar funktion i sållteori är möbiusfunktionen,
\begin{equation*}
    \mu(n) = 
    \begin{cases}
        1, & \text{om}\ n \text{ är ett kvadratfritt, naturligt tal med jämnt antal primdelare}\\
        -1, & \text{om}\ n \text{ är ett kvadratfritt, naturligt tal med udda antal primdelare}\\
        0, & \text{om}\ n \text{ inte är kvadratfri}
    \end{cases}
\end{equation*}
som bland annat förenklar inklusion-exklusionsprincipen så att, för scenariot ovan,
\begin{align*}
    \S{A}{P}{z} = \sum_{\substack{d \divides P(z) \\ d \leq z}} \mu(d) \card{\A_d} .
\end{align*} % sida 72

Två andra viktiga egenskaper hos möbiusfunktionen är den så kallade fundamentala egenskapen,
\begin{equation*}
    \sum_{d \divides n} \mu(d) =
    \begin{cases}
        1, & \text{if}\ n = 1 \\
        0, & \text{if}\ n > 0
    \end{cases}
\end{equation*}
och Möbius inverteringsformel som säger
\begin{equation*}
    f(n) = \sum_{d \divides n} g(d) \implies g(n) = \sum_{d \divides n} \mu(d) f(n/d)
\end{equation*}
om \(f, g : \mathbb{N} \to \mathbb{C}\).


%\subsection{Abels summationsformel}

\section{Eratostenes generaliserade såll}
% 

\subsection{Eratosthenes–Legendre Såll}
\hspace{0.3cm} Den gregiska matematikern, astronomen och poeten Eratosthenes från Cypern (can276 f.v.t – ca 194 f.v.t), var en av de tidigaste matematikerna som arbetade med primtal. I en av hans stora presentationer inom matematik introducerade han ”Eratosthenes såll”, vilket är ett effektivt sätt att identifiera primtal upp till en viss gräns, som senare blev grunden för sållteorin.
År 1808 introducerade den franska matematikern Adrien-Marie Legendre (1752-1833) det som kom att kallas Legendres Såll i sin bok "Théorie des nombres", som är en formulering av Eratosthenes-metoden, som heter "Eratosthenes-Legendre såll" eftersom den bygger på Eratosthenes algoritm. Hur har denna såll startat den moderna sållen och vilka upptäckter gjordes med hjälp av Eratosthenes-Legendre såll?\\



 Legendre formaliserade matematiken bakom Eratosthenes algoritmen genom att använda den inklusiv-exklusivprincipen för att beräkna antal elementer i en given mängd av naturliga tal  $ A\subset Z_{+} $ som inte har några primtalsfaktorer nedan $ z $. Med $  \#A=X $, tar vi alla $ x $ tal, och vi substraherar antalet multiplar av 2, även vi substraherar antalet multiplar av 3, samma sak med multiplar av 5, sen lägger vi till antalet av multiplar $ 2\times3 $ (eftersom multipler av 6 substraherades två gånger, en gång med multiplarna av 2 och den andra med dem av 3); och vi fortsätter på detta sätt genom alla värden på  $ d $ som delar produkten av primtal upp till  $ z $. \\
 

Låt $ P_z $ vara produkten av all primtal i $ \P $ mindre än $ z $, vi har
\[P_z:=\prod_{p<z}p\]
Låt $ S(\A, P) $ beteckna antal heltal i $ \A $ som är relativt prima till $ P_z $, således
\[S(\A, P)=\vert \lbrace n\in \mathbb{N}:1\leq n\leq x, (n,P_z)=1 \rbrace\vert\]


Legendres identiteten uttrycks av: 

\begin{theorem}[Legendres identitet \cite{Terence}] kan uttryckas genom:
\[S(\A, P)=\sum_{d\divides P_z}\mu(d)\sum_{\substack{n\leq x;\\d\divides n}}1. \]
% Where the $ S(\A, P(z)) $ calculate the number of natural numbers $ n\leq x $ in $ A $ coprime to $ P(z) $.
\end{theorem}
Vi kommer att använda denna sats för att komma till en allmän övre gräns för $ S(\A, P) $\\

Vi har
\begin{align*}
S(\A, P) & =\sum_{d\divides P_z}\mu(d)\Bigl\lfloor \frac{x}{d}\Bigr\rfloor\\
 &  =\sum_{d\divides P_z}\mu(d) \left( \frac{x}{d}+ \Bigl\lfloor \frac{x}{d} \Bigr\rfloor - \frac{x}{d} \right)\\
 & =\sum_{d\divides P_z}\mu(d)\frac{x}{d}+\sum_{d\divides P_z}O(1)\\
 & =\sum_{d\divides P_z}\mu(d)\frac{x}{d}+ O\left(\sum_{d\divides P_z}1\right)
\end{align*}

Så, vi kan hitta en uppskattning för den inre summan i Legendres identitet som följande:
\[\sum_{n\leq x;d\divides n}1=\frac{x}{d}+O(1); \]
Vi vet att antalet åtskilda faktorer som $ P(z) $ har är $ 2^{\pi(z)} $ vi härleda
\[S(\A, P)=\sum_{d\divides P_z}\mu(d)\frac{x}{d}+ O(2^{\pi(z)}) \]
För varje $ p $ i $ \P $, vi betecknar $ \omega(p) $ antalet av utvalda restklasser mod $ p $ i mängden $ \lbrace \omega_{1,p},...,\omega_{\omega_{p},p}\rbrace $ för varje primtal i $ \P $, vi definerar $ W(z) $ som
\[W(z)=\prod_{\substack{p\in \P\\p<z}}\left( 1-\frac{\omega(p)}{p}\right)\]
Genom att faktorisera den första termen av summan 
\[\sum_{d\divides P_z}\mu(d)\frac{x}{d}= x\prod_{p< z}\left( 1-\frac{1}{p} \right) \]
och med ersättning med den nya faktoriserad termen, vi får
\[S(\A, P)=\left( \prod_{p< z}\left( 1-\frac{1}{p} \right)\right) x +O(2^{\pi(z)}) \]




\subsection{Eratosthenes Generella Såll}
Efter att vi såg den grundläggande sållen, realiserar vi att den inte ger en nyttig uppskattning av $ S(\A, P) $ på grund av dess stora feltermen. Den grundläggande metoden av Eratosthenes-Legendre såll kan generaliseras, vi kan anpassa den för att den kan användas med andra talföljder av siffror som kvadratfria heltal. En kvadratfri heltal är en heltal som är en produkt av åtskilda primtal. Dess mer generella såll kommer att vara mycket bättre och effektivare med de kvadratfria tal i grunden eftersom de är tätare än primtal, detta kommer att hjälpa oss att hitta intressanta gränser för dessa kvadratfria siffror.

Den generalla såll kan användas för att demonstrera många intressanta resultater. I denna del kommer vi att introducera Eratosthenes generella såll och dess bevis.

Efter att vi sett Legendres identitet i sats 3.1 och förutom definitionerna i förgående avsnittet, så kan vi skriva om sållen och formulera den som följande:


Låt $ P(z) $ vara produkten av alla primtal i $ \P $ mindre än $ z $, d.v.s
\[P(z)=\prod_{\substack{p \in \P\\ p \leq z}} p \]

Vi definerar $\A_{p}$ som en mängd av elementer av $ \A $ på ett sätt där det tillhör åtminstone en av $\omega(p)$ utvalda restklasser mod p.

Låt $ d $ vara kvadratfri som endast är en produkt av primfaktorer i $ \P $, vi sätter $\omega(d)=\prod_{p \divides d} \omega(p)$

Så vi har
\[\omega(d)=\prod_{p\divides d}\omega(p)\]

Låt $ R_{d} $ representera feltermen i uppskattningen av antal elementerna av $ \A_{d} $ 


Låt $ \kappa $ vara siktdensitetkonstanten, det är ett viktat medelvärdet av antalet restklasser som siktas ut av varje primtal.


Antar att $ \vert R_{d}\vert=O(\omega(d))$ and $ \exists\kappa\geq 0 $ så att
\begin{equation}
\label{equationEL3}
\sum_{p\divides P(z)}\frac{\omega(p)\log(p)}{p}\leq \kappa \log(z)+O(1)
\end{equation}  
Med partiell summering får vi
\[\sum_{p\divides P(z)}\frac{\omega(p)}{p}\leq\kappa\log\log z+O(1)\]
   Vi har de följande Lemman som ska användas för att bevisa Eratosthenes generella såll satsen.
\begin{lemma}
     Antar \ref{equationEL3} har vi
$$
\sum_{d<t, d \divides P(z)} \omega(d)=O\left(t(\log z)^{\kappa} \exp \left(-\frac{\log t}{\log z}\right)\right)
$$
där den stora- $O$ gräns är för $z \rightarrow \infty$ och konstanten bero på $P, R_{p}, \kappa$.
\end{lemma}
\begin{lemma}
    Fixa $C>0 .$ Antar (1) har vi
$$
\sum_{d>C x, d \divides P(z)} \frac{\omega(d)}{d}=O\left((\log z)^{\kappa+1} \exp \left(-\frac{\log x}{\log z}\right)\right)
$$
där den stora- $O$ gräns är för $z \rightarrow \infty$ och konstanten bero på $P, R_{p}, \kappa, C .$
 
\end{lemma}

Med de förgående antagendena, Eratosthenes generella såll är formulerad som följande\cite{Dalton}
\begin{theorem}[Eratosthenes Generella Såll]\hfill

För en given $ R_{d} $ antar vi att $ X $ existerar, som verifierar
\[\vert\A_{d}\vert =\frac{\omega(d)}{d}X +R_{d} \]

Låt oss anta att $ \exists \kappa\geq 0 $ och $ \vert R_{d}\vert = O(\omega(d))$ på ett sätt så att

\[\sum_{p\divides P(z)}\frac{\omega(p)\log(p)}{p}\leq \kappa \log(z)+O(1)\]


Med (3) och genom att anta $ \exists y\in \mathbb{R}^{+} $ som verifierar $ \forall d>y, \card{\A_d}=0 $.

får vi
\begin{equation}
\label{equationEL4}
\S\A\P z= XW(z)+O\left( \left( X+\frac{y}{\log(z)} \right)(\log(z))^{\kappa+1}\exp \left( -\frac{\log(y)}{\log(z)} \right) \right)
\end{equation}
\end{theorem}
\begin{proof}
Med de hypoteser som används för Eratosthenes generella sållteorin och den inklusion-exklusionprincipen, får vi
\begin{align*}
\S\A\P z &=\sum_{\substack{d\divides P(z)\\d\leq y}}\mu(d)\#\A_{d} =\sum_{\substack{d\divides P(z)\\d\leq y}}\mu(d)\frac{X\omega(d)}{d}+O(\sum_{\substack{d\leq y,\\ d \divides P(z)}} \omega(d)).
\end{align*}
Så genom att ersätta med de förgående två Lemman, får vi (4)
\end{proof}


\subsection{Applikationer och begränsningar}

En huvudapplikation, är att finna en övre gräns för $ \pi(x) $, mycket gjordes för att förbättra resultatet. Här är några:
\begin{proposition}
    Genom att använda Eratosthenes–Legendre såll, ser vi att den övre gränsen av antalet primtal upp till $ x $ är given av
\[S(\A, P)\ll\frac{x}{\log(\log(x))}\]
\end{proposition}
\begin{proof}
Vi har från sats 3.1
\[S(\A, P)=x \prod_{p\leq z}\left( 1-\frac{1}{p} \right) +O(2^{\pi(z)})\]
För att få en övre gräns för $ S(\A, P) $, vi får en nedre gräns för den reciproka värdet av första termen, 

Genom att använda de geometriska serier, får vi
\[\prod_{p\leq z}\left( 1-\frac{1}{p} \right)^{-1}=\prod_{p\leq z}\sum_{r=0}^{\infty}\frac{1}{p^{r}}\]
Så har vi
\[\prod_{p\leq z}\sum_{r=0}^{\infty}\frac{1}{p^{r}}>\sum_{n<z}\frac{1}{n}\]
För att termen på höger sidan är en delmängd av dem som är på vänster sidan.\\

Genom att jämföra med $ \int_{1}^{z}\frac{1}{x}dx $, får vi
\[\sum_{n<z}\frac{1}{n}>\log(z)\]
Således
\[S(\A, P)<\frac{x}{\log(z)}+O(2^{\pi(z)})\]
För att kontrolera den andra termen (felterm), låt $ z=\log(x)$. Vi vet att $ \pi(z)<z $ så $ 2^{\pi(z)}<2^{z}=2^{\log(x)}=x^{log(2)} $, således får vi
\[S(\A, P)\ll\frac{x}{\log(\log(x))}\]
\end{proof}
Efteråt, en förbättring gjordes, vilket gav en bättre övre gräns för antal primtal upp till $ x $, den ges av nästa förslag
\begin{proposition}
En annan övre gräns för antal primtal upp till x är given av
\[S(\A, P)\ll\frac{x}{\log(x)}(\log\log x).\]
\end{proposition}

En annan viktig tillämpning av Eratosthenes–Legendre såll är att hitta en övre gräns för antalet primstvillning upp till $ x $.
\begin{theorem}[Twin prime upper bound\cite{Aliram}]\hfill

Antal primtal $ p $ with $ p\leq x $ och $ p+2 $ är också en primtal (d.v.s primtalstvillning), är given av
\[\S\A\P z \ll\frac{x}{\log^{2}(x)}(\log\log x)^{2}\]
\end{theorem}
\begin{proof}
Låt $ \A $ beteckna vilken mängs av positiva heltal, $ \P $ betecknar vilken mängd av primtal. Låt ett reelt tal $ z $ such as $ z=z(x) $. Vi skiljer residualklasser 0 och -2 modulo $ p $ för varje primtal $ p<z $ . Given att mängden $ \A_p $ är nollmängd för $ p>x+2 $, och med användning av Eratosthenes Generella Såll (sats 3.4) för $ \kappa=2 $ får vi
\[\S\A\P z = xW(z)+O \left( x(\log z)^{3} \exp \left( -\frac{\log x}{\log z}\right) \right)\]
med
\[W(z)=\prod_{p<z}\left( 1-\frac{2}{p}\right)\]
Vi kan hitta den övre gränsen för $ W(z) $ som följande
\[W(z)=\prod_{p<z}\left( 1-\frac{2}{p}\right)\leq\exp \left( -\sum_{p<z}\frac{2}{p} \right)\ll (\log z)^{-2}\]
Om vi sätter värdet av $ z $ som
\[\log z = \log x/A\log\log x\] med $ A $ en stor positiv konstant
\[\S\A\P z \ll \left( \frac{\log x}{A\log\log x} \right)^{-2}=\left( \frac{A\log\log x}{\log x} \right)^{2}\]
Därför, sammanfattar vi
\[\S\A\P z \ll\frac{x}{\log^{2}(x)}(\log\log x)^{2}\]
\end{proof}
Notera att denna övre gränsen är inte den bästa. 
 


Eratosthenes–Legendre Såll, har många tillämplingar i såll och talteorin. Efter dess presentation, många andra sållmetoder definierades som förlängningar av Eratosthenes–Legendre såll som Brun's såll.


%\begin{thebibliography}{9}

%\bibitem{Terence} 
%Terence Tao. 
%Notes 7: Sieving and expanders, 1 March, 2012,\\
%\url{https://terrytao.wordpress.com/2012/03/01/254b-notes-7-sieving-and-expanders/}

%\bibitem{Dalton} 
%Jack Robert Dalton. 
%An Exposition of Selberg’s Sieve,  Master thesis,  University of Vermont, May, 2017, pages: 32-33.
%\\\url{https://scholarworks.uvm.edu/cgi/viewcontent.cgi?article=1719&context=graddis}

%\bibitem{Aliram}
%Alina Carmen Cojocaru, M. Ram Murty.
%An Introduction to Sieve Methods and Their Applications, Cambridge University Press, London Mathematical Society, 2005, page 73.


%\end{thebibliography}



\section{Bruns såll}
% Skrivet av Nils

År 1915 presenterade norrmanen Viggo Brun (1885-1978) ett såll som senare fått namnet Bruns såll. Brun hade inspirerats av publikationer av Jacques Hadamard som handlade om de framsteg som Jean Merlin gjort i att utöka Eratosthenes teori. Sållet blev inte uppmärksammat direkt, istället tog det ca 30 år innan andra matematiker började intressera sig för det. En anledning till dröjsmålet kan vara att Bruns skrivsätt och val av notation gjorde materialet onödigt svårläst. Sedan det att sållet publicerades har det gjorts flera insatser för att förkorta och förenkla läsningen men trots detta är Bruns såll och dess bevis krävande för läsaren, med detta i åtanke har vi valt att korta ner beviset avsevärt i förhoppning om att stycket ska kännas överkomligt.







%\section{Metod och genomförande}
%% Skrivet av Nils

\textbf{Detta stycke kommer troligtvis inte vara med i den slutgiltiga versionen. Istället kommer dess innehåll fördelas till olika delar av texten där vi finner att det passar.}

Denna rapport är framför allt en litteraturstudie, med stor förankring i \textit{An Introduction to Sieve Methods and Their Applications} av Alina Carmen Cojocaru \& M. Ram Murty (2005). Rapporten ämnar \textbf{att introducera matematikstudenter på sen kandidatnivå till sållteori.} Vi presenterar tre stycken såll som har haft stor teoretisk och historisk betydelse för ämnet och exemplifierar dessa i programmeringsproblem. 

I teoridelen formuleras de mest centrala satserna och idéerna men bevisen har i vissa fall kortats ned och i andra fall helt överlåtits till appendix, detta i syfte att göra texten mer lättillgänglig för läsaren.



%\begin{thebibliography}{FFF}
%\bibitem[UR]{rapp} Utformning av rapporter och kandidatarbetens skriftliga presentation för Civilingenjörsprogrammen vid Chalmers tekniska högskola. 2008. Göteborg: Chalmers Tekniska Högskola

%\end{thebibliography}
\medskip

\printbibliography

\newpage
\appendix
\section{Ordonotation}
I den här uppsatsen gör vi flitig användning av ordonotation för att beteckna olika asymptotiska gränser. För $D \subset \mathbb{C}$ och två funktioner, $f: D \to \mathbb{C}$ och $g: D \to \mathbb{R}_+$ skriver vi
\begin{align*}
    f(x) = O(g(x)) \quad \text{om det finns } A > 0 \text{ så att} \quad \abs{f(x)} \leq A g(x), \forall x \in D.
\end{align*}
Omväxlande skriver vi även \(f(x) \ll g(x)\) med samma betydelse som ovan. Om vi har att \(f(x) \ll g(x)\) och \(g(x) \ll f(x)\) så skriver vi \(f(x) \asymp g(x)\).

\end{document}                 % The input file ends with this command.
