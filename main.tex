\documentclass[a4paper]{article}
\usepackage{graphicx}
\usepackage{amsmath, amssymb}
\usepackage{amsthm}
\usepackage{amssymb}
\usepackage{commath}
\usepackage{array}
\usepackage{multicol}
\usepackage{tikz}
\usetikzlibrary{patterns,angles,calc,arrows.meta,decorations.markings,arrows,intersections,shapes}
\usepackage{pgf,pgfplots}
\pgfplotsset{compat=1.15}
\usepackage{mathrsfs,subcaption,rotating}
\usepackage{float} 
\usepackage[
backend=biber,
sorting=none,
style=vancouver
]{biblatex}
\addbibresource{bibliografi.bib}
\usepackage{verbatim}
\usepackage{comment}  % Tillagt av Nils, för kommentarer; använd /begin{comment}
\usepackage{multirow} % Tillagt av Nils, för sammansatta rader i tabell
\usepackage{float}    % Tillagt av Nils, för placering av figurer; använd [H]
\usepackage{minted}   % Tillagt av Nils, för presentation av kod
\usepackage{tikz}     % Tillagt av Nils, för ritande av geometriska figurer (venndiagrammet i populär-texten)
\usepackage{algorithm}% http://ctan.org/pkg/algorithms
\usepackage[noend]{algpseudocode}% http://ctan.org/pkg/algorithmicx
% Tillagt av Erik, för pseudokod

%\usepackage{epsfig}
\usepackage{floatflt} %för inkapslade bilder.
\usepackage{epstopdf}
\usepackage{fancyhdr}
\usepackage[top=3cm, bottom=3cm,inner=3cm, outer=3cm]{geometry}	
\setlength{\headheight}{61pt}
%\addtolength{\textwidth}{20mm}
%\addtolength{\textheight}{30mm}
%\addtolength{\textheight}{10mm}
%\addtolength{\headheight}{-10mm}
%\addtolength{\oddsidemargin}{-10mm}
\usepackage{eso-pic}								% Create cover page background
\newcommand{\backgroundpic}[3]{
	\put(#1,#2){
	\parbox[b][\paperheight]{\paperwidth}{
	\centering
	\includegraphics[width=\paperwidth,height=\paperheight,keepaspectratio]{#3}}}}

% Ska laddas in sist, i denna ordning.
\usepackage{csquotes}
\usepackage{hyperref}



\DefineBibliographyExtras{swedish}{
  \renewcommand*{\finalnamedelim}{\addcomma\addspace}
}
%\DeclareFieldFormat[article]{number}{\mkbibparens{#1}}

\DeclareFieldFormat{urldate}{[citerad\space#1]}
\DeclareFieldFormat{url}{{Hämtad}\space{från}\addcolon\space\url{#1}}
\renewbibmacro*{url+urldate}{%
  \usebibmacro{urldate}%
  \newunit
  \usebibmacro{url}%
}

\renewbibmacro*{volume+number+eid}{%
  \printfield{volume}%
  \printfield{number}%
 }
\DeclareFieldFormat[article]{number}{\mkbibparens{#1}}






% Här nedan kommer kommandon där du skall göra val
%


\newcommand{\ARBETE}{ %välj arbetsbenämning här, notera att båda kan användas samtidigt
 \newline \noindent Examensarbete för kandidatexamen i matematik vid Göteborgs universitet % om någon i gruppen läser på GU
% \medskip
% \newline \noindent Kandidatarbete inom civilingenjörsutbildningen vid Chalmers % om någon läser på Chalmers
 \medskip
}


\newcommand{\titel}{Matematiska såll och deras tillämpningar} %Skriv in projektets/rapportens titel här
\newcommand{\undertitel}{En introduktion och algoritmisk implementation} %Skriv in ev undertitel här, eller lämna tomt
\newcommand{\engtitel}{Mathematical Sieves and their Applications} %Skriv in engelsk översättning av projektets/rapportens titel här


\newcommand{\namn}{ %Skriv in medlemmarnas namn i bokstavsordning.
    Nils Alexandersson \\
    Erik Dagobert \\
    Coën Lorcan Olofsson
}

\newcommand{\examina}{ % här skall gruppens medlemmar skrivas in vid önskad examen. Aktivera aktuella examina, inte kurskoder. Vid fyra med samma examen blir det snyggast om man gör en tabell.
%  \newline \noindent \begin{tabular}{ll}
%  förnamn efternamn 1 &
%  förnamn efternamn 2  \\
%  förnamn efternamn 3 &
%  förnamn efternamn 4
% \end{tabular}
% Vid fem eller sex görs tabellen med tre kolumner
% Vid flera examina skall \bigskip aktiveras utom för den sista.
%
%
%%%%%%%%%%%%%%%%%% Kurs MMG900 %%%%%%%%%%%%%%%%
% \newline \noindent {\it Examensarbete för kandidatexamen i matematik vid Göteborgs universitet} \smallskip
% \newline \noindent förnamn efternamn 1
% \quad förnamn efternamn 2
% \quad förnamn efternamn 3
% \quad förnamn efternamn 4
% \bigskip
%%%%%%%%%%%%%%%%%% Kurs MMG910  %%%%%%%%%%%%%%%%
\newline \noindent{ \it  Examensarbete för kandidatexamen i matematik inom Matematikprogrammet vid Göteborgs universitet} \smallskip
% \newline \noindent förnamn efternamn 1
% \quad förnamn efternamn 2
% \quad förnamn efternamn 3
% \quad förnamn efternamn 4
% \bigskip
%
\newline \noindent \begin{tabular}{@{}l}
    Nils Alexandersson\\ 
    Erik Dagobert\\
    Coën Lorcan Olofsson
\end{tabular}
\bigskip
%%%%%%%%%%%%%%%%%% Kurs MMG920  %%%%%%%%%%%%%%%%%
%  \newline \noindent {\it Examensarbete för kandidatexamen i matematik inom Matematikprogrammet, inriktning Tillämpad matematik, vid Göteborgs universitet} \smallskip
%  \newline \noindent förnamn efternamn 1
%  \quad förnamn efternamn 2
%  \quad förnamn efternamn 3
%  \quad förnamn efternamn 4
%  \bigskip
%
%%%%%%%%%%%%%%%%%%% Kurs MSG900  %%%%%%%%%%%%%%%%%
% \newline \noindent {\it Examensarbete för kandidatexamen i matematisk statistik vid Göteborgs universitet} \smallskip
% \newline \noindent förnamn efternamn 1
% \quad förnamn efternamn 2
% \quad förnamn efternamn 3
% \quad förnamn efternamn 4
% \bigskip
%
%%%%%%%%%%%%%%%%%%% Kurs MSG910  %%%%%%%%%%%%%%%%%
%\newline \noindent {\it Examensarbete för kandidatexamen i matematisk statistik inom Matematikprogrammet vid Göteborgs universitet} \smallskip
%
%\newline \noindent förnamn efternamn 1
%\quad förnamn efternamn 2
%\quad förnamn efternamn 3
%\quad förnamn efternamn 4
% \bigskip
%
%%%%%%%%%%%%%%%%%%% Kurs MVEX01  %%%%%%%%%%%%%%%%%
% \newline \noindent {\it Kandidatarbete i matematik inom civilingenjörsprogrammet Automation och mekatronik vid Chalmers} \smallskip
% \newline \noindent förnamn efternamn 1
% \quad förnamn efternamn 2
% \quad förnamn efternamn 3
% \quad förnamn efternamn 4
 % \bigskip
%
% \newline \noindent {\it Kandidatarbete i matematik inom civilingenjörsprogrammet Datateknik vid Chalmers} \smallskip
% \newline \noindent förnamn efternamn 1
% \quad förnamn efternamn 2
% \quad förnamn efternamn 3
% \quad förnamn efternamn 4
% \bigskip
%
% \newline \noindent {\it Kandidatarbete i matematik inom civilingenjörsprogrammet Maskinteknik vid Chalmers} \smallskip
%
% \newline \noindent förnamn efternamn 1
% \quad förnamn efternamn 2
% \quad förnamn efternamn 3
% \quad förnamn efternamn 4
% \bigskip
%
%   \newline \noindent {\it Kandidatarbete i matematik inom civilingenjörsprogrammet Teknisk fysik vid Chalmers} \smallskip
%   \newline \noindent förnamn efternamn 1
%   \quad förnamn efternamn 2
%   \quad förnamn efternamn 3
%   \quad förnamn efternamn 4
%   \bigskip
%
%   \newline \noindent {\it Kandidatarbete i matematik inom civilingenjörsprogrammet Teknisk matematik vid Chalmers} \smallskip
%   \newline \noindent förnamn efternamn 1
%   \quad förnamn efternamn 2
%   \quad förnamn efternamn 3
%   \quad förnamn efternamn 4
%   \bigskip
%
 \bigskip

}


\newcommand{\handledare}{% om samtliga handledare är från MV lämnas fältet inst tomt
Lucile Devin \\
& Anders Södergren
%  namn 1& inst \\
%  & namn 2 & inst \\
%  & namn 3 & inst \\
}



% Här nedan kommer kommandon som du inte skall ändra, de ger utformningen av rapporten.

\newcommand{\skribenter} {\begin{tabular}{l} \namn \end{tabular}}

%%%%%%%%%%%%%%%%% Här börjar utformningen av omslaget %%%%%%%%%%%%%%%%
\newcommand{\omslag}{
%\newgeometry{top=3cm, bottom=3cm,left=2.25 cm, right=2.25cm}	% Temporarily change margins	
\newgeometry{top=3cm, bottom=4cm,left=2 cm,right=1cm}	% Temporarily change margins	
%\thispagestyle{empty}
% \addtolength{\textheight}{-20mm}
\pagestyle{fancy}
\pagenumbering{gobble}
\fancyhead[C]{\includegraphics[width=170mm]{Chalmers_GU_svart-eps-converted-to.pdf}\\}
\addtolength{\voffset}{0.3cm}
\renewcommand{\headrulewidth}{1pt}


\parbox{17cm}{
\vspace{60mm}

\noindent{\Huge \titel}
\bigskip

\noindent {\Large \undertitel}
\bigskip

\noindent{\huge \engtitel}

\noindent\hspace*{-1 ex}{\Large \it
\ARBETE
}
\vspace{20mm}

\noindent\hspace*{-1 ex}\parbox{80mm}{\noindent {\huge \skribenter}}}

\renewcommand{\footrulewidth}{1pt}

\fancyfoot[L]{\vspace{0.1mm}\large Institutionen för Matematiska vetenskaper\\
CHALMERS TEKNISKA HÖGSKOLA\\
GÖTEBORGS UNIVERSITET\\
Göteborg, Sverige 2021 }


\newpage
\thispagestyle{empty}
\mbox{}

\newpage}
%%%%%%%%%%%%%% Utformning av omslag slut %%%%%%%%%%%%%%%%%%%%

%%%%%%%%%%%%%%%%%%%% Här börjar utformning av titelsidor %%%%%%%%%%%%%%%%
\newcommand{\titelsidor}{
\newgeometry{top=3cm, bottom=5cm,left=3 cm,right=3cm}

\thispagestyle{fancy}
\fancyhf{}
\renewcommand{\headrulewidth}{0pt}


\mbox{}
\vspace{50mm}

\noindent {\LARGE \titel}\bigskip\bigskip

\noindent {\large \undertitel}
\vspace{30mm}


\noindent {\large \examina}


\vfill
\hspace{-5.8 ex} \begin{tabular}[t]{lll}
Handledare:& \handledare & \end{tabular}


\renewcommand{\footrulewidth}{0pt}

\fancyfoot[L]{\vspace{0.1mm}\large Institutionen för Matematiska vetenskaper\\
CHALMERS TEKNISKA HÖGSKOLA\\
GÖTEBORGS UNIVERSITET\\
Göteborg, Sverige 2021 }




\newpage
\thispagestyle{fancy}
\fancyhf{}
\newgeometry{top=3cm, bottom=3cm,left=3 cm,right=3cm}
\mbox{}
\vfill

\setcounter{page}{0}
\newpage }



\usepackage[T1]{fontenc}                % För svenska bokstäver
\usepackage[swedish,english]{babel}             % För svensk avstavning och svenska rubriker (t ex "Innehållsförteckning")
% Egna kommandon:
% Delar-operator:
\newcommand{\divides}{\mid}
\newcommand{\notdivides}{\nmid}
% Största gemensamma delare
\renewcommand{\gcd}[1]{(#1)}
% Kardinalitet
\newcommand{\card}[1]{\# #1}
% Heltalsmängd:
\newcommand{\A}{\mathcal{A}}
% Primtalsmängd:
\renewcommand{\P}{\mathcal{P}}
% Sållad mängd:
\renewcommand{\S}[3]{S(\mathcal{#1}, \mathcal{#2}, #3)}




% Satser etc. mer info hittas på https://www.overleaf.com/learn/latex/theorems_and_proofs
\newtheorem{theorem}{Sats}[section]
\newtheorem{corollary}{Korollarium}[theorem]
\newtheorem{lemma}[theorem]{Lemma}
\newtheorem{definition}[theorem]{Definition}
\newtheorem{proposition}{Proposition}[section]

% Block-kommentar: använd '\begin{comment} ... \end{comment}'

% Todo
\newcommand{\todo}[1]{\smallskip\noindent\big[To do: #1\big]\smallskip}



\begin{document}
\selectlanguage{swedish}
\omslag

\titelsidor
\thispagestyle{empty}
%\setlength{\textheight}{240mm}
%\addtolength{\topmargin}{-50mm}
\newgeometry{top=3cm, bottom=3cm,left=3 cm,right=3cm}

\section*{Förord}
Denna kandidatrapport har skrivits i syfte att introducera några grundläggande idéer inom sållteori och koppla dessa till nutidens framsteg inom ämnet.
Under arbetets utförande har en gruppdagbok, samt individuella loggböcker förts.
Dessa loggböcker innehåller detaljer angående utvecklingen av rapportens övergripande struktur, mötesanteckningar, och individuella rapporteringar av hur tiden har tillbringats.

Nedan listas huvudförfattare till rapportens respektive avsnitt:
\begin{itemize}
    \item \textbf{Nils Alexandersson}: 
        Populärvetenskaplig presentation,
        Bruns såll (\ref{brun}),
        inledningen till Datorimplementation av eratosthenes såll (\ref{partB}),
        Implementation av algoritmerna i Python (\ref{partB.implementering}),
        samt appendix B.3 och D.
    \item \textbf{Erik Dagobert}: 
        Förberedelser (\ref{forberedelser}), 
        Det allmänna sållproblemet (\ref{sallproblemet}), 
        Eratosthenes generaliserade såll (\ref{Eratosthenes}),
        Grundläggande teori och algoritmer (\ref{partB.algoritmteori}),
        samt appendix A, B.2 och C.
    \item \textbf{Coën Lorcan Olofsson}:
        Inledning (\ref{inledning}), 
        Selbergs såll (\ref{Selberg}),
        Diskussion av sållmetoder (\ref{Diskussion}),
        Tillämpningar och resultat (\ref{partB.applications}),
        samt appendix B.1, B.4 och B.5.
\end{itemize}
Vi vill också tacka våra handledare Anders Södergren och Lucile Devin för all den hjälp de har givit oss. 
Deras engagemang och vägledning har varit ovärderlig för detta arbete.

\newpage

\section*{Populärvetenskaplig presentation}
% Skrivet av Nils

Primtal är förrädiskt oförutsägbara.
De dyker upp när du minst anar det och kan till synes bete sig helt oregelbundet.
Trots detta är de helt deterministiska i sin natur och gränsen för vad som är och inte är ett primtal är mycket tydlig.
Det är kanske just av denna anledning som primtalen har fascinerat matematiker i tusentals år och fortsätter att göra så än idag.

Hur många primtal finns det? Svaret är att det finns oändligt många.
Om vi istället frågar oss hur många primtal det finns som är mindre än en miljon, så är svaret inte lika lätt.
Förutom talet 2 så är primtal aldrig jämna så vi kan åtminstone utesluta vartannat tal och säga att svaret är mindre än en halv miljon.
Hur går vi vidare härifrån?
Ett naturligt andra steg vore att föra samma resonemang för talet 3;
förutom just 3 så är primtal aldrig delbara med 3 så vi borde kunna dra bort ytterligare en tredjedels miljon från svaret.
Detta är dessvärre inte riktigt sant.
Tal som både är jämna och delbara med 3 har ju uteslutits två gånger.
Det har alltså skett en dubbelräkning av alla tal som kan delas med 6 men vi kan kompensera för detta genom att addera en sjättedels miljon till svaret.

\begin{comment}
Dubbelräkningen skedde eftersom vi hade två stycken mängder av tal som överlappade varandra och vi kompenserade genom att återlägga överlappet.

Denna idé kallas för \textit{inklusions-exklusionsprincipen} och kan även användas när vi har fler än bara två mängder.
Tag tre stycken mängder $A$,$B$ samt $C$, och låt överlappet mellan $A$ och $B$ betecknas med $AB$.
Om vi ska beskriva den sammanlagda mängden av $A$, $B$ och $C$ så måste vi 
Dessutom har vi $ABC$ där alla tre mängder överlappar vilken vi måste kompensera ytterligare för. Vi kan således beskriva den sammanlagda mängden av $A$, $B$ och $C$ som
\begin{equation*}
    (A+B+C) - (AB+BC+AC) + ABC
\end{equation*}
\end{comment}



Denna rapport utforskar några metoder som kan användas för att hitta svaret på frågor som ovanstående.
Oftast är det inte möjligt att få ett exakt svar och istället måste vi nöja oss med en uppskattning,
vilket givetvis leder till en följdfråga om hur bra uppskattningen är.



\begin{comment}
Dubbelräkningen skedde eftersom vi hade två stycken mängder av tal som överlappade varandra och vi kompenserade genom att titta på hur stort detta överlapp var.
Denna idé kallas för \textit{inklusions-exklusionsprincipen} och kan även användas när vi har fler än bara två mängder.
Säg nu att vi har tre stycken överlappande mängder som vi kallar för $A$,$B$ och $C$, och vi vill ta reda på den sammanlagda storleken av dem.
För att underlätta beräkningarna kan vi låta $\#\left( A\right)$ beteckna storleken av $A$ och göra sak för $B$ och $C$.
Det första vi kan göra är att addera de individuella storlekarna för $A$,$B$ och $C$, så att vi får 
\begin{align*}
    &\#\left( A\right)+\#\left( B\right)+\#\left( C\right).
\intertext{Som bekant har vi nu dubbelräknat mängdernas överlapp, vi drar ifrån dessa och får}
    &\#\left( A\right)+\#\left( B\right)+\#\left( C\right) - \#\left( AB\right)-\#\left( BC\right)-\#\left( AC\right),
\intertext{där $AB$ representerar överlappet mellan $A$ och $B$, och liknande för $BC$ samt $AC$.
Men vi är inte klara än, det finns nämligen ett ytterligare överlapp; $ABC$ där alla tre mängder överlappar.
Denna mängd trippelräknades i första steget och därefter har vi dragit bort den tre gånger om.
Vi måste därmed lägga till den igen;}
   &\#\left( A\right)+\#\left( B\right)+\#\left( C\right) - \#\left( AB\right)-\#\left( BC\right)-\#\left( AC\right)+\#\left( ABC\right),
\end{align*}
\end{comment}

\newpage
\begin{abstract}
Syftet med denna rapport är att ge läsaren en inblick i det matematiska delområdet sållteori genom att redogöra för dess grundläggande idéer och tillämpningar, samt att presentera en datorimplementation av algoritmen Eratosthenes såll.
%
Vi presenterar Eratosthenes generaliserade såll, samt Bruns och Selbergs såll en åt gången med följande tillvägagångssätt:
Först ges en kortfattad historisk kontext till sållet ifråga, följt av en översiktlig härledning och slutligen, ett exempel på hur sållet kan tillämpas för att ge resultat om bland annat primtalstvillingar och primtal i aritmetiska serier.
%
Efter att de tre sållen har presenterats diskuteras och jämförs orsaken till deras feltermer.
Detta med avsikt att ge läsaren en inblick i vilka möjligheter och begränsningar som finns i sållen som verktyg.
%





%Detta gör vi genom att först ge en mycket kortfattad historisk bakgrund till sållet ifråga

%Detta gör vi genom att kortfattat ge historisk kontext till  för att sedan koncentrera oss på att i stora drag härleda sållen.
%Efter detta visar vi hur sållen kan tillämpas för att ge resultat om bland annat primtalstvillingar och primtal i aritmetiska serier.
%

\end{abstract}

\selectlanguage{english}
\begin{abstract}
This report aims to introduce the reader to the mathematical area of sieve theory through the elucidation of its fundamental principles and applications, as well as presenting an implementation in code of Eratosthenes sieving algorithm. 
We present the generalised sieve of Eratosthenes, as well as the sieves of Brun and Selberg; briefly describing their history and then focusing on outlining their general derivation as well as giving a thorough example of their application on sets such as the twin primes and primes in arithmetic progressions. 
Following the presentation of the sieves is a discussion which aims to focus the reader's attention on the effects and origins of the error terms of the various methods and there effects they have on the quality of estimations which can be made.

Having presented the theory in some detail, we move our attention to an implementation of Eratosthenes original algoritm as based upon the work of Helfgott \cite{HaraldSieve}.
We present the underpinning mathematical principles and general algorithmic structure of Helfgott's implementation as well as our interpretation of his code and eventual improvements to the implementation.
Following that is a presentation of results regarding the distributions of the prime numbers, the twin primes, and the behaviour of prime gaps in the interval \(10^{19}\pm 1.25\times10^9\).
Through the comparison of our code with the prime number theorem as applied to this interval we bolster it's legitimacy, which leads to our investigation of the conjectured distribution of twin primes in our interval.
We conclude this report with an analysis of the frequency of the prime gaps in our interval and discuss briefly its connection to modern day advances in and about the theory.
\end{abstract}

\newpage
\selectlanguage{swedish}
\pagestyle{plain}
\tableofcontents                % Innehållsförteckning

                % Tabellförteckning

\newpage
\pagenumbering{arabic}

\section{Inledning} \label{inledning}
% \documentclass[a4]{article}
% \usepackage{graphicx}
% \usepackage{amsmath, amssymb}
% \usepackage{epsfig}
% \usepackage{floatflt} %för inkapslade bilder.

% \addtolength{\textwidth}{10mm}
% \addtolength{\textheight}{30mm}
% \addtolength{\headheight}{-10mm}

% \usepackage[T1]{fontenc}                % För svenska bokstäver
% \usepackage[swedish]{babel}             % För svensk avstavning och svenska
%                                         % rubriker (t ex "innehållsförteckning)

%Var noggranna med att ange källor till det ni skriver. Vi rekommenderar Vancouver-systemet\footnote{Även annat system accepteras om det används konsekvent.} som är mest använt på MV. Man kan antingen använda siffror [1], [2] etc, eller initialer som associerar till författarnamnet(n) t.ex [BN], [BS] etc. Det senare kan vara lite jobbigt om man har många källor men praktiskt om man har några som huvudreferenser. Läs mer i Fackspråks skrift: Utformning av rapporter och kandidatarbetens skriftliga .... (2008-01-11)\cite{rapp}. I det fall arbetet i huvudsak bygger på en eller ett par källor och det är svårt att identifiera exakt när man använder respektive källa, kan man tala om detta i inledningen. Man kan sedan referera till källan om man återger en definition, en sats eller ett bevis eller på annat sätt ligger nära källan. En direkt översättning kan jämställas med ett citat, återberätta därför som om ursprunget var en skrift på svenska så att ni håller er långt ifrån gränsen för plagiering. Är ni osäkra på något så fråga examinator eller handledare.\footnote{Läs mer om att hantera källor och akademisk hederlighet på Chalmers webbsida \hfill \\ URL: https://writing.chalmers.se/chalmers-skrivguide/att-hantera-kallor  }

\begin{comment}
Förutom att vara klassiska frågor inom talteori angående primtal, vad har primtalstvillingshypotesen, Goldbach hypotesen, och storleken på primtalsgap att göra med varandra? Kort sagt; sållteori. Lite längre sagt; under den senaste århundrade har tekniker inom sållteori utvecklades och tillämpades på dessa, och fler, blandade problem inom talteori, och även vidare ämnen, med relativt stort succé. Med hjälp av sållteori har matematiken lyckats bevisa att det finns oändligt många tal \textit{p} sådan att \textit{p} och \(p+2\) är antingen prima eller semiprima \cite{chen2Prime} och att det det finns alltid två primtal inom 600 heltal av varandra \cite{mayBound}. Dessa tillämpningar väcker frågan, vad är ett matematisk såll för något? 

Ett matematisk såll är en metod med sin ursprung i talteori som försöker att uppskata kardinaliteten av en så kallad siktad mängd där alla element i mängden har någon gemensam egenskap. Ett prototypisk exempel på en sådan mängd som man är intresserad i storleken av är mängden som består av alla primtal mindre än något tal \textit{x}. Sållteori har försökt att hitta och förfina uppskattningar av storleken på precis den mängden sedan Chebychev's berömd uppskattning i 1851. Dock har matematiska såll sin ursprung ännu längre sedan i antikens grekland med arbetet av Erastostenes. Hans idé följer nedanstående mönster;
\begin{enumerate}
    \item Med början vid 2, lista ut alla tal upp till talet du vill använda som övergräns.
    \item Rita en cirkel kring 2 och stryk över alla tal delbar med 2.
    \item Rita en cirkel kring det nästa talet som inte har en linje genom sig och dra en linje igenom alla tal delbar med den.
    \item Upprepa föregående steg tills alla tal på listan har antingen en cirkel kring de eller har en linje ritat över de.
\end{enumerate}
Då man har tillämpat klart Eratosthenes ursprunglig såll, blir alla tal med cirklar kring sig alla primtal som är mindre än övergränsen man valde. Även om det har gått nästan två årtusenden sedan upptäckten av Eratosthenes ursprunglig metod har metoden fortfarande samband med nutidens matematiska såll; samband som en fokus på delbarhet och, följaktligen, specifika modulo klasser av primtal. 

Det är dessa nya matematiska sålltekniker och deras tillämpningar på blandade problem samt numerisk implementering som vår rapport kommer att fokusera på. Vi kommer att hålla oss till att redovisa hur tre olika sållmetoder, nämligen Eratosthenes allmänna såll, Bruns såll, och Selbergs såll, kan tillämpas på XXX och YYY, samt kommer vi redovisa hur man kan implementera i kod ZZZs såll. Vi kommer dessutom att diskutera möjliga förbättringar av både av dessa undersökningar vi har gjort. Men innan vi kan börja med att redovisa vårt arbete kommer vi först att nämna några förkunskaper som en läsare borde ha, sedan kommer vi introducera några grundläggande begrepp och teori inom sållteori, och slutligen kommer vi att introducera mer fullständigt de såll vi har valt att jobba med.
\end{comment}

Den som någon gång har funderat kring primtal, kanske har provat att ta en lista med heltal och börjat markera de tal som är prima. Efter en liten stund kanske man märker att för att hitta alla primtal upp till ett visst tal behöver man bara alla primtal mindre än eller lika med kvadratroten av det talet. Man kanske också börjar lägga märke till mönster som uppträder; såsom att det finns vissa par av primtal som har bara ett tal mellan sig. 

Idén bakom denna process, att hitta primtal i en lista av naturliga tal, har funnits sedan antikens grekland med en algoritm som har tillskrivits den grekiska polyhistorn Eratosthenes (ca. 276 - 194 f.v.t.). Algoritmen har följande struktur;
\begin{enumerate}
    \item Börja med talet 2 och lista alla naturliga tal upp till någon gräns.
    \item Ringa in 2 och stryk över alla andra tal som är delbara med 2.
    \item Ringa in nästa tal som inte är struket och stryk alla andra tal delbara med det nya inringade talet.
    \item Upprepa föregående steg tills varje tal på listan är struket eller inringat. 
\end{enumerate}
När algoritmen är avslutad så har varje primtal i listan blivit inringat och alla andra tal har strukits. Eratosthenes algoritm la grunden för nutidens sållteori; ett område inom matematiken som försöker uppskatta storkleken på så kallade \textit{siktade mängder}. 

En siktad mängd är en mängd där alla element är heltal och har någon gemensam egenskap t.ex. en mängd som består av endast primtal eller mängden av alla heltalslösningar till en ekvation. Den grundläggande sållteorins största fördel är att den är relativt elementär och flexibel, speciellt jämfört med andra metoder inom analytisk talteori. Det krävs inga idéer från komplexanalys som t.ex. Dirichlets L-funktioner eller serier för att ha användning av de enklare sållen och om man kan formulera vikterna på ett korrekt sätt, går det att tillämpa metoderna på nästan vilken mängd som helst. Trots sina enkla konstruktion, kan dessa matematiska såll fortfarande ge starka resultat, även om noggrannhet måste offras lite. En exempel på detta är att asymptotiska beteendet av \(x/\log(x)\) för antalet primtal mindre än \textit{x} som ges i primtalssatsen. Detta kan nästan bevisas utan något arbete med zeta funktioner då man använder Selbergs såll, dock så får man bara en övre gräns av \(x/\log(x)\) istället för asymptotiska beteendet. Mer avancerade sållmetoder har givit svar på frågor närliggande till primtalstvillingshypotesen i \cite{chen2Prime}, och storleken på primtalsgap mellan ett antal primtal i rad i \cite{mayBound}.

Vår rapport kommer att fokusera på små, kombinatoriska såll. Små innebär att de fokuserar på att uppskatta storleken på ett litet antal restklasser modulo primtal, och att ett såll är kombinatorisk innebär att den använder sig av inklusion-exklusionsprincipen för att dela upp mängden på ett lämpligt sätt. Bland denna typ av såll håller vi oss till Eratosthenes allmänna såll, Bruns såll, och Selbergs såll. För varje såll kommer vi att ge en naturlig härledning till dess formulering och en förklaring till hur det används. Efteråt kommer vi att redovisa och analysera en implementering i kod av Eratosthenes algoritm, vilket följer metoden som beskrivs i \cite{HaraldSieve}. Dock, innan vi börjar redovisa någon av sållen vill vi gå igenom några förberedelser och förklara det allmänna sållproblemet.

\section{Förberedelser} \label{forberedelser}
% Skrivet av Erik

I den här texten kommer vi att låna större delen av vår notation från boken \textit{An Introduction to Sieve Methods and their Applications} av Alina Carmen Cojocaru och M. Ram Murty \cite{cojocarumurty}, exempelvis skriver vi \(p, q, \ell\) när vi menar primtal, \(n, d, k\) för naturliga tal och \(x, y, z\) för positivt reella tal. Den största gemensamma delaren betecknas här \(\gcd{d, k}\), funktionen \(\nu(d)\) beskriver antalet distinkta primtalsdelare av \(d\) och \(\pi(z)\) är antalet primtal mindre än eller lika med \(z\). Nedanstående avsnitt ämnar till att etablera mer notation och tekniker som är vanligt förekommande i sållteori med utgångspunkt i \cite{cojocarumurty}.

\subsection{Multiplikativa funktioner} \label{mult}
En särskilt trevlig delmängd av alla funktioner i talteoretiska sammanhang är de multiplikativa funktionerna. Vi säger att en funktion $f$ är \textit{multiplikativ} om $f(1) = 1$ och \(f(mn) = f(m)f(n)\) då $\gcd{m,n} = 1$. Om detta håller för alla $m, n$ så kallar vi $f$ för \textit{fullständigt multiplikativ}. 

Multiplikativa funktioner har fördelen att en viss typ av summor kan omskrivas till produkter, s.k. Eulerprodukter. Om \(f\) är multiplikativ så följer av aritmetikens fundamentalsats att
\begin{align*}
    \sum_{n = 1}^\infty f(n) = \prod_p \sum_{i=0}^\infty f(p^i)
\end{align*}
så länge den första summan konvergerar. 

\subsection{Möbiusfunktionen} \label{Mobius}
En ytterst väsentlig multiplikativ funktion i sållteori är Möbiusfunktionen,
\begin{equation*}
    \mu(n) = 
    \begin{cases}
        1, & \text{om}\ n \text{ är ett kvadratfritt, naturligt tal med jämnt antal primdelare}\\
        -1, & \text{om}\ n \text{ är ett kvadratfritt, naturligt tal med udda antal primdelare}\\
        0, & \text{om}\ n \text{ inte är kvadratfri}
    \end{cases}
\end{equation*}
som vi, bland annat, kommer se förenkla inklusion-exklusionsprincipen i nästa kapitel. Två andra viktiga egenskaper hos Möbiusfunktionen är följande egenskap,
\begin{equation*}
    \sum_{d \divides n} \mu(d) =
    \begin{cases}
        1, & \text{if}\ n = 1 \\
        0, & \text{if}\ n > 0
    \end{cases}
\end{equation*}
och Möbius inverteringsformel som säger
\begin{equation} \label{mobiusinv}
    f(n) = \sum_{d \divides n} g(d) \implies g(n) = \sum_{d \divides n} \mu(d) f(n/d)
\end{equation}
om \(f, g : \mathbb{N} \to \mathbb{C}\).

%\subsection{O-notation}
% Intresserade av en övre asymptotisk gräns --> förklara Ordonotation. 

%Matematisk sållteori är en underkategori av analytisk talteori. Ett av våra viktigaste analytiska verktyg i den här uppsatsen är Ordo-notationen. 

%\subsection{Abels summationsformel}

\section{Det allmänna sållproblemet} \label{sallproblemet}
Det allmänna fallet i sållteori är att vi har en mängd heltal \(\A\), en mängd primtal \(\P\), samt delmängder \(\A_p, \forall p \in \P\) och är intresserade av att sålla fram kardinaliteten \(S(\A, \P) :=\card{(\A \setminus \cup_{p \in \P} \A_p)}\). Vi låter \(P(z)\) beteckna produkten av alla \(p < z\) i \(\P\) med specialfallet \(P(z) = P_z\) då \(\P\) är mängden av alla primtal. Med denna notation så skriver vi 
\begin{align*}
    \S{A}{P}{z} := \card{(\A \setminus \cup_{p \divides P(z)} \A_p)} .
\end{align*}
För \(d\), en kvadratfri produkt av element \(p \in \P\) definierar vi \(\A_d = \cap_{p \divides d} \A_p\) och \(\A_1 = \A\). Använder vi oss av inklusion-exklusionsprincipen, formulerad med hjälp av möbiusfunktionen, får vi då
\begin{align} \label{inclusionexclusion}
    \S{A}{P}{z} = \sum_{d \divides P(z)} \mu(d) \card{\A_d} .
\end{align} % sida 72 

Ett gemensamt drag hos de matematiska sållen diskuterade nedan är att låta \(X\) beteckna kardinaliteten av \(\A\) och hitta \(\delta_p\) så att
\begin{align} \label{deltaX}
    \card{\A_p} = \delta_p X + R_p
\end{align}
där \(\delta_p \in [0, 1)\) kan betraktas som en uppskattning av andelen element i delmängden \(\A_p\) och \(R_p\) som en felterm. % För \(p \neq q\) låter vi \(\delta_{pq} = \delta_p \delta_q\) så att \(\card{\A_{pq}} = \delta_p \delta_q X + R_{pq}\) och utvidgat för ett kvadratfritt tal \(d = p_1 \cdot ... \cdot p_n\) låter vi \(\delta_{d} = \delta_{p_1} \cdot ... \cdot \delta_{p_n}\). 

I avsnitt \ref{Eratosthenes} om Eratosthenes generaliserade såll och avsnitt \ref{brun} om Bruns såll så kommer vi välja \(\delta_p = \omega(p) / p\) där \(\omega(p)\) betecknar antalet utvalda restklasser modulo \(p\) vi vill sålla bort. Som följd låter vi i dessa fall \(\A_p\) vara alla element som tillhör någon av dessa restklasser modulo $p$ och för kvadratfria $d$ låter vi \(\omega(d) := \prod_{p \divides d} \omega(p)\). Därtill definierar vi den användbara funktionen 
\begin{align} \label{Wfunc}
    W(z) := \prod_{\substack{p \in \P \\ p < z}}\left(1 - \frac{\omega(p)}{p}\right).
\end{align} 
Vi kan se att produkten ovan genererar en summa av alla kvadratfria produkter av \(p \in \P\) som är multiplicerat med $-1$ om vi inkluderat ett udda antal primtal. Med Möbiusfunktionen kan detta omformuleras till \(\sum_{d \divides P(z)} \mu(d) \omega(d)/d\).

% Alt. säga omega(d) helt multiplikativ

\section{Eratosthenes generaliserade såll} \label{Eratosthenes}
% Skrivet av Erik

Det klassiska exemplet på ett matematiskt såll, vilket beskrevs i introduktionen, har tillskrivits den grekiska polyhistorn Eratosthenes (ca. 276 - 194 f.v.t.). Idén var raffinerad av Adrien-Marie Legendre 1808 v.t. med hjälp av inklusion-exklusionsprincipen formulerad på formen från avsnitt \ref{sallproblemet}. Nedan kommer vi se hur \cite{cojocarumurty} utvecklar Eratosthenes såll genom användningen av ett trick uppkallat efter matematikern R. A. Rankin och sedan studera en tillämpning av resultatet. 

\subsection{Legendres såll}

Första steget för att utveckla Eratosthenes såll är att omformulera sållexemplet från introduktionen i de termer vi definierade i föregående avsnittet. I exemplet börjar vi med en lista av alla naturliga tal upp till en övre gräns, säg $x$, vilken vi kan skriva som $\A = \{n \in \mathbb{N} : n \leq x\}$. Låt alla primtal upp till $z$ redan vara inringade och mängderna vi kryssar över vara \(\A_p = \{a \in \A :  a \equiv 0 \pmod{p}\}\). Om vi väljer $z = \sqrt{x}$ så blir, med hjälp av (\ref{inclusionexclusion}),
\begin{align*}
    \pi(x) - \pi(\sqrt{x}) + 1 = \sum_{d \divides \P(\sqrt{x})} \mu(d) \left\lfloor \frac{x}{d} \right\rfloor  
\end{align*}
där \(1\):an på vänstersidan tar i hänsyn att \(1 \in \A\) inte sållas bort på högersidan och \(\pi(\sqrt{x})\) är de inringade primtalen. Detta var Legendres idé 1808 när han omformulerade Eratosthenes såll till att räkna primtal 1808 \cite{opera}. 

Eftersom \(\A_p\), för varje $p$, är definierad som en restklass modulo $p$ så säger vi att antalet utvalda restklasser modulo $p$ är $\omega(p) = 1$ för alla $p$ - vi kallar Eratosthenes för ett endimensionellt eller linjärt såll av den här anledningen. Mer allmänt betecknas dimensionen av ett såll med parametern \(\kappa\) om 
\begin{align}
    \sum_{p \divides P(z)} \frac{\omega(p) \log(p)}{p} \leq \kappa \log(z) + O(1)
\end{align}
\todo{Alt. mer lös definition: omega(p) genomsnittligt begränsad av kappa \cite{tenenbaum}}. Vi ser således att ett naturligt nästa steg är att generalisera Eratosthenes-Legendres såll för godtyckliga dimensioner. 

\subsection{Eratosthenes generaliserade såll}

Om vi låter $\A$, $\P$, $P(z)$, $\S{A}{P}{z}$ och $\omega(d)$ vara definierade som i avsnitt \ref{sallproblemet} 

\begin{theorem}[Eratosthenes generaliserade såll]\label{thm:EratosthenesSieve}

\begin{align*}
    \S{A}{P}{z} = X W(z) + O\left(\left(X + \frac{y}{\log z} \right) (\log z)^{\kappa + 1} \exp{\left(-\frac{\log y}{\log z}\right)} \right)
\end{align*}

\end{theorem}


\section{Bruns såll} \label{brun}
% Skrivet av Nils

År 1915 presenterade norrmanen Viggo Brun (1885-1978) ett såll som senare fått namnet Bruns såll. Brun hade inspirerats av publikationer av Jacques Hadamard som handlade om de framsteg som Jean Merlin gjort i att utöka Eratosthenes teori. Sållet blev inte uppmärksammat direkt, istället tog det ca 30 år innan andra matematiker började intressera sig för det. En anledning till dröjsmålet kan vara att Bruns skrivsätt och val av notation gjorde materialet onödigt svårläst. Sedan det att sållet publicerades har det gjorts flera insatser för att förkorta och förenkla läsningen men trots detta är Bruns såll och dess bevis krävande för läsaren, med detta i åtanke har vi valt att korta ner beviset avsevärt i förhoppning om att stycket ska kännas överkomligt.





\section{Selbergs såll} \label{Selberg}
Kring tjugo år efter utvecklingen av succén som var Bruns såll, uppkom två olika såll metoder självständigt av varandra; Linniks Stor Såll och Selbergs Såll. Av dessa har vi valt att fokusera på Selbergs såll, upptäckt av norsk matematiker Atle Selberg, som är av samma kombinatorisk stil som de tidigare såll av Legendre och Brun. Fast till skillnad med de tidigare sållmetoder, valde Selberg att använda sig av en approximation av Möbius funktionen istället för att använda den tidigare funktionen \(W(z)\). Denna approximation tog formen av en kvadratisk form med den centrala idén att for en följd av reella tal \((\lambda_d)\) med \(\lambda_1 = 1\), har vi för alla \textit{k} att 
\begin{equation}
    \sum_{d\divides k} \mu(d) \leq \Bigg(\sum_{d\divides k}\lambda_d\Bigg)^2.\label{eq:SelAppr}
\end{equation}
Nu gäller att göra en bra val på \(\lambda_d\) för att minimera felet med approximationen. 

För att se hur \eqref{eq:SelAppr} är användbar, låt oss nu vända till formuleringen av \(\S{A}{P}{z}\), nämligen
\begin{align}
    \S{A}{P}{z} = \sum_{\substack{a\in \mathcal{A}\\a\not\in\mathcal{A}_p,\forall p\divides P(z)}} 1 = \sum_{d\divides P(z)} \mu(d) \sum_{a\in\mathcal{A}_d}1 = \sum_{a\in A} \Bigg(\sum_{\substack{d\divides P(z)\\a\in A_d}}\mu(d)\Bigg).\label{eq:sellMe}
\end{align}
Här i \eqref{eq:sellMe} kan vi använda oss av uppskattningen i \eqref{eq:SelAppr} med utvidgningen att \(\lambda_d=0\) då \(d>z\), och då får vi att
\begin{equation}
    \eqref{eq:sellMe} \leq \sum_{a\in A}\Bigg(\sum_{\substack{d\divides P(z)\\a\in A_d}}\lambda_d\Bigg)^2 = \sum_{a\in A}\Bigg(\sum_{\substack{d_1,d_2\divides P(z)\\a\in A_{[d_1, d_2]}}}\lambda_{d_1}\lambda_{d_2}\Bigg) =  \sum_{d_1,d_2\leq z}\lambda_{d_1}\lambda_{d_2}\#A_{[d_1, d_2]}\label{eq:2FatLs}
\end{equation}
där \([a, b]\) betecknar lägst gemensam multipel av \textit{a} och \textit{b}. Om vi nu skulle använda oss av en sådan relation som \eqref{deltaX}, då har vi att 
\begin{equation}
    \S{A}{P}{z} \leq X\sum_{\substack{d_1,d_2\leq z\\d_1,d_2\divides P(z)}}\delta_{[d_1, d_2]}\lambda_{d_1}\lambda_{d_2} + O\Bigg(\sum_{\substack{d_1,d_2\leq z\\d_1,d_2\divides P(z)}}|\lambda_{d_1}||\lambda_{d_2}||R_{[d_1, d_2]}|\Bigg)\label{eq:SelbergStart}
\end{equation}
där första termen anses vara huvud uppskattningen och den andra en felterm. \eqref{eq:SelbergStart} utgör en startpunkt för Selbergs såll, där vi nu måste bestämma en vikt \(\delta_p\) och försök att minimera storleken av \(\lambda_d\).
\subsection{Huvudtermen}
Låt oss börja med att bestämma en form på \(\delta_d\), och med det målet följer vi Selberg och bestämmer \(\delta_d = 1/f(d)\) för någon multiplikativ funktion \textit{f}. För denna funktion gäller det att \(f(n) = \sum_{d\divides n}f_1(d)\) för någon multiplikativ funktion \(f_1\), som är entydigt bestämd av Möbius inverteringsformeln \(f_1(n) = \sum_{d\divides n}\mu(d)f(n/d)\). Det är naturligt att betrakta \(f(d)\) som en typ av utveckling av \(\omega(d)/d\) idén från de tidigare såll som nu försöker att underlätta mer komplexa uppdelningar av mängden än bara att fokusera på särskilda ekvivalensklasser av \(p\).

Med en bestämd \(\delta_d\) och hjälpen av en lemma angående multiplikativa funktioner, se Appendix B, kan vi nu börja hantera huvudtermen. Insättningen av vår ny \(\delta_d\) och användningen av lemman ger att
\begin{equation}
    \sum_{\substack{d_1,d_2\leq z\\d_1,d_2\divides P(z)}}\frac {\lambda_{d_1}\lambda_{d_2}}{f([d_1, d_2])} = \sum_{\substack{d_1,d_2\leq z\\d_1,d_2\divides P(z)}}\frac {\lambda_{d_1}\lambda_{d_2}}{f(d_1)f(d_2)}\sum_{\delta\divides (d_1,d_2)}f_1(\delta)=\sum_{\substack{\delta\leq z\\\delta\divides P(z)}}f_1(\delta)\Bigg(\sum_{\substack{d\leq z\\d\divides P(z)\\\delta \divides d}}\frac{\lambda_d}{f(d)}\Bigg)^2\label{eq:diagMe}
\end{equation}
och om vi sätter
\begin{equation}
    u_\delta = \sum_{\substack{d\leq z\\d\divides P(z)\\\delta \divides d}}\frac{\lambda_d}{f(d)}\nonumber
\end{equation}
då får vi den diagonaliserade kvadratformen
\begin{equation}
    \eqref{eq:diagMe} = \sum_{\substack{\delta\leq z\\\delta\divides P(z)}}f_1(\delta) u_\delta^2\label{eq:minimiseMe}
\end{equation}
Vi får även från den Möbius dubbla inverteringsformel att
\begin{equation}
    \frac{\lambda_\delta}{f(\delta)} = \sum_{\substack{d\divides P(z)\\\delta \divides d}}\mu(\frac{d}{\delta})u_d \implies
    \begin{cases}
    u_\delta = 0,\quad\delta \geq z\\
    \sum_{\substack{d\leq z\\d\divides P(z)}} \mu(d) u_d = 1
    \end{cases}\nonumber
\end{equation}


\section{Diskussion av sållmetoder}\label{Diskussion}
%As mentioned in the introduction, sieve methods, while useful, often require quite a bit of work or refinement in order to achieve the results given by more analytic methods. The central reason for this, as the reader may have noticed throguh the course of this report, is due to the error terms attached to the methods. It is a recurring theme of sieve theory in general to attempt to better these error terms in order to minimise the apprroximations which have to be made. The reader need only recall the work done in each of the sections regarding the applications of the sieves where one of the final steps  was to find a z which made the main and error term have equal weight. Out of the sieve methods we have presented the starkest difference in size of error term is between the generalised sieve of eratosthenes and the other two sieves. To see this difference we present one final simple example.

Som nämndes i inledningen kan sållmetoder, även om de är användbara, kräva en stor del arbete eller förfining för att spegla de resultat vilka ges av mer analytiska metoder. 
En uppenbar orsak till  detta är feltermerna kopplade till de sållmetoder som används. 
Det är ett återkommande tema i sållmetoder som helhet att försöka minska dessa feltermer för att minimera de slutliga uppskattningarna.
Feltermens påverkan på dessa uppskattningar kan enkelt upptäckas i alla tillämpningsdelar i de föregående avsnitten.
Den har störst påverkan i det kritiska steget då \textit{z} bestäms till någon funktion av \textit{x} vilket i sin tur förenar huvudtermen och feltermen.
Av de sållmetoderna vi har redovisat är skillnaden mellan feltermerna störst mellan Eratosthenes generaliserade såll och de andra två sållmetoderna.
För att visa påverkan av denna skillnad redovisar vi ett sista exempel.

%Consider the problem of estimate the number of primes less than x, one fo the original muses for much of this theory. Assuming for brevities sake that all the conditions are upheld we see that this is a linear sieving of all n less than x by the n equiv 0 mod p. Choosing y = x, as there is no need to sieve by nubers larger than x, and kappa = 1 gives the error term to be of the order O(x(log z)^2exp(-log x/logz)).Naturally this erro term is quite large indeed being even directly dependent on the lenght of the interval we are sieving as well as how many primes we are using. For comparison, using Brun's sieve with b=1, we find that a suitable lambda  to be labda = ln(1.31) which gives the power of the z in the error term to be less than or equal to 9. Furtherore using Selberg's sieve with #A = ... we find that the error term is once again of the order of a power of z, this time O(z^2). The difference between the erros of the sieves of Brun and Selberg can be quite easily seen however how they compare to that of Eratosthenes is not as obvious. The size of the main terms given by each of these sieves is the same, namely x /log z, which is unsurpising when you consider that all the sieves seeek to estimate the same cardinality, and choosing z = x^1/u in both Bruns and Selbergs sieves, with u > 10 and u > 2 respectively, leads to results whioh reflect are of the order of << x/log x, with the main difference being the size of the implied constant. However if we were to make the same choice of z for the sieve of eratosthenes we would run into an issue where the error term has more weight than the main term and as such must choose a different z, namely log z = logx/Cloglogx for some small positive constant C. This choice of z returns an estimate of pi(x) of the order of << xloglogx/logx, which is naturally worse than those given by the other sieves.  This basic example shows the effects of the error terms on the quality of estimates which can be made using the sieve methods. This then naturally leads to the question of what causes these errors in approximation and how can we go about reducing them?

Låt oss nu återgå till en av sållteorins musor som är att uppskatta storleken på \(\pi(x)\).
Som nämndes i \ref{era.Legendres} motsvarar problemet en linjär sållning av \(\A = \{n\leq x: n\in\mathbb{N}\}\) med \(\A_d = \{n\leq x: n \equiv 0 \pmod{d}\}\). Då vi tillämpar Eratosthenes såll på problemet där vi väljer \(y = x\) eftersom vi behöver sålla bara med tal mindre än \textit{x}, och \(\kappa = 1\) eftersom vi betraktar bara ett restklass av alla primtal. 
Med detta val av parametrar erhåller vi en felterm av storleksordning \(O(x(\log z)^2\exp({-\log x/\log z}))\).
Vi jämför den nu emot feltemerna för både Bruns och Selbergs såll, och vad för uppskattning av \(\pi(x)\) man kan få  från de.
Vi tillämpar Bruns såll genom att sätta \(b = 1\) och, återigen, \(\kappa = 1\). 
Vi också sätter \(\lambda = 0.26\), för att minska potensen av feltermen och då får vi att \(c_3 \leq 9\) och storleksordningen på feltermen är \(O(z^9)\).
För att tillämpa Selbergs såll inser vi att \(\#\A_d = x/d + O(1)\) vilket medför att feltermen i Selbergs såll av storleksordning \(O(z^2)\), precis som i tillämpningen i avsnitt \ref{sel.apl}.
Skillnaden mellan felen i Bruns och Selbergs såll kan ses lätt men hur de jämförs med Eratosthenes är inte lika uppenbart.
För att förstå skillnaden blandar vi in huvudtermen, vilket är av storleksordning \(O(x/\log z)\) för alla tre sållmetoder.
Vi kan förena huvud och feltermen för både Bruns och Selbergs såll genom att sätta \(z = x^{1/u}\), med \(u > 9\) respektive \(u > 2\) och både ger uppskattningen \(\pi(x) \ll x/\log x\). 
Dock så kan vi inte välja \textit{z} på samma stil när vi tillämpar Eratosthenes allmänna såll.
I det fallet måste vi välja \(\log z = \log x/C\log\log x\) för något liten konstant \textit{C} sådan att feltermen och huvudtermen har samma vikt.
Med en sådan \textit{z} erhåller vi från Eratosthenes allmänna såll att \(\pi(x) \ll x\log\log x/\log x\).
Att både Brun och Selbergs såll når resultat som börjar likna primtalssatsen men inte Eratosthenes generaliserade såll  exemplifierar hur feltermerna påverkar kvalitén av slutsatsen man kan dra med de olika metoderna.

%Each of the sieve methods have various different factors which influence the size of their error terms. However, there is one commnon factor between the three sieves which is worth mentioning, that of how the methods use the Moebius function to facilitate estimations cardinalities. We have as a baseline the generalised sieve of earatosthenes where the almost no use is made of the Moeebius function to reduce the error terms size, given that the funciton is directly estimated with \mu(d)\ leq 1. In Bruns sieve however, the first equality in chapter 5 has its roots in a clever use of Moebius inversion alongside the Moebius function to allow the construction of that equality, and therefore the subsequent estimates. Selbergs ssieve makse the greatest use of the Moeebius function with its direct estimation of a sum of Moebius functions as outlined in chapter 6. It is telling of the difficulty of the Moebius function that the sieve method which attempts to remove its dependence on the Moebius function is the method which often gives the best result. Indeed, according to Tao, the Moebius function is the root of a even more central concern of  sieve theory than that of minimising the error terms, namely the parity problem. this problem in esssence states that for specific types of sets sieve methods will either gives over bounds which are too large my a factor of at least two, or under bounds which are trivia. This can be seen to be due to the binary nature of the moebius function where truncating the function leads to too many elemetns being removed or too many ebing added back. Modern sieve methods such as those developed by Friedland and Iwaniec have attempted to be "parity sensitive" for certain sets of primes and has met success in showing that there are infinitely many primes of the form a^2 + b^4.

Var och en av sållmetoderna har olika faktorer som påverkar storleken på deras feltermer.
Det finns dock åtminstone en gemensam faktor mellan de tre metoderna som är värt att nämna, vilket är hur de olika metoderna hanterar Möbiusfunktionen.
Som baslinje har vi Eratosthenes allmänna såll som uppskattar Möbiusfunktionen med \(|\mu(d)| \leq 1\) då vi kommer till feltermen.
Bruns såll  gör lite mer med Möbiusfunktionens och \eqref{brun.eq.firstsum} kan formuleras på ett sådant sätt på grund av Möbiusfunktionen och en tillämpning av Möbius inverteringsformeln på den och funktionen \textit{g}.
Selbergs såll går ett steg vidare och baserar en stor del av hela sin härledning på en uppskattning av en summa av Möbiusfunktioner.
Att sållet som försöker hårdast att undvika Möbiusfunktionen är oftast metoden som ger bäst resultat av de tre påpekar hur svårhanterat Möbiusfunktionen är.
I själva verket, enligt Tao \cite{Tao}, är Möbiusfunktionen roten till ett ännu mer centralt utmaning för sållteorin än att minimera feltermer, nämligen paritetsproblemet.
Detta problem säger i huvudsak att för specifika typer av mängder kommer sållmetoder antingen att ge övre gränser som är för stora med faktor av minst två eller undre gränser som är triviala.
Detta kan ses bero på den binära karaktären hos Möbiusfunktionen vilket leder till att för många kardinaliteter tas bort eller för många läggs tillbaka då man trunkerar Möbiusfunktionen, det vill säga uppskattar den.
Moderna sållmetoder såsom de utvecklade av Friedlander och Iwaniec \cite{abPrimes} försöker undvika paritetsproblemet för vissa mängder av primtal.
Deras metoder har haft succé med att visa att det finns oändlig många primtal på formen \(a^2+b^4\) och även ger en asymptotisk formel för det.


%Other challenges sieve theory has met under its development include the sieveing by mor than a few number of residue classes of prime numbers. The sieves we have presented in the report work best when a small number of residue classes are to be sieved out however advances in the theory have allowed up to half the total of residue classes to be removed. The large sieve as it was called was published around the same time as Selbergs work by Linnik, and uses very different mathematical tools to allow the sieving by up to half of the total number of prime numbers. This sieve has met great succes in showing the Bombieri/Vinogradox theorem, a cornerstone to modern daay sieve theory regarding the distirbution of primes in arithmetic sequences averaged over a range of moduli. Work by Gallagher later simplified much of the sieve and extended it to sieve by arbitrarily many residue calsses.


\section{Datorimplementation av Eratosthenes såll} \label{partB}
% Skrivet av Nils

I numeriska sammanhang är Eratosthenes såll ett kraftfullt verktyg.
Sållet ger oss en metod för att sålla bland eller faktorisera alla tal upp till något $N$,
där förhållandet mellan $N$ och antalat beräkningar som krävs är nära linjärt \cite[s.333]{HaraldSieve}.
I detta avsnitt utforskas de idéer och algoritmer baserade på Eratosthenes såll som presenteras i \textit{An Improved Sieve of Eratosthenes} av Harald Helfgott \cite{HaraldSieve}, samt en datorimplementation av detta skriven i programmeringsspråket Python. 
Avsnittet består av tre delar.
I den första delen beskrivs algoritmernas funktion och deras underliggande matematiska principer.
Del två är en metoddel där vi beskriver vilka beslut som gått in i att skriva ett snabbt och välfungerande program baserat på dessa algoritmer.
Slutligen visar vi hur programmet kan användas för att ge resultat om fördelningen av primtalstvillingar och frekvensen av primtalsgap.

    \subsection{Grundläggande teori och algoritmer} \label{partB.algoritmteori}
    \todo{Förklara övergripande struktur för NewSegSiev och underordnade funktioner.}

\todo{Ge intuition för den bakomliggande teorin i NewSegSiev. Geometrisk tolkning?}

    \subsection{Implementation av algoritmerna i Python}\label{partB.implementering}
    % Skrivet av Nils

Algoritmerna i \cite{HaraldSieve} presenteras i form av pseudokod som för att kunna användas, måste översättas till något programmeringsspråk.
Vår implementation av algoritmerna är skrivna i språket Python.
%Vi har valt att använda språket Python för att implementera algoritmerna.
Det var möjligt att översätta pseudokoden mer eller mindre ordagrant, vilket gjordes och resulterade i en första version av programmet.
Därefter kunde flera förbättringar av koden göras för att korta ned dess körningstid. 
Vissa av förbättringarna var möjliga då pseudokoden i \cite{HaraldSieve} är skriven i syfte att tydligt illustrera algoritmerna,
och är således inte ämnad till att vara färdig, optimerad kod.
Andra förbättringar var språkspecifika och åstadkoms genom att jämföra beräkningstiden hos olika funktioner och metoder i Python, för att sedan implementera de som visade sig vara snabba.
Den förbättrade versionen av koden finns bifogad i appendix.
Nedan följer, utan inbördes ordning, några av de gjorda förbättringar som har haft större inverkan.
\begin{myitemize}
    \item
    I den ursprungliga pseudokoden representeras den sållade mängden av en vektor bestående av booleaner, vilken vi har ersatt med en bitsträng.
    %Algoritmen sållar primtal ur en lista som i det ursprungliga programmet representeras av en vektor bestående av booleaner.%  Det ursprungliga programmet sållar över en vektor av booleaner, vilken vi har valt att ersätta med en bitsträng.   
    %Istället för att spara och göra beräkningar på en vektor av booleaner, väljer vi att uttrycka mängden som en bitsträng. 
    Denna idé föreslås redan i \cite{HaraldSieve} och sparar i första hand minne,
    som i sin tur kan leda till snabbare beräkningar på grund av bättre användning av cache.
    Här användes Python-biblioteket \textit{Bitarray}.
    \item
    På vissa ställen har det varit möjligt att flytta ut beräkningar utanför loopar så att samma beräkning inte behöver göras flera gånger. 
    Dessutom har flera beräkningar kunnat kortas ned eller skrivas ihop för att undvika temporära variabler.
    \item
    I Python kan operatorn \texttt{x//y} användas för division utan rest. Denna har visat sig vara snabbare än sammansättningen \texttt{floor(x/y)} och har således fått ersätta den senare där det varit möjligt. På ett liknande sätt har \texttt{ceil(x/y)} ersatts med \texttt{-(x//-y)}.
\end{myitemize}

En ytterligare förbättring kunde göras efter en noggrann analys av pseudokoden för \textsc{NewSegSiev} i \cite[s.338]{HaraldSieve}.
Efter en approximering erhålls $n'\geq0$, en multipel av $m>0$ som möjligen uppfyller $n'\in[n-\Delta,n+\Delta]$ och $n'>m$.
Enligt \cite{HaraldSieve} ska dessa tre villkor testas en efter en,
men i själva verket är $n'\leq n+\Delta$ alltid uppfyllt eftersom $n'$ definieras som
\begin{equation} \label{helf.nprim}
    n' := \left\lfloor \frac{n+\Delta}{m} \right\rfloor\cdot m,
\end{equation}
vilken givetvis är mindre än eller lika med $\frac{n+\Delta}{m}\cdot m = n+\Delta$. 
Vidare inspekterar vi villkoret $n'>m$, där (\ref{helf.nprim}) ger oss att 
\begin{equation*}
    n'>m \iff
    \left\lfloor \frac{n+\Delta}{m} \right\rfloor > 1 \iff
    \left\lfloor \frac{n+\Delta}{m} \right\rfloor \geq 2 \iff
    (n+\Delta)/2\geq m.
\end{equation*}
Men $m$ väljs ur intervallet $[M,M+2R]$ så det räcker med att undersöka vilka förhållanden som  $M+2R\leq(n+\Delta)/2$ är uppfyllt under:
Vi har att $M\leq\sqrt{n+\Delta}$, $R=\lfloor M\sqrt{\Delta/4n}\rfloor$ samt $\Delta\leq n$ och  
således håller $M+2R \leq 2\sqrt{n+\Delta}$.
Detta är i sin tur mindre eller lika med $(n+\Delta)/2$ så länge $n+\Delta\geq16$.
Att sålla efter primtal i en lista av positiva heltal mindre än $16$ är ointressant.
Därför kan vi rimligtvis introducera kravet $n+\Delta\geq16$ i början av \textsc{NewSegSiev},
så att $n'>m$ alltid håller och därmed inte behöver testas senare i algoritmen.

Det har följaktligen visat sig att utav de tre villkor som skulle testas, 
så är två av dem alltid uppfyllda och behöver därmed inte prövas.
Det totala antalet gånger som detta steg utförs är
\begin{equation*}
    O\left(\Delta\log n + \sqrt{n/\Delta}(\log n)^2\right),
\end{equation*}
enligt \cite[s.346]{HaraldSieve} och vi sparar således en betydande mängd tid på att reducera antalet beräkningar här till en tredjedel av det ursprungliga.


För att visa att de förändringar som gjorts i koden, faktiskt har haft inverkan så lät vi utföra tester.
Testerna gjordes på en hemdator och jämförde körningstid mellan den första versionen av programmet, mot en senare version där förbättringarna har införts.
Som tidigare nämnts så sållar algoritmen fram primtal i ett angivet intervall $[n-\Delta,n+\Delta]$ och 
dess tillvägagångssätt varierar något beroende på förhållandet mellan $\Delta$ och $n$.
Av denna anledning utfördes två stycken tester där detta förhållande sattes till att vara så stort som möjligt respektive så litet som möjligt.
I det första testet valdes således $\Delta:=n$ och tre mätningar utfördes för $n=5\cdot10^3,5\cdot10^5,5\cdot10^7$.
I det andra testet valdes $\Delta:=\sqrt[3]{n}$ och $n$ fick anta värdena $10^9$, $10^{12}$ respektive $10^{15}$.
För alla mätningar visade sig den förbättrade versionen vara minst 10 gånger så snabb som den ursprungliga versionen, dessutom visar mätningarna på en antydan till att denna faktor ökar då $n$ växer. De uppmätta tiderna finnes i tabell \ref{implementering.tidtabell}.


\begin{table}[H]
\centering
\caption{
Uppmätta körningstider för den första, respektive den förbättrade versionen av programmet.
Programmet sållade fram alla primtal i ett intervall på formen $[n-\Delta,n+\Delta]$ där $\Delta=n$ för de tre första mätningarna och $\Delta=\sqrt[3]{n}$ för de tre sista.
Den förbättrade versionen var snabbare än den första versionen med en faktor på minst 10, för alla mätningar.}
\renewcommand{\arraystretch}{1.3} % Changes height of each row

\begin{table}[H]
\begin{center}
\begin{tabular}{|c||r|r|}
    \hline
    \multirow{2}{*}{\ Intervall\ } & \multicolumn{2}{c|}{Körningstid (s)}\\
    \cline{2-3}
    & Utan förbättringar & Med förbättringar\\
    \hline
    $\left[0,10^{6}\right]$ & 0.00534 & 0.000534\\
    $\left[0,10^{9}\right]$ & 0.00534 & 0.000534\\
    $10^{9}\pm 10^{3}$ & 0.00534 & 0.000534\\
    $10^{12}\pm 10^{4}$ & 0.00534 & 0.000534\\
    \hline
\end{tabular}
\end{center}
\label{implementering.tidtabell}
\caption{Körningstider för den första versionen, respektive den förbättrade versionen av programmet. Intervallen har valts till...}
\end{table}

\label{implementering.tidtabell}
\end{table}


Givetvis är programmet inte perfekt och det finns flera knep som kan utforskas ifall ytterligare förbättring av koden eftertraktas.
I den ursprungliga artikeln \cite{HaraldSieve} nämns ett fåtal eventuella förbättringar, här ges ytterligare två stycken.
\begin{myitemize}
    \item
    Istället för att flera gånger om låta \textsc{SimpleSiev(M)} generera en lista med alla primtal upp till $M$ kan det eventuellt spara tid att generera en godtycklig lista vid start av programmet.
    Värdet på $M$ överskrider inte $\sqrt{K\Delta+\sqrt{K\Delta}}$, vilket gör denna taktik rimlig.
    Exempelvis tar en lista med alla primtal upp till en miljard omkring 1GB att spara och kan användas för $n\leq 10^{35}$, så länge som vi håller oss till något mindre interval där $\Delta\leq\sqrt{n}$.
    Detta handlar självklart om en avvägning mellan hur snabbt det går att läsa in sparad data mot hur snabbt det går att beräkna den från grunden och bör undersökas mer innan idén tillämpas.
    \item
    Flera beräkningar kan utföras parallellt, vilket redan föreslås i \cite{HaraldSieve}. 
    I synnerhet kan detta nyttjas i andra delen av \textsc{NewSegSiev} där oberoende beräkningar utförs för varje $j\in[-k-1,k+1]$.
    Dessa beräkningar är alla enkla och på samma form, och det kan således vara gynnsamt att överlåta dessa till datorns grafikkort.
    Detta eftersom grafikkortet ofta visar sig snabbare än processorn på att utföra denna typ av uppgifter.
\end{myitemize}

I nästkommande avsnitt presenteras diverse resultat som erhållits ifrån körning av programmet.

\begin{comment}
    \item
    Deklaration av temporära variabler har i vissa fall kunnat uteslutas genom sammanskrivning av flera uttryck. 
    Ett specialfall av detta nyttjas i \textsc{DiophAppr} där vi har ersatt uttryck på formen \texttt{temp=x; x=y; y=temp;} med det snabbare \texttt{x,y=y,x;}.

    \item Flera while-loopar har kunnat ersättas med for-loopar,
    som ger att istället för att testa ett argument för varje iteration i loopen,
    behöver argumentet bara testas en gång när loopen påbörjas.
    \item Infogande av iteratorer vid iterering över primtalslistor, vilket också resulterar i bättre nyttjande av cache.

    \item
    I \textsc{DiophAppr} beräknas både heltals- och decimaldelen av $\alpha$. Detta görs i nuläget separat men skulle kunna göras samtidigt.
    Förslagsvis skulle då if-satsen ändras till att testa ifall decimaldelen är noll.
\end{comment}

    \subsection{Tillämpningar och resultat} \label{partB.applications}
    %Over the course of this report we have refered to, and approximated, the distribution of variouss classifications of prime numbers. Using now our implementation of Helfgott's origianl code we seek to present the legitimacy of said distributions and also reflect on the quality of the code which we have written. The specific distributions to be presented are those of the regular prime numbers, the twin primes, and a the frequency of various prime gaps will also be presented. For each of these implementations have slight modifications been made to the code, which will be briefly discussed.

Genom rapportens gång har vi hänvisat till, och uppskattat fördelningen av, ett antal klasser av primtal och deras egenskaper. 
Nu kommer vi att redovisa några resultat som ovan nämnda Python-program givit.
Vi börjar med att räkna primtal i ett intervall och jämföra det med primtalssatsen, detta i syftet att övertyga oss om kodens legitimitet. 
Sedan undersöker vi den möjliga sanningen av en hypotetisk fördelning för primtalstvillingar. 
Slutligen redovisas frekvensen av primtalsgap, och mönster som uppträder däri.

\subsubsection{Fördelningen av primtal}\label{app.primes.title}
%Beginning with the classic example of the set of prime numbers, below is illustrated the distribution predicted by the familiar x over log x from sieve theory/Chebychev, the more accurate lorgarithmic integral Li(x), and the actual number of primes found in using our implementation. In order to generate the following graph, no new sieving methods were introduced, simply the counting fucntion in Appendix X

Vi börjar med att jämföra antalet primtal hittat med \textsc{NewSegSiev} emot fördelningen som ges av primtalssatsen, nämligen
\begin{equation}
    \pi(x) \sim \text{Li}(x) = \int_2^x\frac{1}{\log t}dt\label{app.primes.PNT}.
\end{equation}
För att skapa följande figur använder vi vår implementering av Helfgotts kod och en enkel räknefunktion.

\begin{figure}[H]
    \centering
    \includegraphics[width = \textwidth]{coen/Images/Primes.pdf}
    \caption{I figuren till vänster redovisas antalet primtal som förväntas enligt primtalssatsen, \textit{orange}, och antalet primtal som hittades enligt \textsc{NewSegSiev}, \textit{blå}, då \(n = 10^{19}\), \(\Delta = 1.25\cdot10^{9}\), och \(K = 2.5\). 
    Notera gärna att kurvorna ligger nästan på varandra. 
    I grafen till höger redovisas inzoomade versioner av grafen till vänster för att visa skillnaden mellan vår uppskattning och det som förväntas vid början av intervallet, där algoritmen returnerar kring 100 färre primtal än förväntat, och vid slutet av intervallet, där algoritmen returnerar kring 2000 färre.
    Dock att kurvorna inte ligger på varandra kan förvantas till en viss grad eftersom det finns ett fel i \ref{app.primes.PNT} vilket inte visas.}
    %This graph shows the relative distributions of primees as per the aforementioned fucntions. Notice that while x over log x appears relatively close for smaller x the logarithmic integral approximation is a near exact match for the true distribution, so much so that the code line is hidden.
    \label{fig:res.prime}
\end{figure}
%As shown in figure X, should we believe in the PNT it seems as though our code does indeed find the correct number of primes. The rather large error in the x over log x estimate is indicative/reflective of one of the limitations highlighted in Section XXX, namely the impact of the error terms and their reconcillation with the main terms. Howeve, there doest exist a deeper link between the x over log x estimate and that of the logarithmic integral. Throguh the use of integral wizardry you can decompose the logarithmic integral into a series of terms, the first of which being x over log x with the remainder being of the order of sqrt(x). This then accounts for the increase in error as x grows. Thiese kinds of approximations for the prime counting function can also be rather naturally extrapolated to those for twin primes, as discussed below.

Som vi ser i vänstra grafen av figur \ref{fig:res.prime}, då vi nu tror på primtalssatsen så verkar det som vår kod hittar det korrekta antalet primtal i intervallet, eftersom skillnaden mellan kurvorna är nästa osynlig på makronivå. 
Att de inte ligger exakt på varandra är huvudsakligen beroende på felet vilket inte visas i \eqref{app.primes.PNT}.
Notera att vi normaliserar felet vid början av intervallet eftersom betraktar skillnaden \(\text{Li}(x) - \text{Li}(n - \Delta)\), \(x\in[n-\Delta,\; n+\Delta]\) då vi plottar antalet förvantat primtal i intervallet.
Att vår kod hittar ungefär 100 färre primtal än förväntat efter 20 000 heltal och ungefär 2000 färre primtal efter \(2.5\times10^9\) heltal är mycket rimligt då vi anser att felet i \eqref{app.primes.PNT} kan bäst vara \(O(2\Delta)^{1/2 + \varepsilon}),\; \varepsilon > 0\), under antagandet att Riemannhypotesen stämmer \cite[Kapitel 5]{RiemannErr}.

Vi väljer \textit{n} till \(10^{19}\) för att först och främst undersöka så långt ifrån noll som möjligt, med tanke på att försöka öka skillnaden mellan antalet primtal som genererades och antalet primtal som förväntades i intervallet\footnote{Att \textit{n} inte valdes större i detta fallet är på grund av körningstiden för koden.}.
Att vi väljer \(\Delta\) till \(1.25\times10^9\) för att använda Helfgotts förbättringar i \textsc{NewSegSiev}, \textsc{DiophAppr} funktionen, och för att göra det krävs en specifik storlek på intervallet.
Slutligen väljer vi \textit{K} till 2.5 för att det är minsta \textit{K}:et som kan väljas enligt Helfgott.
Tillsammans bildar dessa parametrar, med stöd från primtalssatsen, en bra bas för de följande tillämpning då vi sparar bitarray:n och läser in den istället för att generar alla primtal inför varje körning.

\subsubsection{Fördelningen av primtalstvillingar}
%Continuing our presentation of various sets of prime's distributions, next we turn to the other recurring theme of twin primes. The following figure illustrates the distributions of twin primes as predicted by x over log squared x and the second order logarithmic integral against those primes found using our implementation. It should be noted that a rather simple help function, Appendix XXX, was written which searches for non twin primes in the prime list and removes them.

Med hjälp av de primtal vi hittade i \ref{app.primes.title}, vänder vi oss nu till primtalstvillingar. Det finns ingen sats för primtalstvillingar motsvarande primtalssatsen, dock så finns det en förmodan framlagd av Hardy och Littlewood \cite[Förmodan B]{Hardy} som lyder
\begin{equation}
    \pi_2(x) \sim 2\text{C}_2\cdot \text{Li}_2(x) = 2\text{C}_2\int_2^x\frac{1}{(\log t)^2}dt\label{app.twins.TWN}
\end{equation}
där \(\text{C}_2 = 0.66016...\) är primtalstvillingskonstanten \cite{TwinPrimeConstant}. Jämför vi den hypotetiska fördelningen emot antalet primtalstvillingar vi har genererat, så får vi följande figur.

\begin{figure}[H]
    \centering
    \includegraphics[width = \textwidth]{coen/Images/TwinPrimesNoKapp.pdf}
    \caption{I grafen till vänster redovisas antalet primtalstvillingar som förväntas enligt den hypotetiska fördelningen, \textit{orange}, och de primtalstvillingar som hittades enligt \textsc{NewSegSiev}, \textit{blå}, då \(n = 10^{19}\), \(\Delta = 1.25\times10^9\), och \(\text{C}_2\) har avrundats till 0.66. Notera återigen att kurvorna ligger nästan på varandra. I graferna till höger så redovisas inzoomade versioner av grafen till vänster. 
    Översta grafen till höger visar skillnaden mellan antalet primtalstvilling vi har hittat och antalet som förväntas vid slutet av intervallet, där algoritmen returnerar 500 fler primtalstvillingar. Efter de första 20 000 heltalen i intervallet så har algoritmen genererat nästan det exakta antalet primtalstvillingar som förväntas.}
    %This graph shows the relative distributions of the twin primes as per the aforementioned fucntions. Notice once again the accuarcy of the logarithmic integral, once again hiding our code line, as opposed to that of C x over log squared x, where the constant is 2*C_2, or 2 times the twin prime constant (discussed below).
    \label{fig:res.twins}
\end{figure}

Datan i figur \ref{fig:res.twins} ger oss några intressanta insikter angående primtalstvillingar. 
Den första är att datan ger stöd till den hypotetiska fördelningen av primtalstvillingar. 
Som vi ser i grafen till vänster, då vi nu litar på koden så kan den hypotetiska fördelningen av primtalstvillingar definitivt vara rimligt eftersom kurvorna verkar ligga på varandra.
Vid början av intervallet är skillnaden mellan det förväntade antalet primtalstvillingar och antalet som hittades maximalt 4, och vid slutet så har skillnaden ökat till ungefär 500.
I sammanhanget av ett intervall av längd \(2.5\times10^9\) så är en skillnad av 500 ekvivalent med en avvikelse av \(2\times 10^{-7}\) mer primtalstvillingar per heltal än vad som förväntades.
Att antalet primtalstvillingar blir större än det hypotetiskt förväntade antalet vid slutet på intervallet beror möjligen på att vi avrundar \(\text{C}_2\) till 0.66 i kombination med att vi normaliserar felet vid början av intervallet precis som i föregående tillämpning.
Dock så kunde avrundningen av \(\text{C}_2\) tillsammans med normaliseringen av felet också ha ökat antalet förväntade primtalstvillingar till mer än vad det skulle egentligen vara. 
I sin tur leder detta till att förmodan verkar stödjas av vår implementering mer än vad den borde vara.

Den andra insikten grafen ger är kopplad till anmärkningen vid slutet på tillämpningen av Eratosthenes såll i avsnitt \ref{Eratosthenes}.
Vi noterar där att Bruns sats medför att andelen primtalstvillingar av primtalen är relativt liten och då vi jämför graferna i figurer \ref{fig:res.prime} och \ref{fig:res.twins} kan vi se den förväntade relationen.
Vi ser direkt att antalet primtal bland de första 20 000 heltal i den första figuren är kring 400 vilket är ungefär 30 gånger fler än antalet primtalstvillingar som finns i de första 20 000 heltal som betraktas.
När vi istället tittar vid slutet av intervallet blir kvoten mellan antalet primtal och antalet primtalstvillingar inte mindre; den ökar något till ungefär en faktor av 33 fler enkla primtal jämfört med tvillingar.

Vi kan inte använda vår kod som bevis för den hypotetiska fördelningen för primtalstvillingar. 
Dock så ger den oss en känsla för om fördelningen verkar rimlig i detta fallet, vilket den gör. Vi kan också få en känsla för hur få primtalstvillingar det finns jämfört mot antalet primtal, vilket underlättar förståelsen för en av sållteorins första stora framsteg, nämligen Bruns sats.


%There are a number of things to discuss regardng the above figure. 
\subsubsection{Frekvens av primtalsgap}

Vi fortsätter med att ge visualiseringar av egenskaper av de primtal vi har genererat genom att undersöka frekvensen av olika storlekar av primtalsgap. 
En primtalsgap är avståndet mellan ett primtal och sitt efterföljande primtal.
Man kan, med hjälp av primtalssatsen, visa att det förväntade avståndet mellan ett primtal \(p_n\) och nästa, \(p_{n+1}\), är \(\log p_n\).
För att ge intuition om hur verklig det förväntade avståndet är analyserar vi nu de primtalsgap som finns i listan av primtal vi generade i \ref{app.primes.title} med att först redovisa följande figur.

\begin{figure}[H]
    \centering
    \includegraphics[width = \textwidth]{coen/Images/GapsNoKapps.pdf}
    \caption{I figuren så redovisas frekvensen på olika stora primtalsgap i det intervallet som undersöktes tidigare, då \(n = 10^{19}\), \(\Delta = 1.25\cdot10^{9}\), och \(K = 2.5\). Notera att y-axeln är logaritmerad på grund av hur högt frekvensen är för de små primtalsgap. }
    \label{fig:res.gap}
\end{figure}

I figur \ref{fig:res.gap} får vi några insikter kring beteendet av primtalsgap i vårt intervall angående deras fördelning. 
Det mest överraskande aspekten av grafen, för någon som har inte sett en sådan graf förut, är hur rak den är då vi logaritmerar y-axeln.
Detta påpekar att frekvensen på primtalsgap minskar exponentiellt då dess längd ökar. 
Vi ser att det mest frekvent längd på primtalsgap i vårt intervall var 6 och längsta strecket utan något primtal var av längd 668.
En till detalj vilken inte kan läsas lätt av figuren är att det finns par av primtal i intervallet där avståndet mellan dem är lika med varje jämn tal upp till och inklusive 560.
Vi nämnde tidigare att det förväntade längden på primtalsgapet för ett primtal \textit{p} tills nästa primtal är \(\log(p)\). 
Då vi uppskattar våra primtal i intervallet med \(10^{19}\) får vi att det förväntade längden på ett primtalsgap i intervallet är ungefär 43.7491. 
Vi ser att många avstånd överstiger detta och ännu fler är mycket mindre, dock då vi beräknar den genomsnittliga längden på primtals gap i vårt intervall så får vi det till 43.7505.
Återigen stödjer detta resultat legitimiteten av vår kod eftersom den bygger på den välkända resultaten av primtalssatsen.

Om vi återgår till primtalstvillingar så säger figuren även något om dem. 
Vi kan se att om alla primtal med ett visst primtalsgap till nästa primtal skulle betraktas vara lika speciellt som primtalstvillingar, då skulle primtalstvillingar vara de sjunde vanligaste typ av primtal i vårt intervall, och består av 0.03 procent av alla primtal i intervallet.

Såsom mycket annat angående primtal, finns det åtminstone en förmodad kopplad mellan dessa frekvenser och vilket primtalsgap som mest frekvent upp till en given övregräns. 
Som vi ser i figur \ref{fig:res.gap} är 6 den mest frekventa längden på primtalsgap i vårt intervall, och det som är speciellt med 6 är att det är produkten av de första två primtalen. 
Vi ser också från figuren att 30, vilket är produkten av de första 3 primtalen, hoppar upp lite från översta delen av bandet, och även 210 med. Förmodan är då att varje \textit{primorial}, produkten av de första \textit{m} primtalen, någon gång blir "hoppmästaren" (namnet gavs av J. H. Conway, 1993).
Förmodan har undersökts direkt med hjälp av numeriska uppskattningsmetoder \cite{primeGap} som har lyckats visa att 6 är hoppmästaren fram till \(\approx 1.74\cdot10^{35}\) och att 30 är hoppmästaren mellan föregående och \(\approx 10^{425}\).

Dock, som antyddes kortfattat i inledningden, finns oändligt många par av primtal med maximalt 600 steg emellan sig vilket betyder att, även för de stora primtalen, måste det finnas primtal relativt nära varandra.
Detta bevisades med hjälp av nya sållmetoder, specifikt GPY metoder, där vikterna som använts i beviset liknar de som används i Selbergs såll. 
Ett av delmålen är att en dag kunna reducerar de 600 steg till 2 och därmed bevisa primtalstviliingshypotesen.

%Nutidens forskningen kring primtalsgap representerar kulmen på sållteorins teoretisk tillämpning.
%Teorin har utvecklats markant under de senaste hundra åren och tillämpats framgångsrikt på många problem inom talteori.
%Om teorins utveckling fortsätter på denna bana, så kommer fler och fler problem kunna lösas med hjälp av bättre och bättre sållteori.
 




%\begin{thebibliography}{FFF}
%\bibitem[UR]{rapp} Utformning av rapporter och kandidatarbetens skriftliga presentation för Civilingenjörsprogrammen vid Chalmers tekniska högskola. 2008. Göteborg: Chalmers Tekniska Högskola

\newpage
\emergencystretch=1em
\printbibliography

%\end{thebibliography}
\medskip

\newpage
\appendix

\section{Ordonotation} 
    I den här uppsatsen gör vi flitig användning av ordonotation för att beteckna olika asymptotiska gränser. För $D \subset \mathbb{C}$ och två funktioner, $f: D \to \mathbb{C}$ och $g: D \to \mathbb{R}_+$ skriver vi
\begin{align*}
    f(x) = O(g(x)) \quad \text{om det finns } A > 0 \text{ så att} \quad \abs{f(x)} \leq A g(x), \forall x \in D.
\end{align*}
Omväxlande skriver vi även \(f(x) \ll g(x)\) med samma betydelse som ovan. Om vi har att \(f(x) \ll g(x)\) och \(g(x) \ll f(x)\) så skriver vi \(f(x) \asymp g(x)\). \label{APDX:Ordo}

\section{Några användbara resultat}
    \subsection{Partiell summation}
    Följande sats är hämtad från \cite[Sats 1.3.1]{cojocarumurty} dock så ger vi lite mer förklaring i beviset.
\begin{theorem}\label{APDX:parSum}
Låt \(c_1,c_2,...\) vara en foljd av komplex tal och sätt
\begin{equation}
    S(x) := \sum_{n\leq x}c_n\nonumber
\end{equation}
Låt \(n_0\) vara ett fixt positivt heltal. Om \(c_j = 0\) för \(j < n_0\) och \(f:[n, \infty) \longrightarrow\mathbb{C}\) har en kontinuerlig derivata i \([n_0, \infty)\), då för något heltal \(x > n_0\) har vi att
\begin{equation*}
    \sum_{n \leq x} c_n f(n) = S(x)f(x) - \int_{n_0}^xS(t)f'(t)dt
\end{equation*}
\end{theorem}
\begin{proof}
Vi bevisar satsen genom att först inse att
\begin{equation*}
    c_n = S(n) - S(n-1).
\end{equation*}
Vi kan därför skriva om vänsterleden i satsen till
\begin{align*}
    \sum_{n \leq x} c_n f(n) &= \sum_{n \leq x} (S(n) - S(n-1)) f(n)\\
    &= \sum_{n \leq x} S(n) f(n) - \sum_{n \leq x-1} S(n) f(n+1)\\
    &= S(x)f(x) + \sum_{n \leq x-1} S(n) f(n) - \sum_{n \leq x-1} S(n) f(n+1)\\
    &= S(x)f(x) - \sum_{n \leq x-1} S(n) (f(n+1)-f(n))\\
    &= S(x)f(x) - \sum_{n \leq x-1} S(n) \int_n^{n+1}f'(t)dt\\
    &= S(x)f(x) - \int_{n_0}^{x}S(t)f'(t)dt
\end{align*}
Att vi kan flytta in \(S(x)\) i integralen beror på att \(S(x)\) är en stegfunktion och är konstant på intervaller på formen \([n, n+1)\).
\end{proof}
    \subsection{Summan av primtalsreciproker}
    \cite[Sats 1.4.3]{cojocarumurty}
\begin{theorem}
    $\sum_{p \leq n} \frac{\log p}{p} = \log n + O(1) $.
\end{theorem}
\begin{proof}
Nyckeln till beviset av ovanstående sats i \cite{cojocarumurty} är två observationer om fakultetfunktionen, \(n! = 1 \cdot 2 \cdot ... \cdot n\). Först ser vi att om vi vill omformulera \(n!\) som en produkt av primtalspotenser så ser vi att den bidragande faktorn för varje $p \leq n$ kan formuleras som 
\begin{align*}
    \prod_{\substack{p^\alpha \pdiv k \\ k \leq n}} p^\alpha = \prod_{p^\alpha \leq n} p^{\lfloor n/p^{\alpha} \rfloor} = p^{\lfloor n/p \rfloor + \lfloor n/p^2 \rfloor + ... + \lfloor n / p^{\lfloor \log n / \log p \rfloor} \rfloor}
\end{align*}
där \(p^\alpha \pdiv k\) betyder att \(\alpha\) är den största heltalsexponenten så att $p^\alpha$ delar $k$. Detta ger oss att 
\begin{align*}
    \log n! = %\log\left( \prod_{p \leq n} p^{\lfloor n/p \rfloor + \lfloor n/p^2 \rfloor + ... + \lfloor n / p^{\lfloor \log n / \log p \rfloor} \rfloor} \right) =
    \sum_{p \leq n} \left(\left\lfloor\frac{n}{p} \right\rfloor + \left\lfloor \frac{n}{p^2} \right\rfloor + ... + \left\lfloor n / p^{\lfloor \log n / \log p \rfloor} \right\rfloor\right) \log p
\end{align*}
vari en restterm kan urskiljas och uppskattas till
\begin{align*}
    \sum_{p \leq n} \left(\left\lfloor \frac{n}{p^2} \right\rfloor + ... + \left\lfloor n / p^{\lfloor \log n / \log p \rfloor} \right\rfloor\right) \log p \leq
    \sum_{p \leq n} \left( \frac{n}{p^2}  \sum_{k=0}^\infty \frac{1}{p^k} \right) \log p
    = n \sum_{p \leq n} \frac{1}{p^2} \cdot \frac{\log p}{1 - 1/p} \ll n.
\end{align*} % Det här kan förbättras
Den första observationen är alltså att
\begin{align}
    \log n! \leq n \sum_{p \leq n} \frac{\log p}{p} + O(n).
\end{align}

Den andra observationen ser vi genom att utföra en partiell summation,
\begin{align*}
    \log n! = \log \left(\prod_{k \leq n} k\right) = \sum_{k \leq n} 1 \cdot k = 
    \lfloor n \rfloor \log n - \int_1^n \frac{\lfloor t \rfloor}{t} \text{d} t 
\end{align*}
och att \(\lfloor t \rfloor = t + O(1)\) ger att detta är lika med
\begin{align}
    (n + O(1)) \log n - \int_1^n \frac{t}{t} \text{d} t + O(1) \int_1^n \frac{1}{t} \text{d} t  = n \log n - n + O(\log n).
\end{align}

\end{proof}
    
    \subsection{Asymptotiskt beteende för \texorpdfstring{W(z)}{W(z)}, ett specialfall}
    % Skrivet av Nils
Följande sats beskriver det asymptotiska beteendet hos $W(z)$ i specialfallet då $\omega(2)=1$ och $\omega(p)=2$ för primtal $p>2$. 
Resultatet används i (\ref{brun.application}) samt (\ref{eratosthenes.tillämpning}) och idéerna till beviset är hämtade från beviset av \textit{Merten's Theorem} i \cite[kap 5.2]{cojocarumurty}.

\begin{theorem} \label{APDX:asympW}
Antag att $\omega(2)=1$ och $\omega(p)=2$ för alla primtal $p>2$, och definiera 
\begin{equation*}
    W(z):=\prod_{p<z}\biggl( 1-\frac{\omega(p)}{p} \biggr).   
\end{equation*}
Då gäller det att
\begin{equation} \label{APDX:asympW.main}
    W(z) \asymp (\log z)^{-2},
\end{equation}
då $z\to\infty$.
\end{theorem}


\begin{proof}
Från (\ref{era.app.mainAppr}) och (\ref{era.app.secondMainAppr}) har vi redan att
\begin{align*}
    W(z) \ll \exp \biggl( - \sum_{2 <p < z} \frac{2}{p}  \biggr),
    \quad\text{och}\quad
    \exp \biggl( - 2 \sum_{2 <p < z} \frac{1}{p}  \biggr) \asymp (\log z)^{-2}.
\end{align*}
Det som återstår att bevisa är därmed
\begin{equation} \label{APDX:asympW.gg}
    W(z) \gg \exp \biggl( - \sum_{2 <p < z} \frac{2}{p}  \biggr).
\end{equation}
Låt oss betrakta
\begin{equation*}
    -\log(W(z)) 
    = \sum_{p < z} -\log\biggl( 1-\frac{\omega(p)}{p} \biggr) 
    = \sum_{p < z} \log\biggl( 1+\frac{\omega(p)}{p-\omega(p)} \biggr).
\end{equation*}
Användning av antagandet på $\omega$ implicerar att ovanstående är lika med 
\begin{equation} \label{APDX:asympW.sum}
    \log 2 + \sum_{2<p<z} \log\biggl( 1+\frac{2}{p-2} \biggr)
    \leq \log 2 + \sum_{2<p<z} \frac{2}{p-2}.
\end{equation}
där olikheten håller eftersom $\log(1+x)\leq x$, för alla $x\geq0$. 
Betrakta nu summan i högerledet och gör omskrivningen
\begin{equation*}
    \sum_{2 <p < z} \frac{2}{p-2} = \sum_{2 <p < z} \frac{2}{p} + \sum_{2 <p < z} \frac{2}{p(p-2)}.
\end{equation*}
Här gäller det att den andra summan konvergerar då $z\to\infty$, vilket kan ses genom
\begin{equation*}
    \sum_{2 <p < z} \frac{2}{p(p-2)} < 2\sum_{2 <p < z} \frac{1}{p^2} < 2\sum_{n = 1}^\infty \frac{1}{n^2},
\end{equation*}
där summan till höger konvergerar och är lika med $\pi^2/3$.
Om vi nu återgår till högerledet i (\ref{APDX:asympW.sum}) får vi slutligen att
\begin{align*}
    -\log(W(z)) \gg \sum_{2 <p < z} \frac{2}{p},
\end{align*}
vilket implicerar (\ref{APDX:asympW.gg}) och beviset är klart.
\end{proof}
    
    \subsection{Ett lemma angående multiplikativa funktioner}
    Följande lemmat är hämtad från \cite[Lemma 7.2.2]{cojocarumurty}, där beviset utelämnas.
\begin{lemma}\label{APDX:multFunk}
Låt \textit{f} vara en multiplikativ funktion med \(d_1,d_2\) positiva, kvadratfria heltal. Då
\begin{equation}
    f([d_1,d_2])\cdot f((d_1, d_2)) = f(d_1)f(d_2)\nonumber
\end{equation}
\end{lemma}
\begin{proof}
Vi delar upp beviset i två fall där \(d_1\) och \(d_2\) är antingen relativt prima eller inte. Då \(d_1\) och \(d_2\) är relativt prima är \((d_1,d_2) = 1\) och \([d_1,d_2] = d_1d_2\). Detta medför att
\begin{equation}
    f([d_1,d_2])\cdot f((d_1, d_2)) = f(d_1d_2)f(1) = f(d_1)f(d_2)\nonumber
\end{equation}
där sista likheten gäller för att  \(d_1\) och \(d_2\) är relativt prima. 

Då  \(d_1\) och \(d_2\) inte är relativt prima är både deras lägsta gemensamma multipel och största gemensamma delare också kvadratfria. Genom att faktorisera minsta gemensamma multipeln \([d_1,d_2]\) och största gemensamma delaren \((d_1,d_2)\) som
\begin{align}
    [d_1,d_2] &= p_1p_2...p_m\nonumber\\
    (d_1,d_2) &= q_1q_2...q_l\nonumber
\end{align}
så får vi med användning av formeln \([d_1,d_2]\cdot(d_1,d_2) = d_1d_2\) att
\begin{equation}
    f([d_1,d_2])\cdot f((d_1, d_2)) = f(p_1)f_(p_2)...f(p_m)f(q_1)f(q_2)...f(q_l) = f(d_1)f(d_2)\nonumber
\end{equation}
där i sista likheten arrangerar om de primtal som behövs för att få tillbaka \(d_1\) och \(d_2\). Detta kan vi göra på grund av formeln som nämndes tidigare.
\end{proof}
    
    \subsection{En variant på Möbius inverteringsformel}
    Formuleringen av satsen är hämtad från \cite[Sats 1.2.3]{cojocarumurty}, där beviset utelämnas.
\begin{theorem}\label{APDX:mobDual}
Låt \(\mathcal{D}\) vara en sluten delare mängd av naturliga tal (det vill säga, om \(d \in \mathcal{D}\) och \(d'\divides d\), då är \(d'\in \mathcal{D}\). Låt \textit{f} och \textit{g} vara två komplexvärda funktioner på de naturliga talen. Om
\begin{equation}
    f(n)=\sum_{\substack{n\divides d\\d\in \mathcal{D}}}g)(d)\nonumber
\end{equation}
då
\begin{equation}
    g(n) = \sum_{\substack{n\divides d\\d\in \mathcal{D}}}\mu\bigg(\frac{d}{n}\bigg)f(d)\nonumber
\end{equation}
och motsatsen också gälller (om man antar att alla serier är absolutkonvergenta).
\end{theorem}
\begin{proof}
Vi bevisar bara framåt riktningen, eftersom omvändningen  bevisas  på likadant sätt. Låt
\begin{equation}
    f(n)=\sum_{\substack{n\divides d\\d\in \mathcal{D}}}g(d)\nonumber
\end{equation}
då är
\begin{equation}
    \sum_{\substack{n\divides d\\d\in \mathcal{D}}}\mu\bigg(\frac{d}{n}\bigg)f(d) = \sum_{\substack{n\divides d\\d\in \mathcal{D}}}\mu\bigg(\frac{d}{n}\bigg)\sum_{\substack{d\divides d'\\d'\in \mathcal{D}}}g(d').\nonumber
\end{equation}
Vi nu använder att
\begin{equation}
    d' = \frac{d'}{d}d = \frac{d'}{d}\frac{d}{n}n = kln\nonumber
\end{equation}
med \(k = d'/d\) och \(l = d/n\) för att skriva om föregående summa på följande sätt.
\begin{equation}
    \sum_{\substack{nl = d\\d\in \mathcal{D}}}\mu(l)\sum_{\substack{dk= d'\\d'\in \mathcal{D}}}g(d') = \sum_{\substack{n \divides d'\\d'\in \mathcal{D}}}g(d')\sum_{k\divides \frac{d'}{n}}\mu(k).\label{APDX:dualMob.proof.div}
\end{equation}
Enligt den första egenskapen som redovisas i \ref{Mobius}, så är inre summan i \eqref{APDX:dualMob.proof.div} antingen 1 eller 0 beroende på om \(d'/n = 1\) eller inte. Detta medför att
\begin{equation}
    \eqref{APDX:dualMob.proof.div} = g(n)\nonumber
\end{equation}
eftersom \(n \in \mathcal{D}\) på grund av att \(n \divides d'\).
\end{proof}
    
\section{Kedjebråk}
    % Skrivet av Erik

I avsnitt \ref{partB.algoritmteori}, där vi går igenom en implementering av Eratosthenes såll, så gör vi bruk av en \textit{diofantisk approximation}, \cite[Algoritm 4]{HaraldSieve}. För att förstå det här steget krävs först lite förkunskaper om kedjebråk. 

Ett \textit{ändligt kedjebråk} definieras i \cite[Definition 20.1]{Lindahl} som
\begin{align*}
    \langle a_0, a_1, ..., a_n \rangle := a_0 + \cfrac{1}{a_1 + \cfrac{1}{a_2 + \cfrac{1}{\ddots \raisebox{-3mm}{$a_{n-2}+\cfrac{1}{1 + \cfrac{1}{a_n}}$}}}}
\end{align*}
där \(a_0, a_1, ..., a_n\) är reella tal och \(a_i > 0\) för \(i > 0\). I den här uppsatsen berör vi endast särfallet då \(a_0, a_1, ..., a_n\) är heltal vilket då kallas för ett \textit{enkelt kedjebråk}. 

En omedelbar utmaning är att, givet ett kedjebråk \(\langle a_0, a_1, ..., a_n \rangle \), beräkna \(\langle a_0, a_1, ..., a_{n+1} \rangle \) utan att räkna om hela kedjebråket från term \(a_{n+1}\) till \(a_0\). Lösningen på det här problemet är så kallade \textit{konvergenter}, \cite[Definition 20.4]{Lindahl}, och definieras som ett par \((p_n,q_n)\), där
\begin{align*}
    p_{-2} &= 0, p_{-1} = 1, p_n = a_n p_{n-1} + p_{n-2} \text{ då } n \geq 0 \\
    q_{-2} &= 1, q_{-1} = 0, q_n = a_n q_{n-1} + q_{n-2} \text{ då } n \geq 0,
\end{align*}
och deras kvot, \(c_n := p_n / q_n\), \(n \geq 0\). Observera att definitionen ger oss att \(q_0 = 1\) och \((q_n)_{n=1}^{N}\) är en strikt växande följd om motsvarande kedjebråk är enkelt. Anledningen till att vi introducerar konvergenter ges av följande sats, \cite[Sats 20.5i) och ii)]{Lindahl},
\begin{theorem} \label{app.konvergenter}
    Låt \((a_n)_{n=0}^{N}\) vara en följd av reella tal där \(a_i > 0\) för \(i > 0\) och \((p_n, q_n)\) är deras respektive konvergenter. Då gäller att
    \begin{enumerate}
        \item \(\langle a_0, a_1, ..., a_n \rangle = c_n\) för alla \(n \geq 0\),
        \item \(p_n q_{n-1} - p_{n-1} q_n = (-1)^{n-1}\) för alla \(n \geq -1\).
    \end{enumerate}
\end{theorem}


\todo{Skriv sats 20.9}


%Definitionen av ett \textit{oändligt kedjebråk} (\cite[Definition 20.3]{Lindahl}) följer naturligt av föregående definition som
%\begin{align*}
%    \lim_{n \to \infty} \langle a_0, a_1, ..., a_n \rangle
%\end{align*} \label{APDX:cfrac}

\section{Python-kod tillhörande avsnitt \ref{partB}}
    % Skrivet av Nils

\newcommand{\code}[1]{\inputminted[frame=lines,fontsize=\footnotesize,linenos]{python}{code/#1.py}}

\subsection{Vår förbättrade version av programmet}
Nedan följer väsentliga delar av den förbättrade versionen av programmet som omnämns i \ref{partB.implementering}.
Koden är skrivet i Python och är en tolkning av den pseudokod som Helfgott presenterar i \cite{HaraldSieve}.
Om läsaren vill testa programmet kan all nedanstående kod förslagsvis läggas i samma fil. 
För att programmet ska fungera behöver Python-bibliotek \textit{Bitarray} vara installerat.


Vi börjar med import av några externa funktioner, detta ska göras först i koden för att de senare funktionerna ska fungera.
Biblioteket \texttt{bitarray} ger oss möjlighet att arbeta direkt med bitsträngar och funktionen \texttt{zeros(N)} konstruerar en sådan bitsträng bestående av \texttt{N} stycken nollor.
Funktionerna \texttt{sqrt} och \texttt{floor} ger kvadratroten respektive golvfunktionen av ett tal 
och \texttt{isqrt} returnerar heltalsdelen av kvadratroten.
\code{imports}


Härnäst följer algoritmerna vars funktion beskrivs i \ref{partB.algoritmteori}, 
på flera rader finns det extra kommentarer (på engelska) som förklarar radens syfte.
Alla funktioner, med undantag för \textsc{DiophAppr}, returnerar en bitsträng. Strängen representerar något sållat heltalsintervall $[a,b]$ 
där biten med index $i$ är en nolla om talet $i+a$ har sållats bort och en etta annars.


\subsubsection*{\textsc{SimpleSiev}}
Detta är det vanliga Eratosthenes såll. Funktionen tar in ett heltal $N$ och returnerar en lista med alla primtal i intervallet $[0,N]$.
Exempelvis ger \texttt{SimpleSiev(7)} svaret \texttt{00110101} vilket representerar alla primtal från 0 till 7.
%Med \texttt{P[i]} menas elementet med index \texttt{i} i listan \texttt{P}.
%Vidare åsyftar \texttt{P[a:b:c]} alla element med index \texttt{a+kb} där \texttt{b} är heltal 
\code{SimpleSiev} 


\subsubsection*{\textsc{SimpleSegSiev}}
Denna funktion tar in tre argument; $n$, $\Delta$ och $M$. Därefter returnerar den en lista vilken representerar alla tal i intervallet $[n, n+\Delta]$ som antingen är prima eller saknar delare mindre än, eller lika med $M$. Använder vi notation från avsnitt \ref{sallproblemet}, representerar denna lista mängden $\S{A}{P}{z}$ där $A=[n, n+\Delta]$, $z=M$ och $\P$ är mängden av alla primtal. 
För att utföra sållningen behövs alltså alla primtal i intervallet $[0,M]$. 
Dessa genereras av \textsc{SimpleSiev} som åkallas på rad 5.
\code{SimpleSegSiev}


\subsubsection*{\textsc{SubSegSiev}}
Denna funktion är exakt som \textsc{SimpleSegSiev} gällande indata och utdata.
Skillnaden mot förstnämnda är att intervallet $[0,M]$ delas upp i mindre delintervall.
Funktionen tar sedan ett delintervall i taget, skapar en lista med alla primtal i delintervallet för att sedan sålla med hjälp av dessa primtal. 
På så sätt behöver inte alla primtal i intervallet $[0,M]$ ligga i arbetsminnet på samma gång.
För att generera listorna används \textsc{SimpleSegSiev}.
\code{SubSegSiev}


\subsubsection*{\textsc{NewSegSiev}}
Detta är algoritmens huvudfunktion. Den tar in tre argument; $n$, $\Delta$ och $K$, och returnerar alla primtal i intervallet $[n-\Delta,n+\Delta]$. Det sista argumentet $K$ ska vara ett flyttal större eller lika med $2.5$ och påverkar hur funktionen går tillväga för att sålla. Funktionen sållningen sker i två delar. Först sållas intervallet för multiplar av primtal mindre än eller lika med $K\Delta$. Detta utförs av \textsc{SubSegSiev} och sker på rad 16. Därefter sållas intervallet för resterande tal, vilket är multiplar av tal mellan $K\Delta$ och $\sqrt{n+\Delta}$. Detta görs med hjälp av \textsc{DiophAppr} i while-loopen som börjar på rad 17 och slutar på rad 31.
\code{NewSegSiev} 

\subsubsection*{\textsc{DiophAppr}}
Denna funktion tar in två tal; $\alpha$ och $Q$, och beräknar en rationell approximation $a/q$ av $\alpha$ med hjälp av kedjebråk.
Heltalen $a$ och $q$ uppfyller $\gcd{a,q}=1$ samt att $q\leq Q$ och $\abs{a/q-\alpha}\leq 1/qQ$.
Funktionen returnerar talparet $(q, a^{-1})$ där $a^{-1}$ är den multiplikativa inversen till $a$ modulo $q$.
\code{DiophAppr} 


\subsection{Några extra funktioner}
\subsubsection*{\textsc{RemoveNonTwins}}
Här presenteras den funktion som använts i \ref{partB.applications}.
Denna tar in en redan sållad lista med primtal och sållar bort alla primtal $p$ där $p+2$ inte är prima.
De tal i listan som är kvar efter detta är primtalstvillingarna i intervallet.
Om det sista eller näst sista talet i listan är prima så är det omöjligt för funktionen att avgöra om talet är ett tvillingprimtal eller ej, funktionen sållar inte bort talet men utfärdar en varning om att detta har inträffat.
\code{RemoveNonTwins}\label{code.twins}

Till sist ger vi även en modifierad version av \textsc{NewSegSiev} och \textsc{SubSegSiev} som vi har valt att ge namnen
\textsc{NewSegSievTwins} respektive \textsc{SubSegSievTwins}.
Som namnet antyder kan \textsc{NewSegSiev} användas för att från grunden sålla fram alla primtalstvillingar i det angivna intervallet och funktionen brukas på samma sätt som \textsc{NewSegSiev}.
Funktionen \textsc{SubSegSievTwins} är nödvändig för att \textsc{NewSegSievTwins} ska fungera.

\subsubsection*{\textsc{NewSegSievTwins}}
\code{NewSegSievTwins}
\subsubsection*{\textsc{NewSegSievTwins}}
\code{SubSegSievTwins}


\end{document}                 % The input file ends with this command.
