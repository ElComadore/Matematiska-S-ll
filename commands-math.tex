% Egna kommandon:
% Delar-operatorer:
\newcommand{\divides}{\mid}
\newcommand{\notdivides}{\nmid}
\newcommand{\pdiv}{\mid\!\mid}
% Största gemensamma delare
\renewcommand{\gcd}[1]{(#1)}
% Kardinalitet
\newcommand{\card}[1]{\# #1}
% Heltalsmängd:
\newcommand{\A}{\mathcal{A}}
% Primtalsmängd:
\renewcommand{\P}{\mathcal{P}}
% Sållad mängd:
\renewcommand{\S}[3]{S(\mathcal{#1}, \mathcal{#2}, #3)}


% Satser etc. mer info hittas på https://www.overleaf.com/learn/latex/theorems_and_proofs
\newtheorem{theorem}{Sats}[section]
\newtheorem{corollary}{Korollarium}[theorem]
\newtheorem{lemma}[theorem]{Lemma}
\newtheorem{definition}[theorem]{Definition}
\newtheorem{proposition}{Proposition}[section]

% Block-kommentar: använd '\begin{comment} ... \end{comment}'

% Todo
\newcommand{\todo}[1]{\smallskip\noindent\big[To do: #1\big]\smallskip}


% För att få en mer kompakt lista
\newenvironment{myitemize}
{ \begin{itemize}
    \setlength{\itemsep}{0pt}
    \setlength{\parskip}{0pt}
    \setlength{\parsep}{0pt}     }
{ \end{itemize}                  } 

