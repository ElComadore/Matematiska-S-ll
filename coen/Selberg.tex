Kring tjugo år efter utvecklingen av succén som var Bruns såll, uppkom två olika såll metoder självständigt av varandra; Linniks Stor Såll och Selbergs Såll. Av dessa har vi valt att fokusera på Selbergs såll, upptäckt av norsk matematiker Atle Selberg, som är av samma kombinatorisk stil som de tidigare såll av Legendre och Brun. Fast till skillnad med de tidigare sållmetoder, valde Selberg att använda sig av en approximation av Möbius funktionen istället för att använda den tidigare funktionen \(W(z)\). Denna approximation tog formen av en kvadratisk form med den centrala idén att for en följd av reella tal \((\lambda_d)\) med \(\lambda_1 = 1\), har vi för alla \textit{k} att 
\begin{equation}
    \sum_{d\divides k} \mu(d) \leq \Bigg(\sum_{d\divides k}\lambda_d\Bigg)^2\label{eq:SelAppr}
\end{equation}
där det nu gäller att göra en bra val på \(\lambda_d\) för att minimisera felet med approximationen. 

För att se hur \eqref{eq:SelAppr} är användbar låt oss nu vända till formuleringen av \(\S\)