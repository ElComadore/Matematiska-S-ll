Kring tjugo år efter utvecklingen av succén som var Bruns såll, uppkom två olika såll metoder självständigt av varandra; Linniks Stor Såll och Selbergs Såll. Av dessa har vi valt att fokusera på Selbergs såll, upptäckt av norsk matematiker Atle Selberg, som är av samma kombinatorisk stil som de tidigare såll av Legendre och Brun. Fast till skillnad med de tidigare sållmetoder, valde Selberg att använda sig av en approximation av Möbius funktionen istället för att använda den tidigare funktionen \(W(z)\). Denna approximation tog formen av en kvadratisk form med den centrala idén att for en följd av reella tal \((\lambda_d)\) med \(\lambda_1 = 1\), har vi för alla \textit{k} att 
\begin{equation}
    \sum_{d\divides k} \mu(d) \leq \Bigg(\sum_{d\divides k}\lambda_d\Bigg)^2.\label{eq:SelAppr}
\end{equation}
Nu gäller att göra en bra val på \(\lambda_d\) för att minimera felet med approximationen. 

För att se hur \eqref{eq:SelAppr} är användbar, låt oss nu vända till formuleringen av \(\S{A}{P}{z}\), nämligen
\begin{align}
    \S{A}{P}{z} = \sum_{\substack{a\in \mathcal{A}\\a\not\in\mathcal{A}_p,\forall p\divides P(z)}} 1 = \sum_{d\divides P(z)} \mu(d) \sum_{a\in\mathcal{A}_d}1 = \sum_{a\in A} \Bigg(\sum_{\substack{d\divides P(z)\\a\in A_d}}\mu(d)\Bigg).\label{eq:sellMe}
\end{align}
Här i \eqref{eq:sellMe} kan vi använda oss av uppskattningen i \eqref{eq:SelAppr} med utvidgningen att \(\lambda_d=0\) då \(d>z\), och då får vi att
\begin{equation}
    \eqref{eq:sellMe} \leq \sum_{a\in A}\Bigg(\sum_{\substack{d\divides P(z)\\a\in A_d}}\lambda_d\Bigg)^2 = \sum_{a\in A}\Bigg(\sum_{\substack{d_1,d_2\divides P(z)\\a\in A_{[d_1, d_2]}}}\lambda_{d_1}\lambda_{d_2}\Bigg) =  \sum_{d_1,d_2\leq z}\lambda_{d_1}\lambda_{d_2}\#A_{[d_1, d_2]}\label{eq:2FatLs}
\end{equation}
där \([a, b]\) betecknar lägst gemensam multipel av \textit{a} och \textit{b}. Om vi nu skulle använda oss av en sådan relation som \eqref{deltaX}, då har vi att 
\begin{equation}
    \S{A}{P}{z} \leq X\sum_{\substack{d_1,d_2\leq z\\d_1,d_2\divides P(z)}}\delta_{[d_1, d_2]}\lambda_{d_1}\lambda_{d_2} + O\Bigg(\sum_{\substack{d_1,d_2\leq z\\d_1,d_2\divides P(z)}}|\lambda_{d_1}||\lambda_{d_2}||R_{[d_1, d_2]}|\Bigg)\label{eq:SelbergStart}
\end{equation}
där första termen anses vara huvud uppskattningen och den andra en felterm. \eqref{eq:SelbergStart} utgör en startpunkt för Selbergs såll, där vi nu måste bestämma en vikt \(\delta_p\) och försök att minimera storleken av \(\lambda_d\).
\subsection{Huvudtermen}
