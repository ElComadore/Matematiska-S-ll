Ett annat perspektiv på sållmetoder gavs av yttligare en norsk matematiker, Atle Selberg, under 1940 talet. 
Hans metod, av en kombinatorisk stil som liknade de tidigare, använde sig av en ny typ av vikt och en uppskattning av en summa av Möbius funktioner för att skapa en ny sållmetod.

\subsection{Beskrivning av sållet}
Selbergs nya vikter, vilka motsvarar \(\delta_d\) i \eqref{deltaX}, bestod av ett delat med en multiplikativ funktion 
\begin{equation}
    f(n) = \sum_{d\divides n}f_1(d)\nonumber
\end{equation}
där \(f_1(d)\) är  multiplikativ också och är entydigt bestämd med hjälp av Möbius inverteringsformel. Det krävs också att \(f(p) > 1\) för alla \(p\in\P\). 
Med en sådan mer allmän vikt hade sållteorin möjligheten att uppskatta nya typer av mängder som tidigare ej varit möjligt att uttrycka med endast restklasser modulo primtal. 

Enligt \cite{cojocarumurty}, bestod summans uppskattning huvudsakligen av att för en given reell talföljd, \((\lambda_d)\), med \(\lambda_1 = 1\), har vi att 
\begin{equation}
    \sum_{d\divides n}\mu(d) \leq \Bigg(\sum_{d\divides n}\lambda_d\Bigg)^2.\nonumber
\end{equation}
För att förstå varför denna uppskattning av Möbiusfunktionens summa fungerar, kom ihåg den första egenskapen som redovisas i \ref{Mobius}. 
Summan är antingen noll eller ett beroende på om \textit{n} är lika med ett eller inte, och uppskattningen är minimalt 0 då \textit{n} är skild ifrån ett, eftersom det är ett reellt tal i kvadrat.
Målet då blir att konstruera talföljden så att uppskattningen minimeras, vilket kommer göras senare.
Med hjälp av talföljden \((\lambda_d)\) kan vi nu uppskatta \(\S{A}{P}{z}\) eftersom
\begin{equation}
\S{A}{P}{z} = \sum_{\substack{a\in \mathcal{A}\\a\not\in\mathcal{A}_p,\forall p\divides P(z)}} 1 = \sum_{d\divides P(z)} \mu(d) \sum_{a\in\mathcal{A}_d}1 = \sum_{a\in A} \Bigg(\sum_{\substack{d\divides P(z)\\a\in A_d}}\mu(d)\Bigg)\leq \sum_{a\in A}\Bigg(\sum_{\substack{d\divides P(z)\\a\in A_d}}\lambda_d\Bigg)^2.\label{sel.sieve.ineq1}
\end{equation}
Genom att skriva om kvadraten av summan som en summa över två olika \textit{d} kan vi omvandla \eqref{sel.sieve.ineq1} till
\begin{equation}
    \S{A}{P}{z} \leq \sum_{\substack{d_1,d_2\leq z\\d_1,d_2\divides P(z)}}\lambda_{d_1}\lambda_{d_2}\#A_{[d_1, d_2]}\nonumber
\end{equation}
där \([a, b]\) betecknar den minsta gemensamma multipeln av \textit{a} och \textit{b}. 
Med infogande av \(\delta_d = 1/f(d)\) i \eqref{deltaX} får vi uppskattningen som är grundläggande för Selbergs såll;
\begin{equation}
    \S{A}{P}{z} \leq X\sum_{\substack{d_1,d_2\leq z\\d_1,d_2\divides P(z)}}\frac{\lambda_{d_1}\lambda_{d_2}}{f([d_1,d_2])} + O\Bigg(\sum_{\substack{d_1,d_2\leq z\\d_1,d_2\divides P(z)}}|\lambda_{d_1}||\lambda_{d_2}||R_{[d_1, d_2]}|\Bigg)\label{sel.sieve.main}.
\end{equation}
För att få en form på \eqref{sel.sieve.main} vilken vi kan använda som ett såll kommer vi nu bestämma talföljden \((\lambda_d)\). 
Om vi följer hur resonemanget redovisas i \cite{cojocarumurty}, så bestämmer vi \(\lambda_d\) genom att först betrakta bara huvudtermem. 
Genom att använda multiplikativiteten av \(f(d)\) går det att skriva om huvudtermen till en kvadratisk form och sedan hitta dess minimum. 
Vid detta minimumvärde blir \(|\lambda_d|\leq 1\) vilket då också ger oss en uppskattning av feltermen. Med alla dessa omskrivningar härleder vi följande sats.

\begin{theorem}[Selbergs såll]\label{sel.sieve.thm}
Behåll notationen från tidigare i avsnittet och rapporten. 
Bestäm en funktion
\begin{equation}
    V(z) = \sum_{\substack{d\leq z\\ d\divides P(z)}}\frac{\mu^2(d)}{f_1(d)}.\nonumber
\end{equation}
Då har vi att
\begin{equation}
    \S{A}{P}{z} \leq \frac{X}{V(z)} + O\Bigg(\sum_{\substack{d_1,d_2\leq z\\d_1,d_2\divides P(z)}}|R_{[d_1, d_2]}|\Bigg).\nonumber
\end{equation}
\end{theorem}
Ovanstående formulering är en kortfattad version av hur satsen redovisas i \cite{cojocarumurty}.


\subsection{Selbergs såll och primtal i aritmetiska talföljder}

Med hjälp av Selbergs såll kan vi lätt uppskatta antalet primtal som finns i en aritmetisk talföjld, d.v.s. antalet primtal på formen \(p = a + tk\) där \textit{a}, \textit{k} är valda konstanter och \textit{t} är godtyckligt. 
Det finns också ett krav på \textit{a} och \textit{k}, nämligen att de är relativt prima, utan det kravet så skulle inga primtal finnas i talföjlden. 
Om vi vill uppskatta antalet primtal på ovanstående form mindre än något \textit{x}, då är det samma som att uppskatta storleken på följande kardinalitet;
\begin{equation}
    \pi(x;k,a) = \#\{p\leq x; p \equiv a \bmod{k}\}.\label{sel.apl.arithPrimes}
\end{equation}

Låt oss nu ta något \(z\leq x,\; z \in \mathbb{R}^+\) som vi kommer att bestämma senare. Då kan vi uppskatta \eqref{sel.apl.arithPrimes} genom att dela upp mängden på följande sätt.
\begin{align}
    \pi(x;k,a) &= \pi(z;k,a) + \#\{z<p\leq x; p \equiv a \bmod{k}\} \nonumber\\
    &\leq z + \#\{n\leq x; n \equiv a \bmod k,\; n \not\equiv 0 \bmod p,\; p\leq z\}.\label{sel.apl.over}
\end{align}
Uppskattningen av den andra kardinaliteten som utförs i \eqref{sel.apl.over} motsvarar en sållning av alla tal som finns i talföjlden med primtal mindre än \textit{z}. 
Det är klart att det är en uppskattning eftersom om det finns ett sammansatt tal i följden som endast är delbart med primtal större än \textit{z} så kommer den inte sållas bort. 
Dock så behöver vi faktiskt inte sålla med avseende på alla primtal mindre än \textit{z}, bara de som är relativt prima med \textit{k}. Om till exempel \(k = 6\) så får \textit{a} inte vara delbar med varken 2 eller 3 och därför inte alla tal i följden heller. 
Följaktligen ger detta oss att vi bara behöver primtal ur \(\P = \{p: (p, k) = 1\}\) och mängden vi vill uppskatta är då \(\S{A}{P}{z} = \{n\leq x; n \equiv a \bmod k,\; n \not\equiv 0 \bmod p,\; p\leq z, (p,k) = 1\}\).

För att kunna använda sållet så behöver vi bestämma \(\A\), \(\A_d\), \textit{X}, \(f(d)\) samt \(f_1(d)\), och \(R_d\). 
Det är uppenbart att \(\A = \{n\leq x; n \equiv a \bmod k\}\) och dessutom att \(\A_d = \{n \leq x; n \equiv a \bmod k,\; n \equiv 0 \bmod d\}\). 
Eftersom \textit{k} och \textit{d} är relativt prima ger Kinesiska restsatsen att det finns ett unikt tal modulo \textit{kd} som är kongruent med \textit{n}, dvs. \(\{\exists! n_0:\; n \equiv n_0 \bmod kd\}\). 
Detta delar då upp intervalet \([0, x]\) i \(\lfloor x/{kd}\rfloor\) stycken delintervall där det i varje delintervall finns det ett element som ska sållas bort vilket medför att \(\#\A_d = \frac{x}{kd} + O(1)\). 
Vi erhåller \(X = \frac{x}{k}\), \(f(d) = d\), och \(R_d = O(1)\). 
För att bestämma \(f_1(d)\) använder vi oss av dess definition, att den kan skrivas som en slags Euler-produkt eftersom \textit{f} är multiplikativ, och att varje \textit{d} är kvadratfri. 
Detta görs på följande sätt;
\begin{align}
    f_1(d) &= \sum_{m\divides d}\mu(m)f(d/m) = \sum_{m\divides p_1...p_j}\mu(m)f((p_1...p_j)/m) = \nonumber\\
    &=\sum_{b_1 = 0}^1...\sum_{b_j = 0}^1\mu(p_1^{b_1}...p_j^{b_j})f((p_1...p_j)/(p_1^{b_1}...p_j^{b_j})) = \prod_{h = 1}^j\sum_{b_h = 0}^1\mu(p_h^{b_h})f(p_h/p_h^{b_h}) = \nonumber\\
    &= \prod_{h = 1}^j\Bigg(f(p_h) - 1\Bigg) = \prod_{h = 1}^j p_h\Bigg(1 - \frac{1}{p_h}\Bigg) = d\prod_{h = 1}^j \Bigg(1 - \frac{1}{p_h}\Bigg) = \phi(d).\label{sel.apl.tot}
\end{align}
I \eqref{sel.apl.tot} ser vi att \(f_1(d) = \phi(d)\) där \(\phi(d)\) är Eulers fi-funktion. 
Tillsammans ger ovanstående att \(\S{A}{P}{z} \leq x/(kV(z)) + O(z^2)\). 

För att uppskatta \(1/V(z)\) tillämpar vi en sats från \cite[Kap. 7]{cojocarumurty} som säger att
\begin{equation}
    f(\overline{P}(z))V(z) \geq f_1(\overline{P}(z))\sum_{\delta \leq z}\frac{1}{\Tilde{f}(\delta)}\label{sel.apl.V}
\end{equation}
där \(\overline{P}(z) = \prod_{p\leq z,\; p \not \in \P}p\) och \(\Tilde{f}(d)\) är en fullständigt multiplikativ funktion med \(\Tilde{f}(p) = f(p)\) för alla primtal \textit{p}. 
Eftersom de enda primtal som inte är i \(\P\) är de som delar \textit{k} har vi enligt \eqref{sel.apl.V} att 
\begin{equation}
    kV(z) \geq \phi(k) \sum_{\delta \leq z}\frac{1}{\delta} = \phi(k)(\log z + O(1)) \iff \frac{1}{V(z)} \leq \frac{k}{\phi(k)(\log z + O(1))}\label{sel.apl.vAppr}
\end{equation}
där vi har använt oss av partiell summering vid likhetstecknet. 
Med hjälp av \eqref{sel.apl.vAppr} har vi nu att
\begin{equation}
     \pi(x;k,a) \leq z + \frac{x}{\phi(k)(\log z + O(1))} + O(z^2) = \frac{x}{\phi(k)(\log z + O(1))} + O(z^2)\nonumber
\end{equation}
där likheten gäller eftersom \(z^1\) absorberas av ordo notationen. 
Genom att likställa de två termerna kan vi bestämma \textit{z} så att båda termerna har samma vikt. Vi följer förslaget på \textit{z} i \cite{cojocarumurty} och tar \(z = (2x/k)^{\frac{1}{2}-\varepsilon_0}\) för något \(\varepsilon_0\in(0, 1)\). Detta val av \textit{z} ger oss den slutliga uppskattningen
\begin{equation}
    \pi(x;k,a) \leq \frac{(2+\varepsilon)x}{\phi(k)\log(2x/k)}\nonumber
\end{equation}
för alla \(\varepsilon > 0\) och \textit{x} större än något \(x_0(\varepsilon) >0\). 
Resultatet är en variant på Brun-Titchmarshsatsen vilket bevisades först, på ett svagare form, i 1930 av Titchmarsh \cite{BrunTitch} med hjälp av Bruns såll. 
Att Selbergs såll kan ge än bättre uppskattning är ett mycket icke-trivialt resultat och reflekterar hur stort påverkan feltermen har på en sållmetods slutlig uppskattning.
Denna idé kommer utvecklas mer i nästa avsnitt.