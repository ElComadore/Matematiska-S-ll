\begin{comment}
Kring tjugo år efter utvecklingen av succén som var Bruns såll, uppkom två olika såll metoder självständigt av varandra; Linniks Stor Såll och Selbergs Såll. Av dessa har vi valt att fokusera på Selbergs såll, upptäckt av norsk matematiker Atle Selberg, som är av samma kombinatorisk stil som de tidigare såll av Legendre och Brun. Fast till skillnad med de tidigare sållmetoder, valde Selberg att använda sig av en approximation av Möbius funktionen istället för att använda den tidigare funktionen \(W(z)\). Denna approximation tog formen av en kvadratisk form med den centrala idén att for en följd av reella tal \((\lambda_d)\) med \(\lambda_1 = 1\), har vi för alla \textit{k} att 
\begin{equation}
    \sum_{d\divides k} \mu(d) \leq \Bigg(\sum_{d\divides k}\lambda_d\Bigg)^2.\label{eq:SelAppr}
\end{equation}
Nu gäller att göra en bra val på \(\lambda_d\) för att minimera felet med approximationen. 

För att se hur \eqref{eq:SelAppr} är användbar, låt oss nu vända till formuleringen av \(\S{A}{P}{z}\), nämligen
\begin{align}
    \S{A}{P}{z} = \sum_{\substack{a\in \mathcal{A}\\a\not\in\mathcal{A}_p,\forall p\divides P(z)}} 1 = \sum_{d\divides P(z)} \mu(d) \sum_{a\in\mathcal{A}_d}1 = \sum_{a\in A} \Bigg(\sum_{\substack{d\divides P(z)\\a\in A_d}}\mu(d)\Bigg).\label{eq:sellMe}
\end{align}
Här i \eqref{eq:sellMe} kan vi använda oss av uppskattningen i \eqref{eq:SelAppr} med utvidgningen att \(\lambda_d=0\) då \(d>z\), och då får vi att
\begin{equation}
    \eqref{eq:sellMe} \leq \sum_{a\in A}\Bigg(\sum_{\substack{d\divides P(z)\\a\in A_d}}\lambda_d\Bigg)^2 = \sum_{a\in A}\Bigg(\sum_{\substack{d_1,d_2\divides P(z)\\a\in A_{[d_1, d_2]}}}\lambda_{d_1}\lambda_{d_2}\Bigg) =  \sum_{d_1,d_2\leq z}\lambda_{d_1}\lambda_{d_2}\#A_{[d_1, d_2]}\label{eq:2FatLs}
\end{equation}
där \([a, b]\) betecknar lägst gemensam multipel av \textit{a} och \textit{b}. Om vi nu skulle använda oss av en sådan relation som \eqref{deltaX}, då har vi att 
\begin{equation}
    \S{A}{P}{z} \leq X\sum_{\substack{d_1,d_2\leq z\\d_1,d_2\divides P(z)}}\delta_{[d_1, d_2]}\lambda_{d_1}\lambda_{d_2} + O\Bigg(\sum_{\substack{d_1,d_2\leq z\\d_1,d_2\divides P(z)}}|\lambda_{d_1}||\lambda_{d_2}||R_{[d_1, d_2]}|\Bigg)\label{eq:SelbergStart}
\end{equation}
där första termen anses vara huvud uppskattningen och den andra en felterm. \eqref{eq:SelbergStart} utgör en startpunkt för Selbergs såll, där vi nu måste bestämma en vikt \(\delta_p\) och försök att minimera storleken av \(\lambda_d\).
\subsection{Huvudtermen}
Låt oss börja med att bestämma en form på \(\delta_d\), och med det målet följer vi Selberg och bestämmer \(\delta_d = 1/f(d)\) för någon multiplikativ funktion \textit{f}. För denna funktion gäller det att \(f(n) = \sum_{d\divides n}f_1(d)\) för någon multiplikativ funktion \(f_1\), som är entydigt bestämd av Möbius inverteringsformeln \(f_1(n) = \sum_{d\divides n}\mu(d)f(n/d)\). Det är naturligt att betrakta \(f(d)\) som en typ av utveckling av \(\omega(d)/d\) idén från de tidigare såll som nu försöker att underlätta mer komplexa uppdelningar av mängden än bara att fokusera på särskilda ekvivalensklasser av \(p\).

Med en bestämd \(\delta_d\) och hjälpen av en lemma angående multiplikativa funktioner, se Appendix B, kan vi nu börja hantera huvudtermen. Insättningen av vår ny \(\delta_d\) och användningen av lemman ger att
\begin{equation}
    \sum_{\substack{d_1,d_2\leq z\\d_1,d_2\divides P(z)}}\frac {\lambda_{d_1}\lambda_{d_2}}{f([d_1, d_2])} = \sum_{\substack{d_1,d_2\leq z\\d_1,d_2\divides P(z)}}\frac {\lambda_{d_1}\lambda_{d_2}}{f(d_1)f(d_2)}\sum_{\delta\divides (d_1,d_2)}f_1(\delta)=\sum_{\substack{\delta\leq z\\\delta\divides P(z)}}f_1(\delta)\Bigg(\sum_{\substack{d\leq z\\d\divides P(z)\\\delta \divides d}}\frac{\lambda_d}{f(d)}\Bigg)^2\label{eq:diagMe}
\end{equation}
och om vi sätter
\begin{equation}
    u_\delta = \sum_{\substack{d\leq z\\d\divides P(z)\\\delta \divides d}}\frac{\lambda_d}{f(d)}\nonumber
\end{equation}
då får vi den diagonaliserade kvadratformen
\begin{equation}
    \eqref{eq:diagMe} = \sum_{\substack{\delta\leq z\\\delta\divides P(z)}}f_1(\delta) u_\delta^2\label{eq:minimiseMe}
\end{equation}
Vi får även från den Möbius dubbla inverteringsformel att
\begin{equation}
    \frac{\lambda_\delta}{f(\delta)} = \sum_{\substack{d\divides P(z)\\\delta \divides d}}\mu(\frac{d}{\delta})u_d \implies
    \begin{cases}
    u_\delta = 0,\quad\delta \geq z\\
    \sum_{\substack{d\leq z\\d\divides P(z)}} \mu(d) u_d = 1
    \end{cases}\nonumber
\end{equation}
\end{comment}

En annan perspektiv på sållmetoder gavs av ytligare en till norsk matematiker Atle Selberg, under 1940 talet. Sin metod, av en kombinatorisk stil som liknade de tidigare, använde sig av en ny typ av vikt och en uppskattning en summa av Möbius funktioner för att skapa en ny sållmetod.

\subsection{Beskrivning av sållet}
Selbergs nya vikter, vilkets multiplikativ invers motsvarar \(\delta_d\) i \eqref{deltaX}, bestod av ett delat med en multiplikativ funktion 
\begin{equation}
    f(n) = \sum_{d\divides n}f_1(d)\nonumber
\end{equation}
där \(f_1(d)\) är också multiplikativ och är entydigt bestämd av sin Möbius inverteringsformel. Det också krävs att \(f(p) > 1\) för alla \(p\in\P\). Med en sådan mer allmän vikt hade sållteorin möjligheten att uppskatta nya typer av mängder som kunde inte vara hanterad med hjälp av endast restklasser modulo primtal. 

Enligt \cite{cojocarumurty}, bestod summans uppskattningen huvudsakligen av att för en given reell talföljd, \((\lambda_d)\), med \(\lambda_1 = 1\), har vi att 
\begin{equation}
    \sum_{d\divides n}\mu(d) \leq \Bigg(\sum_{d\divides n}\lambda_d\Bigg)^2.\nonumber
\end{equation}
Detta betyder att vi kan nu uppskatta \(\S{A}{P}{z}\) med hjälp av \((\lambda_d)\) eftersom
\begin{equation}
\S{A}{P}{z} = \sum_{\substack{a\in \mathcal{A}\\a\not\in\mathcal{A}_p,\forall p\divides P(z)}} 1 = \sum_{d\divides P(z)} \mu(d) \sum_{a\in\mathcal{A}_d}1 = \sum_{a\in A} \Bigg(\sum_{\substack{d\divides P(z)\\a\in A_d}}\mu(d)\Bigg)\leq \sum_{a\in A}\Bigg(\sum_{\substack{d\divides P(z)\\a\in A_d}}\lambda_d\Bigg)^2.\label{eq:SelIneq1}
\end{equation}
Genom att skriva om kvadraten av summan som en summa över två olika \textit{d} kan vi omvandla \eqref{eq:SelIneq1} till
\begin{equation}
    \S{A}{P}{z} \leq \sum_{\substack{d_1,d_2\leq z\\d_1,d_2\divides P(z)}}\lambda_{d_1}\lambda_{d_2}\#A_{[d_1, d_2]}\nonumber
\end{equation}
där \([a, b]\) betecknar den lägsta gemensamma multipeln av \textit{a} och \textit{b}. Med insättning av \(\delta_d = 1/f(d)\) i \eqref{deltaX} får vi uppskattningen som är grundläggande  till Selbergs såll;
\begin{equation}
    \S{A}{P}{z} \leq X\sum_{\substack{d_1,d_2\leq z\\d_1,d_2\divides P(z)}}\frac{\lambda_{d_1}\lambda_{d_2}}{f([d_1,d_2])} + O\Bigg(\sum_{\substack{d_1,d_2\leq z\\d_1,d_2\divides P(z)}}|\lambda_{d_1}||\lambda_{d_2}||R_{[d_1, d_2]}|\Bigg)\label{eq:SelbergMain}
\end{equation}
För att få en form på \eqref{eq:SelbergMain} vilket vi kan använda som ett såll behöver vi bestämma talföljden \((\lambda_d)\). Om vi följer resonemanget i \cite{cojocarumurty}, då bestämmer vi \(\lambda_d\) genom att först betrakta bara huvudtermem. Genom att använda multiplikativiteten av \(f(d)\) går det att skriva om huvudtermen till en kvadratisk form och sedan hitta sin minimum. Vid detta minimumvärde blir \(|\lambda_d|\leq 1\) vilket då ger oss en uppskattning av feltermen med. Alla dessa omskrivningar härleder följande sats.

\begin{theorem}[Selbergs såll]\label{thm:SelbergSieve} Behåll notationen från tidigare i avsnittet och bestäm en funktion
\begin{equation}
    V(z) = \sum_{\substack{d\leq z\\ d\divides P(z)}}\frac{\mu^2(d)}{f_1(d)}.\nonumber
\end{equation}
Då har vi att
\begin{equation}
    \S{A}{P}{z} \leq \frac{X}{V(z)} + O\Bigg(\sum_{\substack{d_1,d_2\leq z\\d_1,d_2\divides P(z)}}|R_{[d_1, d_2]}|\Bigg)\nonumber
\end{equation}
\end{theorem}
Ovanstående formulering är en kortfattat version av hur satsen redovisas i \cite{cojocarumurty}. För att underlätta tillämpningen av sållet ger \cite{cojocarumurty} följande uppskattningar av \(V(z)\).
\begin{lemma}[Uppskattningar av \(V(z)\)]\label{thm:SelVApp}
Behåll notationen från tidigare i avsnittet och låt \(\Tilde{f}(\cdot)\) vara en fullständig multiplikativ funktion definierad som \(\Tilde{f}(p) = f(p)\) för all primtal \textit{p}. Bestäm också
\begin{equation}
    \overline{P}(z) = \prod_{\substack{p\not\in\P\\p<z}} p\nonumber
\end{equation}
Då är
\begin{enumerate}
    \item \(
         V(z) \geq \sum_{\substack{\delta\leq z\\p\divides\delta\implies p\divides P(z)}} {\Tilde{f}(\delta)}^{-1}\nonumber\)
    \item \(
        f(\overline{P}(z))V(z) \geq f_1(\overline{P}(z))\sum_{\delta \leq z}\frac{1}{\Tilde{f}(\delta)}\nonumber
    \)
\end{enumerate}
\end{lemma}
För att ha njuta av lemman måste man först kunna bestämma en funktion \(f(\cdot)\). En exempel på hur man kanske skulle göra det redovisas i nästa avsnitt.
\subsection{Selbergs såll och primtalstvillingar}