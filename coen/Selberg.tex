Kring tjugo år efter utvecklingen av succén som var Bruns såll, uppkom två olika såll metoder självständigt av varandra; Linniks Stor Såll och Selbergs Såll. Av dessa har vi valt att fokusera på Selbergs såll, upptäckt av norsk matematiker Atle Selberg, som är av samma kombinatorisk stil som de tidigare såll av Legendre och Brun. Fast till skillnad med de tidigare sållmetoder, valde Selberg att använda sig av en approximation av Möbius funktionen istället för att använda den tidigare funktionen \(W(z)\). Denna approximation tog formen av en kvadratisk form med den centrala idén att for en följd av reella tal \((\lambda_d)\) med \(\lambda_1 = 1\), har vi för alla \textit{k} att 
\begin{equation}
    \sum_{d\divides k} \mu(d) \leq \Bigg(\sum_{d\divides k}\lambda_d\Bigg)^2.\label{eq:SelAppr}
\end{equation}
Nu gäller att göra en bra val på \(\lambda_d\) för att minimera felet med approximationen. 

För att se hur \eqref{eq:SelAppr} är användbar, låt oss nu vända till formuleringen av \(\S{A}{P}{z}\), nämligen
\begin{align}
    \S{A}{P}{z} = \sum_{\substack{a\in \mathcal{A}\\a\not\in\mathcal{A}_p,\forall p\divides P(z)}} 1 = \sum_{d\divides P(z)} \mu(d) \sum_{a\in\mathcal{A}_d}1 = \sum_{a\in A} \Bigg(\sum_{\substack{d\divides P(z)\\a\in A_d}}\mu(d)\Bigg).\label{eq:sellMe}
\end{align}
Här i \eqref{eq:sellMe} kan vi använda oss av uppskattningen i \eqref{eq:SelAppr} med utvidgningen att \(\lambda_d=0\) då \(d>z\), och då får vi att
\begin{equation}
    \eqref{eq:sellMe} \leq \sum_{a\in A}\Bigg(\sum_{\substack{d\divides P(z)\\a\in A_d}}\lambda_d\Bigg)^2 = \sum_{a\in A}\Bigg(\sum_{\substack{d_1,d_2\divides P(z)\\a\in A_{[d_1, d_2]}}}\lambda_{d_1}\lambda_{d_2}\Bigg) =  \sum_{d_1,d_2\leq z}\lambda_{d_1}\lambda_{d_2}\#A_{[d_1, d_2]}\label{eq:2FatLs}
\end{equation}
där \([a, b]\) betecknar lägst gemensam multipel av \textit{a} och \textit{b}. Om vi nu skulle använda oss av en sådan relation som \eqref{deltaX}, då har vi att 
\begin{equation}
    \S{A}{P}{z} \leq X\sum_{\substack{d_1,d_2\leq z\\d_1,d_2\divides P(z)}}\delta_{[d_1, d_2]}\lambda_{d_1}\lambda_{d_2} + O\Bigg(\sum_{\substack{d_1,d_2\leq z\\d_1,d_2\divides P(z)}}|\lambda_{d_1}||\lambda_{d_2}||R_{[d_1, d_2]}|\Bigg)\label{eq:SelbergStart}
\end{equation}
där första termen anses vara huvud uppskattningen och den andra en felterm. \eqref{eq:SelbergStart} utgör en startpunkt för Selbergs såll, där vi nu måste bestämma en vikt \(\delta_p\) och försök att minimera storleken av \(\lambda_d\).
\subsection{Huvudtermen}
Låt oss börja med att bestämma en form på \(\delta_d\), och med det målet följer vi Selberg och bestämmer \(\delta_d = 1/f(d)\) för någon multiplikativ funktion \textit{f}. För denna funktion gäller det att \(f(n) = \sum_{d\divides n}f_1(d)\) för någon multiplikativ funktion \(f_1\), som är entydigt bestämd av Möbius inverteringsformeln \(f_1(n) = \sum_{d\divides n}\mu(d)f(n/d)\). Det är naturligt att betrakta \(f(d)\) som en typ av utveckling av \(\omega(d)/d\) idén från de tidigare såll som nu försöker att underlätta mer komplexa uppdelningar av mängden än bara att fokusera på särskilda ekvivalensklasser av \(p\).

Med en bestämd \(\delta_d\) och hjälpen av en lemma angående multiplikativa funktioner, se Appendix B, kan vi nu börja hantera huvudtermen. Insättningen av vår ny \(\delta_d\) och användningen av lemman ger att
\begin{equation}
    \sum_{\substack{d_1,d_2\leq z\\d_1,d_2\divides P(z)}}\frac {\lambda_{d_1}\lambda_{d_2}}{f([d_1, d_2])} = \sum_{\substack{d_1,d_2\leq z\\d_1,d_2\divides P(z)}}\frac {\lambda_{d_1}\lambda_{d_2}}{f(d_1)f(d_2)}\sum_{\delta\divides (d_1,d_2)}f_1(\delta)=\sum_{\substack{\delta\leq z\\\delta\divides P(z)}}f_1(\delta)\Bigg(\sum_{\substack{d\leq z\\d\divides P(z)\\\delta \divides d}}\frac{\lambda_d}{f(d)}\Bigg)^2\label{eq:diagMe}
\end{equation}
och om vi sätter
\begin{equation}
    u_\delta = \sum_{\substack{d\leq z\\d\divides P(z)\\\delta \divides d}}\frac{\lambda_d}{f(d)}\nonumber
\end{equation}
då får vi den diagonaliserade kvadratformen
\begin{equation}
    \eqref{eq:diagMe} = \sum_{\substack{\delta\leq z\\\delta\divides P(z)}}f_1(\delta) u_\delta^2\label{eq:minimiseMe}
\end{equation}
Vi får även från den Möbius dubbla inverteringsformel att
\begin{equation}
    \frac{\lambda_\delta}{f(\delta)} = \sum_{\substack{d\divides P(z)\\\delta \divides d}}\mu(\frac{d}{\delta})u_d \implies
    \begin{cases}
    u_\delta = 0,\quad\delta \geq z\\
    \sum_{\substack{d\leq z\\d\divides P(z)}} \mu(d) u_d = 1
    \end{cases}\nonumber
\end{equation}
