Formuleringen av satsen är hämtad från \cite[Sats 1.2.3]{cojocarumurty}, där beviset utelämnas.
\begin{theorem}\label{APDX:mobDual}
Låt \(\mathcal{D}\) vara en sluten delare mängd av naturliga tal (det vill säga, om \(d \in \mathcal{D}\) och \(d'\divides d\), då är \(d'\in \mathcal{D}\). Låt \textit{f} och \textit{g} vara två komplexvärda funktioner på de naturliga talen. Om
\begin{equation}
    f(n)=\sum_{\substack{n\divides d\\d\in \mathcal{D}}}g)(d)\nonumber
\end{equation}
då
\begin{equation}
    g(n) = \sum_{\substack{n\divides d\\d\in \mathcal{D}}}\mu\bigg(\frac{d}{n}\bigg)f(d)\nonumber
\end{equation}
och motsatsen också gälller (om man antar att alla serier är absolutkonvergenta).
\end{theorem}
\begin{proof}
Vi bevisar bara framåt riktningen, eftersom omvändningen  bevisas  på likadant sätt. Låt
\begin{equation}
    f(n)=\sum_{\substack{n\divides d\\d\in \mathcal{D}}}g(d)\nonumber
\end{equation}
då är
\begin{equation}
    \sum_{\substack{n\divides d\\d\in \mathcal{D}}}\mu\bigg(\frac{d}{n}\bigg)f(d) = \sum_{\substack{n\divides d\\d\in \mathcal{D}}}\mu\bigg(\frac{d}{n}\bigg)\sum_{\substack{d\divides d'\\d'\in \mathcal{D}}}g(d').\nonumber
\end{equation}
Vi nu använder att
\begin{equation}
    n\divides d \implies \exists l\in\mathbb{N}:nl = d\nonumber
\end{equation}
och likadant för \textit{d} och \(d'\). Vi får att summan kan då skrivas på formen
\begin{equation}
    \sum_{\substack{nl = d\\d\in \mathcal{D}}}\mu(l)\sum_{\substack{dk= d'\\d'\in \mathcal{D}}}g(d') = \sum_{\substack{n \divides d'\\d'\in \mathcal{D}}}g(d')\sum_{k\divides \frac{d'}{n}}\mu(k).\label{APDX:dualMob.proof.div}
\end{equation}
där vi nu summerar över \textit{l} på yttersta summan i vänsterled. Enligt den första egenskapen som redovisas i \ref{Mobius}, så är inre summan i \eqref{APDX:dualMob.proof.div} antingen 1 eller 0 beroende på om \(d'/n = 1\) eller inte. Detta medför att
\begin{equation}
     \sum_{\substack{n\divides d\\d\in \mathcal{D}}}\mu\bigg(\frac{d}{n}\bigg)f(d) = g(n)\nonumber
\end{equation}
eftersom \(n \in \mathcal{D}\) på grund av att \(\mathcal{D}\) är en sluten delare mängd.
\end{proof}