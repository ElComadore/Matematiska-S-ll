%As mentioned in the introduction, sieve methods, while useful, often require quite a bit of work or refinement in order to achieve the results given by more analytic methods. The central reason for this, as the reader may have noticed throguh the course of this report, is due to the error term. It is a recurring theme of sieve theory in general to attempt to better these error terms in order to minimise the apprroximations which have to be made. The reader need only recall the work done in each of the sections regarding the applications of the sieves where one of the final steps  was to find a z which made the main and error term have equal weight. In order to further show the importance of the error term let us present one final example. Take the approximation of the number of primes less than x, the original muse for much of this theory. For each of the sieves we are sieveing out the euivalence class n equiv 0 mod p, leading, for the sieves of Eratosthenes generalised and Brun, omega(p) = 1 and, in the case of Selberg, f(d) = d. If we consider now the main terms we see, for  both Eratosthenes and Brun, that ... and for Selberg, using the approximation in the application with f1 once again being Eulers totient function, that ....Notice how both main terms are of the same magnitude. Why then can Selbergs sieve return an upper bound reminiscient of the PNT while both Brun and Eratosthenes fall markedly short? Natrually this is due to the error bounds.

%How then do we go about minimising these errors? Naturally, it is not that the error terms arise from nowehre; they are a direct consequence of how the vairous mehtods attempt to manage the Möbisu function. Taking Eratosthenes generla sieve as a baseline of sorts for the handling of the Möbius function, we have presented two examples of methods to manage this function. In Brun's sieve, instead of working with sum mob directly, a function g is introduced such that f = sum mob g, which in combination with use of Möbius inversion and the clever subdivision of the sets comprising S(A,P,Z), we are able to come to the form in section 5, and subsequently form approximations of those sets cardinality. In Selberg's sieve on the otherhand, most of that work is sidestepped entirely, with the introduction of the approximation as outlined in chapter 6. Both of these allowedd the reduction in the size of the error term, with varying degrees of success, which is directly due to their handling of the Möbius function. This does, however, come at a cost; we lose the dersireable asymptotic nature of Eratosthenes general sieve. Yet it is hard to say how valuable that is given the fact that the main term needs approximation in order to yield results regardless. 

%The work with error terms exemplifies one of the major internal issues sieve theory has had to tackle throughout its development. Other areas include the sieveing of sets by large numbers of residue classes. These large sieves were first introduced by Linnik around the same time as Seelberg's sieve was developed and use almost entirely di