Följande lemmat är hämtad från \cite[Lemma 7.2.2]{cojocarumurty}, där beviset utelämnas.
\begin{lemma}\label{APDX:multFunk}
Låt \textit{f} vara en multiplikativ funktion med \(d_1,d_2\) positiva, kvadratfria heltal. Då
\begin{equation}
    f([d_1,d_2])\cdot f((d_1, d_2)) = f(d_1)f(d_2)\nonumber
\end{equation}
\end{lemma}
\begin{proof}
Vi delar upp beviset i två fall där \(d_1\) och \(d_2\) är antingen relativt prima eller inte. Då \(d_1\) och \(d_2\) är relativt prima är \((d_1,d_2) = 1\) och \([d_1,d_2] = d_1d_2\). Detta medför att
\begin{equation}
    f([d_1,d_2])\cdot f((d_1, d_2)) = f(d_1d_2)f(1) = f(d_1)f(d_2)\nonumber
\end{equation}
där sista likheten gäller för att  \(d_1\) och \(d_2\) är relativt prima. 

Då  \(d_1\) och \(d_2\) inte är relativt prima är både deras lägsta gemensamma multipel och största gemensamma delare också kvadratfria. Genom att faktorisera minsta gemensamma multipeln \([d_1,d_2]\) och största gemensamma delaren \((d_1,d_2)\) som
\begin{align}
    [d_1,d_2] &= p_1p_2...p_m\nonumber\\
    (d_1,d_2) &= q_1q_2...q_l\nonumber
\end{align}
så får vi med användning av formeln \([d_1,d_2]\cdot(d_1,d_2) = d_1d_2\) att
\begin{equation}
    f([d_1,d_2])\cdot f((d_1, d_2)) = f(p_1)f_(p_2)...f(p_m)f(q_1)f(q_2)...f(q_l) = f(d_1)f(d_2)\nonumber
\end{equation}
där i sista likheten arrangerar om de primtal som behövs för att få tillbaka \(d_1\) och \(d_2\). Detta kan vi göra på grund av formeln som nämndes tidigare.
\end{proof}