Denna kandidatrapport har skrivits med syfte att introducera några grundläggande idéer inom sållteorin och koppla de till nutidens utvecklar inom ämnet.
Under arbetets utförande har en gruppdagbok samt individuella loggböcker förts.
Dessa loggböcker innehåller detaljer angående utvecklingen av rapportens övergripande struktur, mötesanteckningar, och individuella rapporteringar av hur tiden har tillbringats.

Uppdelningen av skrivandet för rapporten är som följande;
\begin{itemize}
    \item \textbf{Nils Alexandersson}: Populärvetenskaplig presentation, kapitel \ref{brun}, inledningen till kapitel \ref{partB}, \ref{partB.implementering} vår implementation av algoritmerna, och appendix B.3 samt D.
    \item \textbf{Erik Dagobert}: Kapitel \ref{forberedelser} förberedelser, kapitel \ref{sallproblemet} det allmänna sållproblemet, kapitel \ref{Eratosthenes} Eratosthenes generaliserade såll, \ref{partB.algoritmteori} grundläggande teori och algoritmer, och appendix A, B.2 samt C. 
    \item \textbf{Coën Lorcan Olofsson}: Kapitel \ref{inledning} inledning, kapitel \ref{Selberg} Selbergs såll, kapitel \ref{Diskussion} diskussion av sållmetoder, \ref{partB.applications} tillämpningar och resultat, och appendix B.1, B.4 samt B.5.
\end{itemize}
Vi vill också tacka vår handledare Anders Södergren och Lucile Devin för all hjälp de har gett under rapportens gång. 
Deras engagemang och vägledning har varit ovärderlig för arbetet.