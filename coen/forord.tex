Denna kandidatrapport har skrivits i syfte att introducera några grundläggande idéer inom sållteori och koppla dessa till nutidens framsteg inom ämnet.
Under arbetets utförande har en gruppdagbok, samt individuella loggböcker förts.
Dessa loggböcker innehåller detaljer angående utvecklingen av rapportens övergripande struktur, mötesanteckningar, och individuella rapporteringar av hur tiden har tillbringats.

Nedan listas huvudförfattare till rapportens respektive avsnitt.
\begin{itemize}
    \item \textbf{Nils Alexandersson}: 
        Populärvetenskaplig presentation,
        \ref{brun} Bruns såll,
        \ref{partB} Datorimplementation av Eratosthenes såll (inledningen),
        \ref{partB.implementering} Implementation av algoritmerna i Python,
        samt appendix B.3 och D med tillhörande kod.
    \item \textbf{Erik Dagobert}: 
        \ref{forberedelser} Förberedelser, 
        \ref{sallproblemet} Det allmänna sållproblemet, 
        \ref{Eratosthenes} Eratosthenes generaliserade såll,
        \ref{partB.algoritmteori} Grundläggande teori och algoritmer,
        samt appendix A, B.2 och C.
    \item \textbf{Coën Lorcan Olofsson}:
        \ref{inledning} Inledning, 
        \ref{Selberg} Selbergs såll,
        \ref{Diskussion} Diskussion av sållmetoder,
        \ref{partB.applications} Tillämpningar och resultat,
        samt appendix B.1, B.4 och B.5.
\end{itemize}
Vi vill också tacka våra handledare Anders Södergren och Lucile Devin för all den hjälp de har givit oss. 
Deras engagemang och vägledning har varit ovärderlig för detta arbete.