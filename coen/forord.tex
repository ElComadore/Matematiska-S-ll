Denna kandidatrapport har skrivits med syfte att introducera några grundläggande idéer inom sållteorin och koppla de till nutidens utvecklar inom ämnet.
Under arbetets utförande har en gruppdagbok samt individuella loggböcker förts.
Dessa loggböcker innehåller detaljer angående utvecklingen av rapportens övergripande struktur, mötesanteckningar, och individuella rapporteringar av hur tiden har tillbringats.

Uppdelningen av skrivandet för rapporten är som följande;
\begin{itemize}
    \item \textbf{Nils Alexandersson}: Populärvetenskaplig presentation, kapitel \ref{brun}, inledningen till kapitel \ref{partB}, och \ref{partB.implementering} vår implementation av algoritmerna.
    \item \textbf{Erik Dagobert}: Kapitel \ref{forberedelser} förberedelser, kapitel \ref{sallproblemet} det allmänna sållproblemet, kapitel \ref{Eratosthenes} Eratosthenes generaliserade såll, och \ref{partB.algoritmteori} grundläggande teori och algoritmer. 
    \item \textbf{Coën Lorcan Olofsson}: Kapitel \ref{inledning} inledning, kapitel \ref{Selberg} Selbergs såll, kapitel \ref{Diskussion} diskussion av feltermer, och \ref{partB.applications} tillämpningar och resultat.
\end{itemize}