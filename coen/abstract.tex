This report aims to introduce the reader to the mathematical area of sieve theory through the elucidation of its fundamental principles and applications, as well as presenting an implementation in code of Eratosthenes sieving algorithm. 
We present the generalised sieve of Eratosthenes, as well as the sieves of Brun and Selberg; briefly describing their history and then focusing on outlining their general derivation as well as giving a thorough example of their application on sets such as the twin primes and primes in arithmetic progressions. 
Following the presentation of the sieves is a discussion which aims to focus the reader's attention on the effects and origins of the error terms of the various methods and there effects they have on the quality of estimations which can be made.

Having presented the theory in some detail, we move our attention to an implementation of Eratosthenes original algoritm as based upon the work of Helfgott \cite{HaraldSieve}.
We present the underpinning mathematical principles and general algorithmic structure of Helfgott's implementation as well as our interpretation of his code and eventual improvements to the implementation.
Following that is a presentation of results regarding the distributions of the prime numbers, the twin primes, and the behaviour of prime gaps in the interval \(10^{19}\pm 1.25\times10^9\).
Through the comparison of our code with the prime number theorem as applied to this interval we bolster it's legitimacy, which leads to our investigation of the conjectured distribution of twin primes in our interval.
We conclude this report with an analysis of the frequency of the prime gaps in our interval and discuss briefly its connection to modern day advances in and about the theory.