% \documentclass[a4]{article}
% \usepackage{graphicx}
% \usepackage{amsmath, amssymb}
% \usepackage{epsfig}
% \usepackage{floatflt} %för inkapslade bilder.

% \addtolength{\textwidth}{10mm}
% \addtolength{\textheight}{30mm}
% \addtolength{\headheight}{-10mm}

% \usepackage[T1]{fontenc}                % För svenska bokstäver
% \usepackage[swedish]{babel}             % För svensk avstavning och svenska
%                                         % rubriker (t ex "innehållsförteckning)

%Var noggranna med att ange källor till det ni skriver. Vi rekommenderar Vancouver-systemet\footnote{Även annat system accepteras om det används konsekvent.} som är mest använt på MV. Man kan antingen använda siffror [1], [2] etc, eller initialer som associerar till författarnamnet(n) t.ex [BN], [BS] etc. Det senare kan vara lite jobbigt om man har många källor men praktiskt om man har några som huvudreferenser. Läs mer i Fackspråks skrift: Utformning av rapporter och kandidatarbetens skriftliga .... (2008-01-11)\cite{rapp}. I det fall arbetet i huvudsak bygger på en eller ett par källor och det är svårt att identifiera exakt när man använder respektive källa, kan man tala om detta i inledningen. Man kan sedan referera till källan om man återger en definition, en sats eller ett bevis eller på annat sätt ligger nära källan. En direkt översättning kan jämställas med ett citat, återberätta därför som om ursprunget var en skrift på svenska så att ni håller er långt ifrån gränsen för plagiering. Är ni osäkra på något så fråga examinator eller handledare.\footnote{Läs mer om att hantera källor och akademisk hederlighet på Chalmers webbsida \hfill \\ URL: https://writing.chalmers.se/chalmers-skrivguide/att-hantera-kallor  }

Förutom att vara klassiska frågor inom talteori angående primtal, vad har primtalstvillingshypotesen, Goldbach hypotesen, och storleken på primtals på primtalsgap att göra med varandra? Kort sagt; sållteori. Lite längre sagt; under den senaste århundrade har tekniker inom sållteori utvecklades och tillämpades på dessa, och fler, blandade problem med relativt stort succé. Vi har lyckats att att bevisa att det finns oändligt många tal \textit{p} sådan att \textit{p} och \(p+2\) är antingen prim eller semiprim \cite{chen2Prime},