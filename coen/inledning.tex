% \documentclass[a4]{article}
% \usepackage{graphicx}
% \usepackage{amsmath, amssymb}
% \usepackage{epsfig}
% \usepackage{floatflt} %för inkapslade bilder.

% \addtolength{\textwidth}{10mm}
% \addtolength{\textheight}{30mm}
% \addtolength{\headheight}{-10mm}

% \usepackage[T1]{fontenc}                % För svenska bokstäver
% \usepackage[swedish]{babel}             % För svensk avstavning och svenska
%                                         % rubriker (t ex "innehållsförteckning)

%Var noggranna med att ange källor till det ni skriver. Vi rekommenderar Vancouver-systemet\footnote{Även annat system accepteras om det används konsekvent.} som är mest använt på MV. Man kan antingen använda siffror [1], [2] etc, eller initialer som associerar till författarnamnet(n) t.ex [BN], [BS] etc. Det senare kan vara lite jobbigt om man har många källor men praktiskt om man har några som huvudreferenser. Läs mer i Fackspråks skrift: Utformning av rapporter och kandidatarbetens skriftliga .... (2008-01-11)\cite{rapp}. I det fall arbetet i huvudsak bygger på en eller ett par källor och det är svårt att identifiera exakt när man använder respektive källa, kan man tala om detta i inledningen. Man kan sedan referera till källan om man återger en definition, en sats eller ett bevis eller på annat sätt ligger nära källan. En direkt översättning kan jämställas med ett citat, återberätta därför som om ursprunget var en skrift på svenska så att ni håller er långt ifrån gränsen för plagiering. Är ni osäkra på något så fråga examinator eller handledare.\footnote{Läs mer om att hantera källor och akademisk hederlighet på Chalmers webbsida \hfill \\ URL: https://writing.chalmers.se/chalmers-skrivguide/att-hantera-kallor  }

Förutom att vara klassiska frågor inom talteori angående primtal, vad har primtalstvillingshypotesen, Goldbach hypotesen, och storleken på primtalsgap att göra med varandra? Kort sagt; sållteori. Lite längre sagt; under den senaste århundrade har tekniker inom sållteori utvecklades och tillämpades på dessa, och fler, blandade problem inom talteori, och även vidare ämnen, med relativt stort succé. Med hjälp av sållteori har matematiken lyckats att bevisa att det finns oändligt många tal \textit{p} sådan att \textit{p} och \(p+2\) är antingen prim eller semiprim \cite{chen2Prime} och att det det finns alltid två primtal inom 600 heltal av varandra \cite{mayBound}. Dessa tillämpningar väcker frågan, vad är en matematisk såll för något? 

En matematisk såll är en metod/teknik med sin ursprung i talteori som försöker att uppskata kardinaliteten av en, så kallad, siktad mängd där alla element i mängden har någon gemensam egenskap. En prototypisk exempel på en sådan mängd som man är intresserad i storleken av är mängden som består av alla primtal mindre än något tal \textit{x}. Sållteori har försökt att hitta och förfinna uppskattningar av storleken på precis den mängden sedan Chebychev's berömd uppskattning i 1851. Dock har matematiska såll sin ursprung ännu längre sedan i antikens grekland med arbetet av Erastosthenes. Sin idé följde nedanstående mönster;
\begin{enumerate}
    \item Med början vid 2, lista ut alla tal upp till talet du vill använda som övergräns.
    \item Skriv en cirkel kring 2 och dra en linje igenom alla tal delbar med 2.
    \item Skriv en cirkel kring det nästa talet som inte har en linje genom sig och dra en linje igenom alla tal delbar med den.
    \item Upprep föregående steg tills alla tal på listan har antingen en cirkel kring de eller har en linje ritat över de.
\end{enumerate}
Då man har tillämpat klart Eratosthenes ursprunglig såll, blir alla tal med cirklar kring sig alla primtal som är mindre än övergränsen man valde. Även om det har gått nästan två årtusenden sedan upptäckten av Eratosthenes ursprunglig metod har metoden fortfarande samband med nutidens matematiska såll; samband som en fokus på delbarhet och, följaktligen, specifika modulo klasser av primtal. 

Det är dessa nya matematiska sålltekniker och deras tillämpningar på blandade problem samt numerisk implementering som vår rapport kommer att fokusera på. Vi kommer att hålla oss till att redovisa hur tre olika sållmetoder, nämligen Eratosthenes allmänna såll, Bruns såll, och Selbergs såll, kan tillämpas på XXX och YYY, samt kommer vi redovisa hur man kan implementera i kod ZZZs såll. Vi kommer dessutom att diskutera möjliga förbättringar av både av dessa undersökningar vi har gjort. Men innan vi kan börja med att redovisa vårt arbete kommer vi först att nämna några förkunskaper som en läsare borde ha, sedan kommer vi introducera några grundläggande begrepp och teori inom sållteori, och slutligen kommer vi att introducera mer fullständigt de sållen vi har valt att jobba med.