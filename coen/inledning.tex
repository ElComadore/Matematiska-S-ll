% \documentclass[a4]{article}
% \usepackage{graphicx}
% \usepackage{amsmath, amssymb}
% \usepackage{epsfig}
% \usepackage{floatflt} %för inkapslade bilder.

% \addtolength{\textwidth}{10mm}
% \addtolength{\textheight}{30mm}
% \addtolength{\headheight}{-10mm}

% \usepackage[T1]{fontenc}                % För svenska bokstäver
% \usepackage[swedish]{babel}             % För svensk avstavning och svenska
%                                         % rubriker (t ex "innehållsförteckning)

%Var noggranna med att ange källor till det ni skriver. Vi rekommenderar Vancouver-systemet\footnote{Även annat system accepteras om det används konsekvent.} som är mest använt på MV. Man kan antingen använda siffror [1], [2] etc, eller initialer som associerar till författarnamnet(n) t.ex [BN], [BS] etc. Det senare kan vara lite jobbigt om man har många källor men praktiskt om man har några som huvudreferenser. Läs mer i Fackspråks skrift: Utformning av rapporter och kandidatarbetens skriftliga .... (2008-01-11)\cite{rapp}. I det fall arbetet i huvudsak bygger på en eller ett par källor och det är svårt att identifiera exakt när man använder respektive källa, kan man tala om detta i inledningen. Man kan sedan referera till källan om man återger en definition, en sats eller ett bevis eller på annat sätt ligger nära källan. En direkt översättning kan jämställas med ett citat, återberätta därför som om ursprunget var en skrift på svenska så att ni håller er långt ifrån gränsen för plagiering. Är ni osäkra på något så fråga examinator eller handledare.\footnote{Läs mer om att hantera källor och akademisk hederlighet på Chalmers webbsida \hfill \\ URL: https://writing.chalmers.se/chalmers-skrivguide/att-hantera-kallor  }

\begin{comment}
Förutom att vara klassiska frågor inom talteori angående primtal, vad har primtalstvillingshypotesen, Goldbach hypotesen, och storleken på primtalsgap att göra med varandra? Kort sagt; sållteori. Lite längre sagt; under den senaste århundrade har tekniker inom sållteori utvecklades och tillämpades på dessa, och fler, blandade problem inom talteori, och även vidare ämnen, med relativt stort succé. Med hjälp av sållteori har matematiken lyckats bevisa att det finns oändligt många tal \textit{p} sådan att \textit{p} och \(p+2\) är antingen prima eller semiprima \cite{chen2Prime} och att det det finns alltid två primtal inom 600 heltal av varandra \cite{mayBound}. Dessa tillämpningar väcker frågan, vad är ett matematisk såll för något? 

Ett matematisk såll är en metod med sin ursprung i talteori som försöker att uppskata kardinaliteten av en så kallad siktad mängd där alla element i mängden har någon gemensam egenskap. Ett prototypisk exempel på en sådan mängd som man är intresserad i storleken av är mängden som består av alla primtal mindre än något tal \textit{x}. Sållteori har försökt att hitta och förfina uppskattningar av storleken på precis den mängden sedan Chebychev's berömd uppskattning i 1851. Dock har matematiska såll sin ursprung ännu längre sedan i antikens grekland med arbetet av Erastostenes. Hans idé följer nedanstående mönster;
\begin{enumerate}
    \item Med början vid 2, lista ut alla tal upp till talet du vill använda som övergräns.
    \item Rita en cirkel kring 2 och stryk över alla tal delbar med 2.
    \item Rita en cirkel kring det nästa talet som inte har en linje genom sig och dra en linje igenom alla tal delbar med den.
    \item Upprepa föregående steg tills alla tal på listan har antingen en cirkel kring de eller har en linje ritat över de.
\end{enumerate}
Då man har tillämpat klart Eratosthenes ursprunglig såll, blir alla tal med cirklar kring sig alla primtal som är mindre än övergränsen man valde. Även om det har gått nästan två årtusenden sedan upptäckten av Eratosthenes ursprunglig metod har metoden fortfarande samband med nutidens matematiska såll; samband som en fokus på delbarhet och, följaktligen, specifika modulo klasser av primtal. 

Det är dessa nya matematiska sålltekniker och deras tillämpningar på blandade problem samt numerisk implementering som vår rapport kommer att fokusera på. Vi kommer att hålla oss till att redovisa hur tre olika sållmetoder, nämligen Eratosthenes allmänna såll, Bruns såll, och Selbergs såll, kan tillämpas på XXX och YYY, samt kommer vi redovisa hur man kan implementera i kod ZZZs såll. Vi kommer dessutom att diskutera möjliga förbättringar av både av dessa undersökningar vi har gjort. Men innan vi kan börja med att redovisa vårt arbete kommer vi först att nämna några förkunskaper som en läsare borde ha, sedan kommer vi introducera några grundläggande begrepp och teori inom sållteori, och slutligen kommer vi att introducera mer fullständigt de såll vi har valt att jobba med.
\end{comment}

Nästan alla som har även ett ytligt intresse i primtal har, någon gång, börjat skriva en lista av naturliga tal och försökt markera vilka av de är prima eller inte. Då man har gjort det i en liten stund kanske märker man att för att hitta alla primtal upp till ett visst tal behöver man bara alla primtal mindre eller lika med kvadratroten ur det talet. Man kanske också börjar lägga märke till mönster som uppträder; såsom att det finns vissa par av primtal som har bara ett tal mellan de. Idén bakom det som man har precis hållit på med, att hitta primtal i en lista av naturliga tal, har funnits sedan antikens grekland. En algoritm som returnerar alla primtal i en sådan lista har tillskrivits den grekiska polyhistorn Eratosthenes (ca. 276 - 194 f.v.t.), och har följande struktur;
\begin{enumerate}
    \item Med början vid 2, lista ut alla naturliga tal upp till talet du vill använda som övergräns.
    \item Rita en cirkel kring 2 och stryk över alla tal delbar med 2.
    \item Rita en cirkel kring det nästa talet som inte har en linje genom sig och dra en linje igenom alla tal delbar med den.
    \item Upprepa föregående steg tills alla tal på listan har antingen en cirkel kring de eller har en linje ritat över de. 
\end{enumerate}
Då man har kört klart Eratosthenes algoritm är alla tal som har cirklar kring sig prima. Eratothenes algoritm skapade grundläggningen för nutidens sållteori; ett område inom matematiken som försöker uppskatta storkleken på, så kallad, \textit{siktade mängder}. En siktad mängd är en mängd där alla element är heltal och har något gemensam egenskap t.ex. en mängd som består av endast primtal eller mängden av alla heltals lösningar till en ekvation. Grundläggande sållteorins störst fördel är att den är ganska elementär och flexibel, speciellt jämfört med några andra metoder inom analytisk talteorin. Det krävs inga idéer från komplex analys som t.ex. Dirichlets L-funktioner eller serier och, då man kan formulera vikterna på ett korrekt sätt, går det att tillämpa på vilken mängd som helst. Dock så kan dessa matematiska såll fortfarande ge vettiga resultat, även om deras noggrannhet måste offras lite. En exempel på detta är att asymptotiska beteendet av \(x/\log(x)\) för antalet primtal mindre än \textit{x} som ges i primtalssatsen kan nästan bevisas utan något arbete med zeta funktioner då man använder Selbergs såll, dock så får man bara en övre gräns av \(x/\log(x)\) istället för asymptotiska beteendet. Mer avancerade sållmetoder har hittills lyckats att svara på frågor angående primtalstvillingshypotesen i \cite{chen2Prime}, och storleken på primtalsgap mellan ett antal primtal i rad i \cite{mayBound}.

Vår rapport kommer att fokusera på små, kombinatoriska såll. Små innebär att de fokuserar på att uppskatta storleken på få antal restklasser modulo primtal när de siktar en mängd, och att ett såll är kombinatorisk innebär att den använder sig av inklusion-exklusionsprincipen för att dela upp en mängd på ett lämpligt sätt. Bland dessa typer av såll kommer vi håller oss till Eratosthenes allmänna såll, Bruns såll, och Selbergs såll. För varje såll kommer vi försök att ge en naturlig härledning till deras formulering och en förklaring till hur de används. Efteråt, kommer vi att redovisa och analysera en implementering i kod av Eratothenes algoritm, vilket följer metoden som beskrivs i \cite{HaraldSieve}.