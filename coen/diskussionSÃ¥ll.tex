%As mentioned in the introduction, sieve methods, while useful, often require quite a bit of work or refinement in order to achieve the results given by more analytic methods. The central reason for this, as the reader may have noticed throguh the course of this report, is due to the error terms attached to the methods. It is a recurring theme of sieve theory in general to attempt to better these error terms in order to minimise the apprroximations which have to be made. The reader need only recall the work done in each of the sections regarding the applications of the sieves where one of the final steps  was to find a z which made the main and error term have equal weight. Out of the sieve methods we have presented the starkest difference in size of error term is between the generalised sieve of eratosthenes and the other two sieves. To see this difference we present one final simple example.

Som nämndes i inledningen kan sållmetoder, även om de är användbara, kräva en stor del arbete eller förfining för att spegla de resultat vilka ges av mer analytiska metoder. 
En uppenbar orsak till  detta är feltermerna kopplade till de sållmetoder som används. 
Det är ett återkommande tema i sållmetoder som helhet att försöka minska dessa feltermer för att minimera de slutliga uppskattningarna.
Feltermens påverkan på dessa uppskattningar kan enkelt upptäckas i alla tillämpningsdelar i de föregående avsnitten.
Den har störst påverkan i det kritiska steget då \textit{z} bestäms till någon funktion av \textit{x} vilket i sin tur förenar huvudtermen och feltermen.
Av de sållmetoderna vi har redovisat är skillnaden mellan feltermerna störst mellan Eratosthenes generaliserade såll och de andra två sållmetoderna.
För att visa påverkan av denna skillnad redovisar vi ett sista exempel.

%Consider the problem of estimate the number of primes less than x, one fo the original muses for much of this theory. Assuming for brevities sake that all the conditions are upheld we see that this is a linear sieving of all n less than x by the n equiv 0 mod p. Choosing y = x, as there is no need to sieve by nubers larger than x, and kappa = 1 gives the error term to be of the order O(x(log z)^2exp(-log x/logz)).Naturally this erro term is quite large indeed being even directly dependent on the lenght of the interval we are sieving as well as how many primes we are using. For comparison, using Brun's sieve with b=1, we find that a suitable lambda  to be labda = ln(1.31) which gives the power of the z in the error term to be less than or equal to 9. Furtherore using Selberg's sieve with #A = ... we find that the error term is once again of the order of a power of z, this time O(z^2). The difference between the erros of the sieves of Brun and Selberg can be quite easily seen however how they compare to that of Eratosthenes is not as obvious. The size of the main terms given by each of these sieves is the same, namely x /log z, which is unsurpising when you consider that all the sieves seeek to estimate the same cardinality, and choosing z = x^1/u in both Bruns and Selbergs sieves, with u > 10 and u > 2 respectively, leads to results whioh reflect are of the order of << x/log x, with the main difference being the size of the implied constant. However if we were to make the same choice of z for the sieve of eratosthenes we would run into an issue where the error term has more weight than the main term and as such must choose a different z, namely log z = logx/Cloglogx for some small positive constant C. This choice of z returns an estimate of pi(x) of the order of << xloglogx/logx, which is naturally worse than those given by the other sieves.  This basic example shows the effects of the error terms on the quality of estimates which can be made using the sieve methods. This then naturally leads to the question of what causes these errors in approximation and how can we go about reducing them?

Låt oss nu återgå till en av sållteorins musor som är att uppskatta storleken på \(\pi(x)\).
Som nämndes i \ref{era.Legendres} motsvarar problemet en linjär sållning av \(\A = \{n\leq x: n\in\mathbb{N}\}\) med \(\A_d = \{n\leq x: n \equiv 0 \pmod{d}\}\). Låt oss först anta att alla kriterium är uppfyllda för att tillämpa Eratosthenes generaliserade såll på problemet. 
Vi väljer \(y = x\) eftersom vi behöver sålla bara med tal mindre än \textit{x}, och \(\kappa = 1\) eftersom vi betraktar bara ett restklass av alla primtal. 
Med detta val av parametrar erhåller vi en felterm av storleksordning \(O(x(\log z)^2\exp({-\log x/\log z}))\).
Detta är ganska stort eftersom det beror direkt på intervallets längd, men får att se hur mycket den påverkar slutliga uppskattningen jämför vi den emot feltemerna för både Bruns och Selbergs såll, och vad för slutsats man kan dra med de metoder.
För att tillämpa Bruns såll anta igen att vi uppfyller alla kriterium. Då sätter vi \(b = 1\) och, återigen, \(\kappa = 1\). 
Vi också sätter \(\lambda = \log 1.30\), för att minska potensen av feltermen och vi får att \(c_3 \leq 9\) och storleksordningen på feltermen är \(O(z^9)\).
För att tillämpa Selbergs såll inser vi att \(\#\A_d = x/d + O(1)\) vilket medför att feltermen i Selbergs såll av storleksordning \(O(z^2)\), precis som i tillämpningen i avsnitt \ref{sel.apl}.
Skillnaden mellan felen i Brun och Selbergs såll kan ses lätt men hur de jämförs med Eratosthenes är inte lika uppenbart.
För att förstå det behöver vi blanda in huvudtermen, vilket är av storleksordning \(O(x/\log z)\) för alla tre sållmetoder.
Att samma storleksordning erhålls från alla tre metoder är inte konstigt eftersom alla försöker uppskatta samma kardinalitet.
Vi kan förena huvud och feltermen för både Bruns och Selbergs såll genom att sätta \(z = x^{1/u}\), med \(u > 9\) respektive \(u > 2\) och både ger uppskattningen \(\pi(x) \ll x/\log x\), där ordo konstanten som fås skiljer sig mellan metoderna. 
Dock så kan vi inte välja \textit{z} på samma stil när vi tillämpar Eratosthenes allmänna såll.
I det fallet måste vi välja \(\log z = \log x/C\log\log x\) för något liten konstant \textit{C} sådan att feltermen och huvudtermen har samma vikt.
Då vi gör det så erhåller vi från Eratosthenes allmänna såll att \(\pi(x) \ll x\log\log x/\log x\).
Att både Brun och Selbergs såll når resultat som börjar likna primtalssatsen men inte Eratosthenes generaliserade såll  exemplifierar hur feltermerna påverkar kvalitén av slutsatsen man kan dra med de olika metoderna.
Detta leder då naturligtvis till frågan om vad som orsakar dessa fel i uppskattning och vad vi kan göra för att minska dem.


%Each of the sieve methods have various different factors which influence the size of their error terms. However, there is one commnon factor between the three sieves which is worth mentioning, that of how the methods use the Moebius function to facilitate estimations cardinalities. We have as a baseline the generalised sieve of earatosthenes where the almost no use is made of the Moeebius function to reduce the error terms size, given that the funciton is directly estimated with \mu(d)\ leq 1. In Bruns sieve however, the first equality in chapter 5 has its roots in a clever use of Moebius inversion alongside the Moebius function to allow the construction of that equality, and therefore the subsequent estimates. Selbergs ssieve makse the greatest use of the Moeebius function with its direct estimation of a sum of Moebius functions as outlined in chapter 6. It is telling of the difficulty of the Moebius function that the sieve method which attempts to remove its dependence on the Moebius function is the method which often gives the best result. Indeed, according to Tao, the Moebius function is the root of a even more central concern of  sieve theory than that of minimising the error terms, namely the parity problem. this problem in esssence states that for specific types of sets sieve methods will either gives over bounds which are too large my a factor of at least two, or under bounds which are trivia. This can be seen to be due to the binary nature of the moebius function where truncating the function leads to too many elemetns being removed or too many ebing added back. Modern sieve methods such as those developed by Friedland and Iwaniec have attempted to be "parity sensitive" for certain sets of primes and has met success in showing that there are infinitely many primes of the form a^2 + b^4.

Var och en av sållmetoderna har olika faktorer som påverkar storleken på deras feltermer.
Det finns dock åtminstone en gemensam faktor mellan de tre metoderna som är värt att nämna, vilket är hur de olika metoderna använder Möbiusfunktionen för att underlätta uppskattningar av kardinaliteter.
Som baslinje har vi Eratosthenes allmänna såll som uttnytar nästan inga aspekter hos Möbiusfunktionen då den uppsakttar kardinaliteten förutom att \(|\mu(d)| \leq 1\) då vi kommer till feltermen.
Dock har Bruns såll större användning av Möbiusfunktionens egenskaper eftersom \eqref{brun.eq.firstsum} kan formuleras på ett sådant sätt på grund av Möbiusfunktionen och en tillämpning av Möbius inverteringsformeln på den och funktionen \textit{g}.
Selbergs såll går ett steg vidare och baserar en stor del av hela sin härledning på en uppskattning av en summa av Möbiusfunktioner.
Att sållet som försöker hårdast att undvika Möbiusfunktionen är oftast metoden som ger bäst resultat av de tre påpekar hur svårhanterat Möbiusfunktionen är.
I själva verket, enligt Tao \cite{Tao}, är Möbiusfunktionen roten till ett ännu mer centralt utmaning för sållteorin än att minimera feltermer, nämligen paritetsproblemet.
Detta problem säger i huvudsak att för specifika typer av mängder kommer sållmetoder antingen att ge övre gränser som är för stora med faktor på minst två eller undre gränser som är triviala.
Detta kan ses bero på den binära karaktären hos Möbiusfunktionen vilket leder till att för många kardinaliteter tas bort eller för många läggs tillbaka då man trunkerar Möbiusfunktionen, det vill säga uppskattar den.
Moderna sållmetoder såsom de utvecklade av Friedlander och Iwaniec \cite{abPrimes} försöker undvika paritetsproblemet för vissa mängder av primtal.
Deras metoder har haft succé med att visa att det finns oändlig många primtal på formen \(a^2+b^4\).


%Other challenges sieve theory has met under its development include the sieveing by mor than a few number of residue classes of prime numbers. The sieves we have presented in the report work best when a small number of residue classes are to be sieved out however advances in the theory have allowed up to half the total of residue classes to be removed. The large sieve as it was called was published around the same time as Selbergs work by Linnik, and uses very different mathematical tools to allow the sieving by up to half of the total number of prime numbers. This sieve has met great succes in showing the Bombieri/Vinogradox theorem, a cornerstone to modern daay sieve theory regarding the distirbution of primes in arithmetic sequences averaged over a range of moduli. Work by Gallagher later simplified much of the sieve and extended it to sieve by arbitrarily many residue calsses.
