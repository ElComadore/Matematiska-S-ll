% Skrivet av Erik

I den här texten kommer vi att låna större delen notationen från \cite{cojocarumurty}, exempelvis skriver vi \(p, q, l\) när vi menar primtal, \(n, d, k\) för naturliga tal och \(x, y, z \in \mathbb{R}_+\). Den största gemensamma delaren noteras \(\gcd{d, k}\) och funktionen \(\nu(d)\) beskriver antalet distinkta primtalsdelare av \(d\). Nedanstående sektioner ämnar till att etablera mer notation och tekniker som är vanligt förekommande i sållteori med utgångspunkt i Cojocarus och Murtys bok.

\subsection{Det allmänna sållproblemet}
Det allmänna fallet i sållteori är att vi har en mängd heltal \(\A\), en mängd primtal \(\P\) samt delmängder \(\A_p, \forall p \in \P\) och är intresserade av att sålla fram delmängden \(\mathcal{S}(\A, \P) := \A \setminus \cup_{p \in \P} \A_p\) eller kardinaliteten av dito, \(S(\A, \P) :=\card{(\A \setminus \cup_{p \in \P} \A_p)}\). För \(d\), en kvadratfri produkt (dvs. utan delare \(p^\alpha\) för \(\alpha > 1\)) av element \(p \in \P\) definierar vi \(\A_d = \cap_{p \divides d} \A_p\) och \(\A_1 = \A\). Om \(P(z) = \prod_{\substack{p\in \mathcal{P} \\ p < z}} p\) så låter vi
\begin{align*}
    \mathcal{S}(\A, \P, z) := (\A \setminus \cup_{p \divides P(z)} \A_p)
    \quad \text{och} \quad
    \S{A}{P}{z} := \card{(\A \setminus \cup_{p \divides P(z)} \A_p)} .
\end{align*}

En gemensam teknik för de matematiska sållen diskuterade nedan är att låta \(X\) beteckna kardinaliteten av \(\A\) och hitta \(\delta_p\) så att
\begin{align*}
    \card{\A_p} = \delta_p X + R_p
\end{align*}
där \(\delta_p \in [0, 1)\) kan betraktas som en uppskattning av andelen element i delmängden \(\A_p\) och \(R_p\) som en felterm. För \(p \neq q\) låter vi \(\delta_{pq} = \delta_p \delta_q\) så att \(\card{\A_{pq}} = \delta_p \delta_q X + R_{pq}\). Dessutom används, som en teknisk detalj, att \(R_{p,p} = R_p\).


\subsection{Möbiusfunktionen}
En användbar funktion i sållteori är möbiusfunktionen,
\begin{equation*}
    \mu(n) = 
    \begin{cases}
        1, & \text{om}\ n \text{ är ett kvadratfritt, naturligt tal med jämnt antal primdelare}\\
        -1, & \text{om}\ n \text{ är ett kvadratfritt, naturligt tal med udda antal primdelare}\\
        0, & \text{om}\ n \text{ inte är kvadratfri}
    \end{cases}
\end{equation*}
som bland annat förenklar inklusion-exklusionsprincipen så att, för scenariot ovan,
\begin{align*}
    \S{A}{P}{z} = \sum_{\substack{d \divides P(z) \\ d \leq z}} \mu(d) \card{\A_d} .
\end{align*} % sida 72

Två andra viktiga egenskaper hos möbiusfunktionen är den så kallade fundamentala egenskapen,
\begin{equation*}
    \sum_{d \divides n} \mu(d) =
    \begin{cases}
        1, & \text{if}\ n = 1 \\
        0, & \text{if}\ n > 0
    \end{cases}
\end{equation*}
och Möbius inverteringsformel som säger
\begin{equation*}
    f(n) = \sum_{d \divides n} g(d) \implies g(n) = \sum_{d \divides n} \mu(d) f(n/d)
\end{equation*}
om \(f, g : \mathbb{N} \to \mathbb{C}\).

%\subsection{O-notation}
% Intresserade av en övre asymptotisk gräns --> förklara Ordonotation. 

%Matematisk sållteori är en underkategori av analytisk talteori. Ett av våra viktigaste analytiska verktyg i den här uppsatsen är Ordo-notationen. 

%\subsection{Abels summationsformel}