% Skrivet av Erik

%\subsection{O-notation}

\subsection{Det allmänna sållproblemet}
% Kanske kan koppla till inklusions-exklusionsprincipen. 
Det generella fallet i sållteori är att vi har en mängd heltal \(\A\), en mängd primtal \(\P\) samt delmängder \(\A_p, \forall p \in \P\) och är intresserade av att sålla fram delmängden \(\mathcal{S}(\A, \P) := \A \setminus \cup_{p \in \P} \A_p\) eller, vanligare, kardinaliteten av dito. För \(d\), en kvadratfri produkt (dvs. utan delare \(p^\alpha\) för \(\alpha > 1\)) av element \(p \in \P\) definierar vi \(\A_d = \cap_{p \divides d} \A_p\). Om \(P(z) = \prod_{\substack{p\in \mathcal{P} \\ p < z}} p\) så betecknar vi kardinaliteten av den sållade mängden som
\begin{align*}
    \S{A}{P}{z} = \card{(\A \setminus \cup_{p \divides P(z)} \A_p)} .
\end{align*}


%\subsection{O-notation}
% Intresserade av en övre asymptotisk gräns --> förklara Ordonotation. 

\subsection{Möbiusfunktionen}
En användbar funktion i sållteori är möbiusfunktionen,
\begin{equation*}
    \mu(n) = 
    \begin{cases}
        1, & \text{om}\ n \text{ är ett kvadratfritt, naturligt tal med jämnt antal primdelare}\\
        -1, & \text{om}\ n \text{ är ett kvadratfritt, naturligt tal med udda antal primdelare}\\
        0, & \text{om}\ n \text{ inte är kvadratfri}
    \end{cases}
\end{equation*}
som bland annat förenklar inklusion-exklusionsprincipen så att, för scenariot ovan,
\begin{align*}
    \S{A}{P}{z} = \sum_{\substack{d \divides P(z) \\ d \leq z}} \mu(d) \card{\A_d} .
\end{align*} % sida 72

Två andra viktiga egenskaper hos möbiusfunktionen är den så kallade fundamentala egenskapen,
\begin{equation*}
    \sum_{d \divides n} \mu(d) =
    \begin{cases}
        1, & \text{if}\ n = 1 \\
        0, & \text{if}\ n > 0
    \end{cases}
\end{equation*}
och Möbius inverteringsformel som säger
\begin{equation*}
    f(n) = \sum_{d \divides n} g(d) \implies g(n) = \sum_{d \divides n} \mu(d) f(n/d)
\end{equation*}
om \(f, g : \mathbb{N} \to \mathbb{C}\).


%\subsection{Abels summationsformel}