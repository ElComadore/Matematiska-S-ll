Avsnitt \ref{era.Legendres} introducerar dimensionen av ett såll, \(\kappa\), som skulle uppfylla (\ref{dimension}). Vi visar här ett resultat tillskrivet Tjebysjov, hämtat ur \cite[Sats 1.4.3]{cojocarumurty}.
\begin{theorem} \label{APDX:THM1.4.3}
    $\sum_{p \leq n} \frac{\log p}{p} = \log n + O(1) $.
\end{theorem}
\begin{proof}
Nyckeln till beviset av ovanstående sats i \cite{cojocarumurty} är två observationer om fakultetfunktionen, \(n! = 1 \cdot 2 \cdot ... \cdot n\). Först ser vi att om vi vill omformulera \(n!\) som en produkt av primtalspotenser så ser vi att den bidragande faktorn för varje $p \leq n$ kan formuleras som 
\begin{align*}
    \prod_{\substack{p^\alpha \pdiv k \\ k \leq n}} p^\alpha = \prod_{p^\alpha \leq n} p^{\lfloor n/p^{\alpha} \rfloor} = p^{\lfloor n/p \rfloor + \lfloor n/p^2 \rfloor + ... + \lfloor n / p^{\lfloor \log n / \log p \rfloor} \rfloor}
\end{align*}
där \(p^\alpha \pdiv k\) betyder att \(\alpha\) är den största heltalsexponenten så att $p^\alpha$ delar $k$. Detta ger oss att 
\begin{align*}
    \log n! = %\log\left( \prod_{p \leq n} p^{\lfloor n/p \rfloor + \lfloor n/p^2 \rfloor + ... + \lfloor n / p^{\lfloor \log n / \log p \rfloor} \rfloor} \right) =
    \sum_{p \leq n} \left(\left\lfloor\frac{n}{p} \right\rfloor + \left\lfloor \frac{n}{p^2} \right\rfloor + ... + \left\lfloor n / p^{\lfloor \log n / \log p \rfloor} \right\rfloor\right) \log p
\end{align*}
vari en restterm kan urskiljas och uppskattas till
\begin{align*}
    \sum_{p \leq n} \left(\left\lfloor \frac{n}{p^2} \right\rfloor + ... + \left\lfloor n / p^{\lfloor \log n / \log p \rfloor} \right\rfloor\right) \log p \leq
    \sum_{p \leq n} \left( \frac{n}{p^2}  \sum_{k=0}^\infty \frac{1}{p^k} \right) \log p
    = n \sum_{p \leq n} \frac{1}{p^2} \cdot \frac{\log p}{1 - 1/p} \ll n.
\end{align*} % Det här kan förbättras
Den första observationen är alltså att
\begin{align} \label{APDX:obser1Thm1.4.3}
    \log n! = n \sum_{p \leq n} \frac{\log p}{p} + O(n).
\end{align}

Den andra observationen ser vi genom att utföra en partiell summation,
\begin{align*}
    \log n! = \log \Big(\prod_{k \leq n} k\Big) = \sum_{k \leq n} 1 \cdot k = 
    \lfloor n \rfloor \log n - \int_1^n \frac{\lfloor t \rfloor}{t} \text{d} t 
\end{align*}
och att \(\lfloor t \rfloor = t + O(1)\) ger att detta är lika med
\begin{align} \label{APDX:obser2Thm1.4.3}
    (n + O(1)) \log n - \int_1^n \frac{t}{t} \text{d} t + O(1) \int_1^n \frac{1}{t} \text{d} t  = n \log n - n + O(\log n).
\end{align}

Sätter vi samman observationerna, (\ref{APDX:obser1Thm1.4.3}) och (\ref{APDX:obser2Thm1.4.3}), så får vi 
\begin{align*}
    n \sum_{p \leq n} \frac{\log p}{p} + O(n) &= n \log n - n + O(\log n) \\
    n \sum_{p \leq n} \frac{\log p}{p} &= n \log n + O(n)
\end{align*}
vilket efter division med $n$ ger det önskade resultatet.
\end{proof}

När vi hanterar Eratosthenes generaliserade såll, sats \ref{eratosthenes.gen.såll}, och Bruns såll, sats \ref{brun.thm.brun}, resulterar det ofta i att vi vill uppskatta \(W(z)\)-funktionen. En användbar sats för detta är \cite[Sats 1.4.4]{cojocarumurty} som säger
\begin{theorem} \label{APDX:THM1.4.4}
    \(\sum_{p \leq n} \frac{1}{p} = \log n + O(1)\).
\end{theorem}
\begin{proof}
Låt
\begin{align*}
    c_k := 
    \begin{cases}
    \log p \quad \text{för } k = p \\
    0 \quad \text{annars}
    \end{cases}
\end{align*}
då får vi att \(S(x) = \sum_{k \leq x} c_k\) är lika med summan i sats \ref{APDX:THM1.4.3}. Om vi använder föregående sats efter en partiell summation så ser vi att
\begin{align*}
    \sum_{p \leq n} \frac{1}{p} &= \frac{S(n)}{\log n} + \int_2^n \frac{S(t)}{t(\log t)^2} \text{d}t 
    = \frac{\log n + O(1)}{\log n} + \int_2^n \frac{\log t}{t(\log t)^2} \text{d}t + \int_2^n \frac{O(1)}{t(\log t)^2} \text{d}t \\
    &= O(1) + \left[\log \log t \right]_2^n + O(1) \left[- \frac{1}{\log t} \right]_2^n
    = \log \log n  + O(1).
\end{align*}

\end{proof}