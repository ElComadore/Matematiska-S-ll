% Skrivet av Erik

I avsnitt \ref{partB.algoritmteori}, där vi går igenom en implementering av Eratosthenes såll, så gör vi bruk av en \textit{diofantisk approximation}, \cite[algoritm 4]{HaraldSieve}. För att förstå det här steget krävs först lite förkunskaper om kedjebråk. 

Ett \textit{ändligt kedjebråk} definieras i \cite[definition 20.1]{Lindahl} som
\begin{align*}
    \langle a_0, a_1, ..., a_n \rangle := a_0 + \cfrac{1}{a_1 + \cfrac{1}{a_2 + \cfrac{1}{\ddots \raisebox{-3mm}{$a_{n-2}+\cfrac{1}{a_{n-1} + \cfrac{1}{a_n}}$}}}}
\end{align*}
där \(a_0, a_1, ..., a_n\) är reella tal och \(a_i > 0\) för \(i > 0\). I den här uppsatsen berör vi endast särfallet då \(a_0, a_1, ..., a_n\) är heltal vilket då kallas för ett \textit{enkelt kedjebråk}. 

En omedelbar utmaning är att, givet ett kedjebråk \(\langle a_0, a_1, ..., a_n \rangle \), beräkna \(\langle a_0, a_1, ..., a_{n+1} \rangle \) utan att räkna om hela kedjebråket från term \(a_{n+1}\) till \(a_0\). Lösningen på det här problemet är så kallade \textit{konvergenter}, \cite[Definition 20.4]{Lindahl}, och definieras som ett par \((p_n,q_n)\), där
\begin{align*}
    p_{-2} &= 0, p_{-1} = 1, p_n = a_n p_{n-1} + p_{n-2} \text{ då } n \geq 0 \\
    q_{-2} &= 1, q_{-1} = 0, q_n = a_n q_{n-1} + q_{n-2} \text{ då } n \geq 0,
\end{align*}
och deras kvot, \(c_n := p_n / q_n\), \(n \geq 0\). Observera att definitionen ger oss att \(q_0 = 1\) och \((q_n)_{n=1}^{N}\) är en strikt växande följd av heltal om motsvarande kedjebråk är enkelt. Anledningen till att vi introducerar konvergenter ges av följande sats, \cite[Sats 20.5i) och ii)]{Lindahl},
\begin{theorem} \label{app.konvergenter}
    Låt \((a_n)_{n=0}^{N}\) vara en följd av reella tal där \(a_i > 0\) för \(i > 0\) och \((p_n, q_n)\) är deras respektive konvergenter. Då gäller att
    \begin{enumerate}
        \item \(\langle a_0, a_1, ..., a_n \rangle = c_n\) för alla \(n \geq 0\),
        \item \(p_n q_{n-1} - p_{n-1} q_n = (-1)^{n-1}\) för alla \(n \geq -1\).
    \end{enumerate}
\end{theorem}
\begin{proof}
    Vi följer här beviset från \cite{Lindahl}.
    
    (i): För att visa det första påståendet använder vi ett induktionsargument. 
    Vi ser att påståendet håller för $n = 0$ ty \(\langle a_0 \rangle = a_0\) och \(c_0 = p_0 / q_0 = (a_0 \cdot 1 + 0) / 1 = a_0\).
    
    Antag nu att påståendet gäller för kedjebråk med $n$ element. Vi ska visa att detta medför att \(\langle a_0, a_1, ..., a_n, a_{n+1} \rangle = c_{n+1}\). Vi gör detta genom skriva om det sista elementet i kedjebråket så att vi får samma kedjebråk fast med $n$ element,
    \begin{align} \label{APDX:conv.ind}
        \langle a_0, a_1, ..., a_n, a_{n+1} \rangle = \langle a_0, a_1, ..., a_n + 1 / a_{n+1} \rangle .
    \end{align}
    Låt $a_n'$ beteckna det nya kedjebråkets $n$:te element. Det nya kedjebråket har då samma konvergenter upp till $n$. Vi skriver den sista konvergenten som $c_n' = p_n'/q_n'$. Alltså, per induktionsantagandet, har vi
    \begin{align*}
        c_{n}' &= \frac{p_n'}{q_n'} = \frac{\left( a_{n} + 1/a_{n+1} \right) p_{n-1} + p_{n-2}}{\left( a_{n} + 1/a_{n+1} \right) q_{n-1} + q_{n-2}} \\
        %&= \frac{\left( a_{n+1} a_{n} + 1 \right) p_{n-1} + a_{n+1} p_{n-2}}{\left( a_{n+1} a_{n} + 1 \right) q_{n-1} + a_{n+1} q_{n-2}} \\
        &= \frac{ a_{n+1} \left( a_{n} p_{n-1} + p_{n-2} \right) + p_{n-1}}{a_{n+1} \left( a_{n} q_{n-1} + q_{n-2} \right) + q_{n-1}}\\
        &= \frac{ a_{n+1} p_n + p_{n-1}}{a_{n+1} q_n + q_{n-1}} 
    \end{align*}
    vilket är definitionen av \(c_{n+1} = p_{n+1}/q_{n+1}\). Därav ger (\ref{APDX:conv.ind}) att
    \begin{align*}
        \langle a_0, a_1, ..., a_n, a_{n+1} \rangle = c_{n+1}
    \end{align*}
    
    Det följer nu av induktion att \(\langle a_0, a_1, ..., a_n \rangle = c_n\) för alla \(n \geq 0\).
    
    (ii): För det andra påståendet gör vi ett till induktionsbevis. Vi börjar den här gången med att kolla basfallet $n = -1$,
    \begin{align*}
        p_{-1} q_{-2} - p_{-2} q_{-1} = 1 \cdot 1 - 0 \cdot 0 = (-1)^{-2} 
    \end{align*}
    vilket stämmer med påståendet vi vill visa. 
    
    Härnäst antar vi att \(z_n := p_n q_{n-1} - p_{n-1} q_n = (-1)^{n-1}\) för något $n$ och vill visa att \(z_{n+1} = (-1)^{n}\). Detta är lätt att se med hjälp av den rekursiva definitionen av konvergenterna:
    \begin{align*}
        z_{n+1} = p_{n+1} q_{n} - p_{n} q_{n+1} = (a_n p_n + p_{n-1}) q_n - p_{n} (a_n q_n + q_{n-1}) = p_{n-1} q_n - p_n q_{n-1} = - z_n.
    \end{align*}
    Således ger induktionsantagandet att \(z_{n+1} = (-1) \cdot (-1)^{n-1}\) vilket skulle visas och påståendet följer av induktion. 
\end{proof}

\todo{Koppla samman med sats 20.9}

\begin{theorem} \label{app.kovfel}
    Låt \((a_n)_{n=0}^{N}\) vara en följd av heltal med positiva termer då \(n > 0\) och med konvergenter \(p_n / q_n = c_n\). Kedjebråket \(\xi = \langle a_0, a_1, \dots, a_N \rangle\) uppfyller då för alla \(n \geq 0\)
    \begin{enumerate}
        \item \(c_{2n} < \xi < c_{2n + 1}\) samt
        \item \(\abs{\xi - c_n} < \frac{1}{q_n q_{n + 1}}\) % Undre gräns överflödig?
    \end{enumerate}
\end{theorem}
\begin{proof}
    \todo{Bevisa}
\end{proof}

%Definitionen av ett \textit{oändligt kedjebråk} (\cite[Definition 20.3]{Lindahl}) följer naturligt av föregående definition som
%\begin{align*}
%    \lim_{n \to \infty} \langle a_0, a_1, ..., a_n \rangle
%\end{align*}