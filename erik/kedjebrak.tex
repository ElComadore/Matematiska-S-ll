% Skrivet av Erik

I avsnitt \ref{partB.algoritmteori}, där vi går igenom en implementering av Eratosthenes såll, så gör vi bruk av en \textit{diofantisk approximation}, \cite[Algoritm 4]{HaraldSieve}. För att förstå det här steget krävs först lite förkunskaper om kedjebråk. 

Ett \textit{ändligt kedjebråk} definieras i \cite[Definition 20.1]{Lindahl} som
\begin{align*}
    \langle a_0, a_1, ..., a_n \rangle := a_0 + \cfrac{1}{a_1 + \cfrac{1}{a_2 + \cfrac{1}{\ddots \raisebox{-3mm}{$a_{n-2}+\cfrac{1}{1 + \cfrac{1}{a_n}}$}}}}
\end{align*}
där \(a_0, a_1, ..., a_n\) är reella tal och \(a_i > 0\) för \(i > 0\). I den här uppsatsen berör vi endast särfallet då \(a_0, a_1, ..., a_n\) är heltal vilket då kallas för ett \textit{enkelt kedjebråk}. 

En omedelbar utmaning är att, givet ett kedjebråk \(\langle a_0, a_1, ..., a_n \rangle \), beräkna \(\langle a_0, a_1, ..., a_{n+1} \rangle \) utan att räkna om hela kedjebråket från term \(a_{n+1}\) till \(a_0\). Lösningen på det här problemet är så kallade \textit{konvergenter}, \cite[Definition 20.4]{Lindahl}, och definieras som ett par \((p_n,q_n)\), där
\begin{align*}
    p_{-2} &= 0, p_{-1} = 1, p_n = a_n p_{n-1} + p_{n-2} \text{ då } n \geq 0 \\
    q_{-2} &= 1, q_{-1} = 0, q_n = a_n q_{n-1} + q_{n-2} \text{ då } n \geq 0,
\end{align*}
och deras kvot, \(c_n := p_n / q_n\), \(n \geq 0\). Observera att definitionen ger oss att \(q_0 = 1\) och \((q_n)_{n=1}^{N}\) är en strikt växande följd om motsvarande kedjebråk är enkelt. Anledningen till att vi introducerar konvergenter ges av följande sats, \cite[Sats 20.5i) och ii)]{Lindahl},
\begin{theorem} \label{app.konvergenter}
    Låt \((a_n)_{n=0}^{N}\) vara en följd av reella tal där \(a_i > 0\) för \(i > 0\) och \((p_n, q_n)\) är deras respektive konvergenter. Då gäller att
    \begin{enumerate}
        \item \(\langle a_0, a_1, ..., a_n \rangle = c_n\) för alla \(n \geq 0\),
        \item \(p_n q_{n-1} - p_{n-1} q_n = (-1)^{n-1}\) för alla \(n \geq -1\).
    \end{enumerate}
\end{theorem}


\todo{Skriv sats 20.9}


%Definitionen av ett \textit{oändligt kedjebråk} (\cite[Definition 20.3]{Lindahl}) följer naturligt av föregående definition som
%\begin{align*}
%    \lim_{n \to \infty} \langle a_0, a_1, ..., a_n \rangle
%\end{align*}