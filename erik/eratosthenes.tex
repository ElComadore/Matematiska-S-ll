% Skrivet av Erik

Det klassiska exemplet på ett matematiskt såll, vilket beskrevs i introduktionen, har tillskrivits den grekiska polyhistorn Eratosthenes (ca. 276 - 194 f.v.t.). Idén var raffinerad av Adrien-Marie Legendre 1808 v.t. med hjälp av inklusion-exklusionsprincipen formulerad på formen från avsnitt \ref{sallproblemet}. Nedan kommer vi se hur \cite{cojocarumurty} utvecklar Eratosthenes såll genom användningen av ett trick uppkallat efter matematikern R. A. Rankin och sedan studera en tillämpning av resultatet. 

\subsection{Legendres såll}

Första steget för att utveckla Eratosthenes såll är att omformulera sållexemplet från introduktionen i de termer vi definierade i föregående avsnittet. I exemplet börjar vi med en lista av alla naturliga tal upp till en övre gräns, säg $x$, vilken vi kan skriva som $\A = \{n \in \mathbb{N} : n \leq x\}$. Låt alla primtal upp till $z$ redan vara inringade och mängderna vi kryssar över vara \(\A_p = \{a \in \A :  a \equiv 0 \pmod{p}\}\). Om vi väljer $z = \sqrt{x}$ så blir, med hjälp av (\ref{inclusionexclusion}),
\begin{align*}
    \pi(x) - \pi(\sqrt{x}) + 1 = \sum_{d \divides \P(\sqrt{x})} \mu(d) \left\lfloor \frac{x}{d} \right\rfloor  
\end{align*}
där \(1\):an på vänstersidan tar i hänsyn att \(1 \in \A\) inte sållas bort på högersidan och \(\pi(\sqrt{x})\) är de inringade primtalen. Detta var Legendres idé 1808 när han omformulerade Eratosthenes såll till att räkna primtal 1808 \cite{opera}. 

Eftersom \(\A_p\), för varje $p$, är definierad som en restklass modulo $p$ så säger vi att antalet utvalda restklasser modulo $p$ är $\omega(p) = 1$ för alla $p$ - vi kallar Eratosthenes för ett endimensionellt eller linjärt såll av den här anledningen. Mer allmänt betecknas dimensionen av ett såll med parametern \(\kappa\) om 
\begin{align}
    \sum_{p \divides P(z)} \frac{\omega(p) \log(p)}{p} \leq \kappa \log(z) + O(1)
\end{align}
\todo{Alt. mer lös definition: omega(p) genomsnittligt begränsad av kappa \cite{tenenbaum}}. Vi ser således att ett naturligt nästa steg är att generalisera Eratosthenes-Legendres såll för godtyckliga dimensioner. 

\subsection{Eratosthenes generaliserade såll}

Om vi låter $\A$, $\P$, $P(z)$, $\S{A}{P}{z}$ och $\omega(d)$ vara definierade som i avsnitt \ref{sallproblemet} 

\begin{theorem}[Eratosthenes generaliserade såll]\label{thm:EratosthenesSieve}

\begin{align*}
    \S{A}{P}{z} = X W(z) + O\left(\left(X + \frac{y}{\log z} \right) (\log z)^{\kappa + 1} \exp{\left(-\frac{\log y}{\log z}\right)} \right)
\end{align*}

\end{theorem}
