% Skrivet av Erik

Även om Eratosthenes såll fortfarande är mest känt som en enkel algoritm som utan större möda kan utföras med papper och penna så har dess idé raffinerats genom historien till att inkorperera mer avancerad teori för att tillämpas på nya sätt. Adrien-Marie Legendre 1808 v.t. utvecklade sållet med hjälp av inklusion-exklusionsprincipen formulerad på formen från avsnitt \ref{sallproblemet}. Nedan kommer vi gå igenom denna formen av Eratosthenes såll samt, i efterföljande avsnitt, se hur \cite{cojocarumurty} utvidgar sållet till en mer generell form. 

\subsection{Legendres såll}

Första steget för att utveckla Eratosthenes såll är att omformulera sållexemplet från introduktionen i de termer vi definierade i föregående avsnittet. I exemplet börjar vi med en lista av alla naturliga tal upp till en övre gräns, säg $x$, vilken vi kan skriva som $\A = \{n \in \mathbb{N} : n \leq x\}$. Låt alla primtal upp till $z$ redan vara inringade och mängderna vi kryssar över vara \(\A_p = \{a \in \A :  a \equiv 0 \pmod{p}\}\). Om vi väljer $z = \sqrt{x}$ så blir, med hjälp av (\ref{inclusionexclusion}),
\begin{align*}
    \pi(x) - \pi(\sqrt{x}) + 1 = \sum_{d \divides \P(\sqrt{x})} \mu(d) \left\lfloor \frac{x}{d} \right\rfloor  
\end{align*}
där \(1\):an på vänstersidan tar i hänsyn att \(1 \in \A\) inte sållas bort på högersidan och \(\pi(\sqrt{x})\) är de inringade primtalen. Detta var Legendres idé 1808 när han omformulerade Eratosthenes såll till att räkna primtal 1808 \cite{opera}. 

Legendres formel är bra för att räkna primtal exakt men är långsam och golvfunktionen kan vara svårhanterlig. Vi kan underlätta för oss genom att ersätta golvfunktionen med $x/d$ och dess bråkdel, \(\{x/d\} = O(1)\), som restterm vilket leder till \(\A_d = x/d + O(1)\). Dessvärre resulterar det här i en väldigt stor felterm då restermerna ackumulerar, så att \(\pi(x) - \pi(\sqrt{x}) + 1 = x \sum_{d \divides P(z)}\frac{\mu(d)}{d} + O(2^{\pi(\sqrt{x})}) \).

Eftersom \(\A_p\), för varje $p$, är definierad som en restklass modulo $p$ så säger vi att antalet utvalda restklasser modulo $p$ är $\omega(p) = 1$ för alla $p$ - vi kallar Eratosthenes för ett endimensionellt eller linjärt såll av den här anledningen. Mer allmänt betecknas dimensionen av ett såll med parametern \(\kappa\) om 
\begin{align} \label{dimension}
    \sum_{p \divides P(z)} \frac{\omega(p) \log(p)}{p} \leq \kappa \log(z) + O(1)
\end{align}
\todo{Alt. mer lös definition: omega(p) genomsnittligt begränsad av kappa \cite{tenenbaum}}. Vi ser således att ett naturligt nästa steg är att generalisera Eratosthenes-Legendres såll för godtyckliga dimensioner. 

\subsection{Eratosthenes generaliserade såll}
% "Effektivt" kräver en förklaring, i kontrast mot exaktheten av Eratosthenes såll?
Om vi låter $\A_d$, $\P$, $P(z)$ och $\omega(d)$ vara definierade som i avsnitt \ref{sallproblemet} så kan vi, med utgångspunkt i Eratosthenes och Legendres grundtankar, skapa ett mer flexibelt och effektivt såll. Den stora skillnaden mot vad vi gjorde i föregående avsnitt är att vi här tillåter fler restklasser sållas bort per primtal $p$. 

Utöver vad som står i avsnitt \ref{sallproblemet} kommer vi göra antagandena att \(\card{A_d} = 0\) för alla \(d\) större än något positivt $y$, att $\abs{R_d} = O(\omega(d))$ samt att (\ref{dimension}) håller. Vi kan lätt övertyga oss om att dessa är rimliga antaganden i det klassiska Eratosthenes såll-fallet. 
%Eftersom vi utgår från en ändlig mängd $\A$ så håller det första antagandet då \(\card{\A_p} = 0\) för \(p > X\) innebär att, väldigt grovt, \(y \leq P_{X}\). D
%Det andra antagandet räcker det med summa över $p \leq n$ log p / n $\leq$ log n + O(1) ... och $\abs{R_d} = \abs{ x/d - \lfloor x / d \rfloor} = O(1)$. 
\todo{Skriv färdigt resonemang om rimlighet i antaganden}

Vi börjar återigen med Legendres grundidé, inklusion-exklusionsprincipen, men infogar det mer generella uttrycket för $\card{A_d}$ i formeln,
\begin{align*}
    \S{A}{P}{z} = \sum_{d \divides P(z)} \mu(d) \card{A_d} = X \sum_{\substack{d \divides P(z) \\ d \leq y}} \frac{\omega(d)}{d} \mu(d) + \sum_{\substack{d \divides P(z)  \\ d \leq y}} \mu(d) R_d 
\end{align*}
där trunkeringen av summan kommer av antagandet \(\A_d = 0, \forall d > y\). 

Studerar vi den första summan ovan ser vi att vi kan skriva om den som summan över alla \(d \divides P(z)\) minus summan över alla \(d \divides P(z)\) så att \(d > y\).
%\begin{align*}
%    \sum_{\substack{d \divides P(z) \\ d \leq y}} \frac{\omega(d)}{d} \mu(d) =
%    \sum_{\substack{d \divides P(z)}} \frac{\omega(d)}{d} \mu(d) - \sum_{\substack{d \divides P(z) \\ d > y}} \frac{\omega(d)}{d} \mu(d) .
%\end{align*}
Med observationen att \(W(z) := \prod_{p \divides P(z)} (1 - \omega(p)/p) = \sum_{d \divides P(z)} \frac{\mu(d)}{d} \omega(d)\) då \(\frac{\omega(d)}{d}\) är multiplikativ kan vi således dela upp den siktade mängden i en huvuddel och en rest på följande vis
\begin{align*}
    \S{A}{P}{z} = \left(X W(z) \right) + \left(- X \sum_{\substack{d \divides P(z) \\ d > y}} \frac{\omega(d)}{d} \mu(d) + \sum_{\substack{d \divides P(z)  \\ d \leq y}} \mu(d) R_d \right).
\end{align*} % Saknas X

Vi kan skriva om resttermen, först med hjälp av triangelolikheten och sedan med $\abs{\mu(x)} \leq 1$ och vårt antagande $\abs{R_d} = O(\omega(d))$ så att
\begin{align*}
    \sum_{\substack{d \divides P(z) \\ d > y}} \abs{\frac{\omega(d)}{d} \mu(d)} + \sum_{\substack{d \divides P(z)  \\ d \leq y}} \abs{ \mu(d) R_d} \leq 
    \sum_{\substack{d \divides P(z) \\ d > y}} \abs{\frac{\omega(d)}{d}} + \sum_{\substack{d \divides P(z)  \\ d \leq y}} \abs{R_d} \leq
    \sum_{\substack{d \divides P(z) \\ d > y}} \frac{\omega(d)}{d} + O\left(\sum_{\substack{d \divides P(z)  \\ d \leq y}} \omega(d)\right).
\end{align*}
Summan innanför Ordo-tecknet kan vi förenkla med Rankins trick som säger att indikator funktionen, \(1_{n \leq x} \leq \frac{x}{n}\). Därav skriver vi 
\begin{align*}
    \sum_{\substack{d \divides P(z)  \\ d \leq y}} \omega(d) =
    \sum_{\substack{d \divides P(z)}} \omega(d) (1_{d \leq y})^{\delta} \leq
    \sum_{\substack{d \divides P(z)}} \omega(d) \left(\frac{y}{d}\right)^{\delta}  
    %  = \exp{\left( \delta \log(y) + \log \left(\sum_{d \divides P(z)} \frac{\omega(d)}{d^\delta}\right) \right)}
    % \exp{\left( \delta \log(y) + \prod_{p \divides P(z)} \log \left( 1 + \frac{\omega(p)}{p^\delta}\right) \right)}
\end{align*}
för alla \(\delta > 0\). Med olikheterna \(\log(1 + x) \leq x\) och \(e^x \leq 1 + x e^x\) för $x \geq 0$, och sist partiell summering av (\ref{dimension}) får vi att
\begin{align*}
    \sum_{\substack{d \divides P(z)  \\ d \leq y}} \omega(d) = O\left( \frac{y}{\log(z)} (\log(z))^{\kappa + 1} \exp{\left(- \frac{\log(y)}{\log(z)} \right)} \right).
\end{align*}
På ett liknande sätt omvandlar vi den första summan i feltermen med partiell summering av antagandet (\ref{dimension}) och sedan använder resultatet vi fick för den andra summan. 

Sammanställt, huvud- och felterm, erhåller vi nästa sats.

\begin{theorem}[Eratosthenes generaliserade såll]\label{thm:EratosthenesSieve}
Med notationen från avsnitt \ref{sallproblemet}, så gäller att
\begin{align*}
    \S{A}{P}{z} = X W(z) + O\left(\left(X + \frac{y}{\log z} \right) (\log z)^{\kappa + 1} \exp{\left(-\frac{\log y}{\log z}\right)} \right). 
\end{align*}
givet att följande antaganden håller
\begin{enumerate}
    \item \(\card{\A_d} = 0, \forall d > y\) för något \(y \in \mathbb{R}_+\)
    \item \(\abs{R_d} = O(\omega(d))\)
    \item \(\sum_{p \divides P(z)} \frac{\omega(p) \log(p)}{p} \leq \kappa \log(z) + O(1)\).
\end{enumerate}

\end{theorem}

\subsection{Tillämpningar}

\todo{Skriva om en tillämpning på tvillingprimtal}
