% Skrivet av Erik

Även om Eratosthenes såll fortfarande är mest känt som en enkel algoritm som utan större möda kan utföras med papper och penna så har dess idé raffinerats genom historien till att inkorperera mer avancerad teori för att tillämpas på nya sätt. År 1808 utvecklade Adrien-Marie Legendre (1752-1833) sållet med hjälp av inklusion-exklusionsprincipen formulerad på formen från avsnitt \ref{sallproblemet}. Nedan kommer vi gå igenom denna form av Eratosthenes såll samt, i efterföljande avsnitt, följa hur \cite{cojocarumurty} presenterar en utvidgning av sållet till en mer generell form. 

\subsection{Legendres såll} \label{era.Legendres}

Första steget för att utveckla Eratosthenes såll är att omformulera sållexemplet från introduktionen i de termer vi definierade i föregående avsnittet. I exemplet börjar vi med en lista av alla positiva heltal upp till en övre gräns, säg $x$, vilken vi kan skriva som $\A = \{n \in \mathbb{N} : n \leq x\}$. Låt alla primtal upp till $z$ redan vara inringade och mängderna vi kryssar över vara \(\A_p = \{a \in \A :  a \equiv 0 \pmod{p}\}\) --- mängden av multiplar av $p$ som är mindre än eller lika med $x$. Det är enkelt att se att kardinaliteten av \(\A_d\) är \(\lfloor x/d \rfloor\) så att, om vi väljer $z = \sqrt{x}$ så blir, med hjälp av (\ref{inclusionexclusion}),
\begin{align*}
    \pi(x) - \pi(\sqrt{x}) + 1 = \sum_{d \divides P(\sqrt{x})} \mu(d) \left\lfloor \frac{x}{d} \right\rfloor  
\end{align*}
där \(1\):an på vänstersidan tar i hänsyn att \(1 \in \A\) inte sållas bort på högersidan och \(\pi(\sqrt{x})\) är de inringade primtalen. Detta var Legendres idé när han omformulerade Eratosthenes såll till att räkna primtal \cite[kapitel 1.1]{opera}. 

Legendres formel är bra för att räkna primtal exakt men är långsam och golvfunktionen kan vara svårhanterlig. Vi kan underlätta för oss genom att ersätta golvfunktionen med $x/d$ och med bråkdelen som restterm. Skriver vi bråkdelen som \(\{x/d\} := x/d - \lfloor x/d \rfloor = O(1)\) leder det oss till \(\card{\A_d} = x/d + O(1)\). Dessvärre resulterar det här i en väldigt stor felterm då restermerna ackumulerar, så att \(\pi(x) - \pi(\sqrt{x}) + 1 = x \sum_{d \divides P(\sqrt{x})}\frac{\mu(d)}{d} + O(2^{\pi(\sqrt{x})}) \).

Eftersom \(\A_p\), för varje $p$, är definierad som en restklass modulo $p$ så säger vi att antalet utvalda restklasser modulo $p$ är $\omega(p) = 1$ för alla $p$ -- vi kallar Eratosthenes för ett endimensionellt eller linjärt såll av den här anledningen. Mer allmänt betecknas dimensionen av ett såll med parametern \(\kappa\) om det är ett minimum för en genomsnittlig övre gräns för \(\omega(p)\) för alla $p$ \cite{tenenbaum}, så att
\begin{align} \label{dimension}
    \sum_{p \divides P(z)} \frac{\omega(p) \log(p)}{p} \leq \kappa \log(z) + O(1).
\end{align}
Vi ser således att ett naturligt nästa steg är att generalisera Eratosthenes-Legendres såll för godtyckliga dimensioner. 

\subsection{Eratosthenes generaliserade såll} \label{eratosthenes.gen.såll}
% "Effektivt" kräver en förklaring, i kontrast mot exaktheten av Eratosthenes såll?
Vi vill alltså härleda ett generellt verktyg från idéerna från föregående avsnittet vilket vi kommer göra genom att följa \cite[Kapitel 5.4]{cojocarumurty}. Om vi låter $\A_d$, $\P$, $P(z)$ och $\omega(d)$ vara definierade som i avsnitt \ref{sallproblemet} så kan vi, med utgångspunkt i Eratosthenes och Legendres grundtankar, skapa ett mer flexibelt och effektivt såll. Den stora skillnaden mot vad vi gjorde i föregående avsnitt är att vi här tillåter fler restklasser sållas bort per primtal $p$. 

Utöver vad som står i avsnitt \ref{sallproblemet} kommer vi göra antagandena att \(\card{\A_d} = 0\) för alla \(d\) större än något positivt $y$, att \(\abs{R_d} = O(\omega(d))\) samt att (\ref{dimension}) håller. Vi kan lätt övertyga oss om att dessa är rimliga antaganden i det klassiska Eratosthenes såll-fallet. Eftersom vi utgår från en ändlig mängd $\A$ så håller det första antagandet då \(\card{\A_p} = 0\) för \(p > X\) innebär att, väldigt grovt, \(y \leq P_{X}\). Att det andra antagandet gäller såg vi i föregående avsnitt ty \(\{x/d\} \leq \omega(p) = 1\). Sist så ser vi att (\ref{dimension}) håller genom att sätta in \(\omega(p) = 1\) och utnyttja att \begin{align} \label{Thm.1.4.4}
    \sum_{p\leq n}\frac{\log p}{p} = \log n + O(1).
\end{align}

Vi börjar återigen med Legendres grundidé, inklusion-exklusionsprincipen, men infogar det mer generella uttrycket för $\card{\A_d}$ i formeln,
\begin{align*}
    \S{A}{P}{z} = \sum_{d \divides P(z)} \mu(d) \card{\A_d} = X \sum_{\substack{d \divides P(z) \\ d \leq y}} \frac{\omega(d)}{d} \mu(d) + \sum_{\substack{d \divides P(z)  \\ d \leq y}} \mu(d) R_d 
\end{align*}
där trunkeringen av summan kommer av antagandet \(\card{\A_d} = 0, \forall d > y\). 

Studerar vi den första summan ovan ser vi att vi kan skriva om den som summan över alla \(d \divides P(z)\) minus summan över alla \(d \divides P(z)\) så att \(d > y\).
%\begin{align*}
%    \sum_{\substack{d \divides P(z) \\ d \leq y}} \frac{\omega(d)}{d} \mu(d) =
%    \sum_{\substack{d \divides P(z)}} \frac{\omega(d)}{d} \mu(d) - \sum_{\substack{d \divides P(z) \\ d > y}} \frac{\omega(d)}{d} \mu(d) .
%\end{align*}
Med observationen att \(W(z) = \prod_{p \divides P(z)} (1 - \omega(p)/p) = \sum_{d \divides P(z)} \frac{\mu(d)}{d} \omega(d)\) då \(\frac{\omega(d)}{d}\) är multiplikativ kan vi således dela upp den siktade mängden i en huvudterm och en rest på följande vis
\begin{align*}
    \S{A}{P}{z} = X W(z) + \Biggl(- X \sum_{\substack{d \divides P(z) \\ d > y}} \frac{\omega(d)}{d} \mu(d) + \sum_{\substack{d \divides P(z)  \\ d \leq y}} \mu(d) R_d \Biggr).
\end{align*}

Vi kan skriva om resttermen, först med hjälp av triangelolikheten och sedan med $\abs{\mu(x)} \leq 1$ och vårt antagande $\abs{R_d} = O(\omega(d))$ så att
\begin{align*}
    \sum_{\substack{d \divides P(z) \\ d > y}} \abs{\frac{\omega(d)}{d} \mu(d)} + \sum_{\substack{d \divides P(z)  \\ d \leq y}} \abs{ \mu(d) R_d} \leq 
    \sum_{\substack{d \divides P(z) \\ d > y}} \abs{\frac{\omega(d)}{d}} + \sum_{\substack{d \divides P(z)  \\ d \leq y}} \abs{R_d} \leq
    \sum_{\substack{d \divides P(z) \\ d > y}} \frac{\omega(d)}{d} + O\Bigg(\sum_{\substack{d \divides P(z)  \\ d \leq y}} \omega(d)\Bigg).
\end{align*}
Summan innanför Ordo-tecknet kan vi förenkla med Rankins trick som, enligt \cite[s.68]{cojocarumurty}, säger att indikator funktionen, \(1_{n \leq x} \leq \frac{x}{n}\). Därav skriver vi 
\begin{align*}
    \sum_{\substack{d \divides P(z)  \\ d \leq y}} \omega(d) =
    \sum_{\substack{d \divides P(z)}} \omega(d) (1_{d \leq y})^{\delta} \leq
    \sum_{\substack{d \divides P(z)}} \omega(d) \left(\frac{y}{d}\right)^{\delta}  
    %  = \exp{\left( \delta \log(y) + \log \left(\sum_{d \divides P(z)} \frac{\omega(d)}{d^\delta}\right) \right)}
    % \exp{\left( \delta \log(y) + \prod_{p \divides P(z)} \log \left( 1 + \frac{\omega(p)}{p^\delta}\right) \right)}
\end{align*}
för alla \(\delta > 0\). Med olikheterna \(\log(1 + x) \leq x\) och \(e^x \leq 1 + x e^x\) för $x \geq 0$, och sist partiell summering av (\ref{dimension}) får vi att
\begin{align*}
    \sum_{\substack{d \divides P(z)  \\ d \leq y}} \omega(d) = O\left( \frac{y}{\log(z)} (\log(z))^{\kappa + 1} \exp{\left(- \frac{\log(y)}{\log(z)} \right)} \right).
\end{align*}
På ett liknande sätt omvandlar vi den första summan i feltermen med partiell summering av antagandet (\ref{dimension}) och sedan använder resultatet vi fick för den andra summan. 

Sammanställt, huvud- och felterm, erhåller vi nästa sats, \cite[Sats 5.4.1]{cojocarumurty}:

\begin{theorem}[Eratosthenes generaliserade såll]\label{thm:EratosthenesSieve}
Med notationen från avsnitt \ref{sallproblemet} och följande tre antaganden 
\begin{enumerate}
    \item \(\card{\A_d} = 0, \forall d > y\) för något \(y \in \mathbb{R}_+\)
    \item \(\abs{R_d} = O(\omega(d))\)
    \item \(\sum_{p \divides P(z)} \frac{\omega(p) \log(p)}{p} \leq \kappa \log(z) + O(1)\)
\end{enumerate}
så gäller att
\begin{align*}
    \S{A}{P}{z} = X W(z) + O\left(\left(X + \frac{y}{\log z} \right) (\log z)^{\kappa + 1} \exp{\left(-\frac{\log y}{\log z}\right)} \right). 
\end{align*}

\end{theorem}
    
% Återför vi fokuset till räknandet av primtal så ger sats \ref{thm:EratosthenesSieve} oss att \(\Phi(x,z) = x W(z) + O((x + ))\)...

\subsection{En högredimensionell tillämpning av Eratosthenes såll}\label{eratosthenes.tillämpning}

Ett av målen med avsnitt \ref{eratosthenes.gen.såll} var att kunna använda Eratosthenes såll till att sålla bort flera kongruensklasser per primtal $p$. I det här avsnittet kommer vi utnyttja denna egenskap till att sålla fram primtalstvillinigar och sedan bevisa en variant av Bruns sats \cite[Korollarium 5.4.5]{cojocarumurty},
\begin{theorem}[Bruns sats] Summan av reciproker,
    \[\sum_{\substack{p \\ p + 2 \text{ prima}}} \frac{1}{p} < \infty .\]
\end{theorem} 
Satsen bevisades först av Brun 1919 i artikeln \textit{La série} \(1/5 + 1/7 + 1/11 + 1/13 + 1/17 + 1/19 + 1/29 + 1/31 + 1/41 + 1/43 + 1/59 + 1/61 \dots\) \textit{ou les dénominateurs sont \guillemotleft nombres premiers jumeaux \guillemotright est convergente ou finie} men vi kommer här följa ett annat bevis, ur \cite[s.72-73]{cojocarumurty}, som använder sig av sats \ref{thm:EratosthenesSieve}.

För att hitta primtalstvillinigar mindre än $x$ (\textit{här}: \( \P_2(x) := \{p < x : p + 2 \text{ är ett primtal}\}\)) så vill vi utesluta multiplar av primtal samt tal två mindre än dessa. Med andra ord väljer vi kongruensklasserna $0$ och $-2$ modulo $p$ i sålldefinitionen så att dimensionen \(\kappa = 2\) samt \(\omega(2) = 1\) och \(\omega(p) = 2\) för alla primtal $p > 2$. Vårt val av kongruensklasser ger att \(y = x + 2\) i det första antagandet av sats \ref{thm:EratosthenesSieve} som då säger att
\begin{align*}
    \card{\P_2(x)} \leq \pi(z) + \S{A}{P}{z} \leq z + x W(z) + O\Big(x (\log x)^3 \exp \Big( - \frac{\log x}{\log z} \Big) \Big).
\end{align*}
Vi kan enkelt se att \(W(z)\) i huvudtermen är lika med
\begin{align}
    \prod_{p < z}\left( 1 - \frac{\omega(p)}{p} \right) = \frac{1}{2} \exp \Bigg( \sum_{2 <p < z} \log \left( 1 - \frac{2}{p} \right) \Bigg) \ll \exp \Bigg( - \sum_{2 <p < z} \frac{2}{p}  \Bigg) \label{era.app.mainAppr}
\end{align}
där sista steget följer av olikheten \(\log(1 + x) \leq x, \text{ om } x > -1\). Med en partiell summation av (\ref{Thm.1.4.4}) erhåller vi att \(\sum_{p \leq z} \frac{1}{p} = \log \log z + O(1)\) som, när vi sätter in i (\ref{era.app.mainAppr}), ger oss att
\begin{align} \label{era.app.secondMainAppr} 
    \exp \Bigg( - 2 \sum_{2 <p < z} \frac{1}{p}  \Bigg) \asymp \exp \big( -2 \log \log z + O(1)  \big) \asymp (\log z)^{-2}.
\end{align}
Slutligen får vi alltså att huvudtermen \(x W(z) \ll x (\log z)^{-2}\) och väljer vi \(\log z = \log x / (6 \log \log x)\) så leder det oss till \cite[Sats 5.4.4]{cojocarumurty}, 
\begin{align} \label{PrimTwinCard}
    \card{\P_2(x)} \ll \frac{x (\log \log x)^2}{\log^2 x} .
\end{align}

Genom att partialsummera summan av de reciproka värdena, med \(c_n\) som indikatorfunktionen på \(\P_2\) och \(f(n) = 1 / n\), så får vi 
\begin{align*}
    \sum_{\substack{p \leq x \\ p + 2 \text{ prima}}} \frac{1}{p} = \card{\P_2(x)} \cdot \frac{1}{x} + \int_2^x \card{\P_2(t)} \cdot \frac{1}{t^2} \text{d}t . 
\end{align*}
Med hjälp av (\ref{PrimTwinCard}) får vi att summan i Bruns sats är \(\ll \int_2^\infty \frac{(\log \log t)^2}{t \log^2 t} \text{d}t\) som är ändlig vilket medför att summan är begränsad. 

Vad vi nu visat för summan över \(\P_2\) är i kontrast mot en annan identitet vi använde i ovanstående bevis, \(\sum_{p \leq n} \frac{1}{p} = \log \log n + O(1)\) vilken implicerar att samma summa över primtalen divergerar. Detta innebär inte att primtalstvillinigar är en ändlig delmängd av primtalen men det säger oss att andelen primtalstvillingar är relativt liten. I nästa del kommer vi se ett annat resultat tillskrivet Viggo Brun, Bruns såll, och med hjälp av detta visa ett resultat i motsatt riktning för en mängd snarlik mängden primtalstvillinigar. 