I den här uppsatsen gör vi flitig användning av ordonotation för att beteckna olika asymptotiska gränser. För $D \subset \mathbb{C}$ och två funktioner, $f: D \to \mathbb{C}$ och $g: D \to \mathbb{R}_+$ skriver vi
\begin{align*}
    f(x) = O(g(x)) \quad \text{om det finns } A > 0 \text{ så att} \quad \abs{f(x)} \leq A g(x), \forall x \in D.
\end{align*}
Omväxlande skriver vi även \(f(x) \ll g(x)\) med samma betydelse som ovan. Om vi har att \(f(x) \ll g(x)\) och \(g(x) \ll f(x)\) så skriver vi \(f(x) \asymp g(x)\).