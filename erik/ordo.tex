% Skrivet av Erik

I den här uppsatsen gör vi flitig användning av ordonotation för att beteckna olika asymptotiska gränser. För $D \subset \mathbb{C}$ och två funktioner, $f: D \to \mathbb{C}$ och $g: D \to \mathbb{R}_+$ skriver vi
\begin{align*}
    f(x) = O(g(x)) \quad \text{om det finns } A > 0 \text{ så att} \quad \abs{f(x)} \leq A g(x), \forall x \in D.
\end{align*}
Omväxlande skriver vi även \(f(x) \ll g(x)\) med samma betydelse som ovan. Om vi har att \(f(x) \ll g(x)\) och \(g(x) \ll f(x)\) så skriver vi \(f(x) \asymp g(x)\). Sist så kallar vi $f(x)$ och $g(x)$ för \textit{asymptotiskt ekvivalenta} och skriver \(f(x) \sim g(x)\) om $g(x)$ är nollskild för alla \(x > x_0\), för någon konstant \(x_0\), och \(\lim_{x \to \infty} \frac{f(x)}{g(x)} = 1\).

% Ordo-notation används också för att beteckna \textit{tids-} och \textit{minneskomplexitet}. 