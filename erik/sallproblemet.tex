Det allmänna fallet i sållteori är att vi har en mängd heltal \(\A\), en mängd primtal \(\P\), samt delmängder \(\A_p, \forall p \in \P\) och är intresserade av att sålla fram delmängden \(\mathbb{S}(\A, \P) := \A \setminus \cup_{p \in \P} \A_p\) eller kardinaliteten av dito, \(S(\A, \P) :=\card{(\A \setminus \cup_{p \in \P} \A_p)}\). Vi låter \(P(z)\) beteckna produkten av alla \(p < z\) i \(\P\) med specialfallet \(P(z) = P_z\) då \(\P\) är mängden av alla primtal och \(P_{z_m, z}\) - produkten av alla \(p\) i \(\P\) mellan \(z_m\) och \(z\). Med denna notation så skriver vi %Om \(P(z) = \prod_{\substack{p\in \mathcal{P} \\ p < z}} p\) så låter vi
\begin{align*}
    \mathbb{S}(\A, \P, z) := (\A \setminus \cup_{p \divides P(z)} \A_p)
    \quad \text{och} \quad
    \S{A}{P}{z} := \card{(\A \setminus \cup_{p \divides P(z)} \A_p)} .
\end{align*}
För \(d\), en kvadratfri produkt (dvs. utan delare \(p^\alpha\) för \(\alpha > 1\)) av element \(p \in \P\) definierar vi \(\A_d = \cap_{p \divides d} \A_p\) och \(\A_1 = \A\). Använder vi oss av inklusion-exklusionsprincipen, formulerad med hjälp av möbiusfunktionen, får vi då
\begin{align} \label{inclusionexclusion}
    \S{A}{P}{z} = \sum_{\substack{d \divides P(z) \\ d \leq z}} \mu(d) \card{\A_d} .
\end{align} % sida 72 

Ett gemensamt drag hos de matematiska sållen diskuterade nedan är att låta \(X\) beteckna kardinaliteten av \(\A\) och hitta \(\delta_p\) så att
\begin{align} 
    \card{\A_p} = \delta_p X + R_p\label{deltaX}
\end{align}
där \(\delta_p \in [0, 1)\) kan betraktas som en uppskattning av andelen element i delmängden \(\A_p\) och \(R_p\) som en felterm. För \(p \neq q\) låter vi \(\delta_{pq} = \delta_p \delta_q\) så att \(\card{\A_{pq}} = \delta_p \delta_q X + R_{pq}\) och utvidgat för ett kvadratfritt tal \(d = p_1 \cdot ... \cdot p_n\) låter vi \(\delta_{d} = \delta_{p_1} \cdot ... \cdot \delta_{p_n}\). Dessutom används, som en teknisk detalj, att \(R_{pp} = R_p\). 

I avsnitt \ref{Eratosthenes} om Eratosthenes generaliserade såll och avsnitt \ref{brun} om Bruns såll så kommer vi välja \(\delta_p = \omega(p) / p\) där \(\omega(p)\) betecknar antalet utvalda restklasser modulo \(p\) vi vill sålla bort. Som följd låter vi i dessa fall \(\A_p\) vara alla element som tillhör någon av dessa restklasser modulo $p$ och för kvadratfria $d$ låter vi \(\omega(d) := \prod_{p \divides d} \omega(p)\). % Alt. säga omega(d) helt multiplikativ
\todo{Skriv mer om omegafunktionen, definiera W(z) och städa upp i texten}