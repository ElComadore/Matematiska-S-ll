% Skrivet av Erik

Den här texten lånar större delen av sin notation från boken \textit{An Introduction to Sieve Methods and their Applications} av Alina Carmen Cojocaru och M. Ram Murty \cite{cojocarumurty}, exempelvis skriver vi \(p, q, \ell\) när vi menar primtal, \(n, d, k\) för naturliga tal och \(x, y, z\) för positivt reella tal. Den största gemensamma delaren betecknas här \(\gcd{d, k}\), funktionen \(\nu(d)\) beskriver antalet distinkta primtalsdelare av \(d\) och \(\pi(z)\) är antalet primtal mindre än eller lika med \(z\). Nedanstående avsnitt ämnar till att etablera mer notation och tekniker som är vanligt förekommande i sållteori med utgångspunkt i \cite{cojocarumurty}.

\subsection{Multiplikativa funktioner} \label{mult}
En särskilt trevlig delmängd av alla funktioner i talteoretiska sammanhang är de multiplikativa funktionerna. Vi säger att en funktion $f$ är \textit{multiplikativ} om $f(1) = 1$ och \(f(mn) = f(m)f(n)\) då $\gcd{m,n} = 1$. Om detta håller för alla $m, n$ så kallar vi $f$ för \textit{fullständigt multiplikativ}. 

Multiplikativa funktioner har fördelen att en viss typ av summor kan omskrivas till produkter, s.k. Eulerprodukter. Om \(f\) är multiplikativ så följer av aritmetikens fundamentalsats att
\begin{align*}
    \sum_{n = 1}^\infty f(n) = \prod_p \sum_{i=0}^\infty f(p^i)
\end{align*}
så länge den första summan konvergerar. 

\subsection{Möbiusfunktionen} \label{Mobius}
En ytterst väsentlig multiplikativ funktion i sållteori är Möbiusfunktionen,
\begin{equation*}
    \mu(n) = 
    \begin{cases}
        1, & \text{om}\ n \text{ är ett kvadratfritt, naturligt tal med jämnt antal primdelare}\\
        -1, & \text{om}\ n \text{ är ett kvadratfritt, naturligt tal med udda antal primdelare}\\
        0, & \text{om}\ n \text{ inte är kvadratfri}
    \end{cases}
\end{equation*}
som vi, bland annat, kommer se förenkla inklusion-exklusionsprincipen i nästa kapitel. Två andra viktiga egenskaper hos Möbiusfunktionen är följande egenskap,
\begin{equation*}
    \sum_{d \divides n} \mu(d) =
    \begin{cases}
        1, & \text{om}\ n = 1 \\
        0, & \text{om}\ n > 0
    \end{cases}
\end{equation*}
och Möbius inverteringsformel som säger
\begin{equation} \label{mobiusinv}
    f(n) = \sum_{d \divides n} g(d) \implies g(n) = \sum_{d \divides n} \mu(d) f(n/d)
\end{equation}
om \(f, g : \mathbb{N} \to \mathbb{C}\).

%\subsection{O-notation}
% Intresserade av en övre asymptotisk gräns --> förklara Ordonotation. 

%Matematisk sållteori är en underkategori av analytisk talteori. Ett av våra viktigaste analytiska verktyg i den här uppsatsen är Ordo-notationen. 

%\subsection{Abels summationsformel}