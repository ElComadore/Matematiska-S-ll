% Skrivet av Nils
Följande sats beskriver det asymptotiska beteendet hos $W(z)$ i specialfallet då $\omega(2)=1$ och $\omega(p)=2$ för primtal $p>2$. 
Resultatet används i (\ref{brun.application}) samt (\ref{eratosthenes.tillämpning}) och idéerna till beviset är hämtade från beviset av \textit{Merten's Theorem} i \cite[kap 5.2]{cojocarumurty}.

\begin{theorem} \label{APDX:asympW}
Antag att $\omega(2)=1$ och $\omega(p)=2$ för alla primtal $p>2$, och definiera 
\begin{equation*}
    W(z):=\prod_{p<z}\biggl( 1-\frac{\omega(p)}{p} \biggr).   
\end{equation*}
Då gäller det att
\begin{equation} \label{APDX:asympW.main}
    W(z) \asymp (\log z)^{-2},
\end{equation}
då $z\to\infty$.
\end{theorem}


\begin{proof}
Från (\ref{era.app.mainAppr}) och (\ref{era.app.secondMainAppr}) har vi redan att
\begin{align*}
    W(z) \ll \exp \biggl( - \sum_{2 <p < z} \frac{2}{p}  \biggr),
    \quad\text{och}\quad
    \exp \biggl( - 2 \sum_{2 <p < z} \frac{1}{p}  \biggr) \asymp (\log z)^{-2}.
\end{align*}
Det som återstår att bevisa är därmed
\begin{equation} \label{APDX:asympW.gg}
    W(z) \gg \exp \biggl( - \sum_{2 <p < z} \frac{2}{p}  \biggr).
\end{equation}
Låt oss betrakta
\begin{equation*}
    -\log(W(z)) 
    = \sum_{p < z} -\log\biggl( 1-\frac{\omega(p)}{p} \biggr) 
    = \sum_{p < z} \log\biggl( 1+\frac{\omega(p)}{p-\omega(p)} \biggr).
\end{equation*}
Användning av antagandet på $\omega$ implicerar att ovanstående är lika med 
\begin{equation} \label{APDX:asympW.sum}
    \log 2 + \sum_{2<p<z} \log\biggl( 1+\frac{2}{p-2} \biggr)
    \leq \log 2 + \sum_{2<p<z} \frac{2}{p-2}.
\end{equation}
där olikheten håller eftersom $\log(1+x)\leq x$, för alla $x\geq0$. 
Betrakta nu summan i högerledet och gör omskrivningen
\begin{equation*}
    \sum_{2 <p < z} \frac{2}{p-2} = \sum_{2 <p < z} \frac{2}{p} + \sum_{2 <p < z} \frac{2}{p(p-2)}.
\end{equation*}
Här gäller det att den andra summan konvergerar då $z\to\infty$, vilket kan ses genom
\begin{equation*}
    \sum_{2 <p < z} \frac{2}{p(p-2)} < 2\sum_{2 <p < z} \frac{1}{p^2} < 2\sum_{n = 1}^\infty \frac{1}{n^2} = \frac{\pi^2}{3}.
\end{equation*}
Om vi nu återgår till högerledet i (\ref{APDX:asympW.sum}) får vi slutligen att
\begin{align*}
    -\log(W(z)) \gg \sum_{2 <p < z} \frac{2}{p},
\end{align*}
vilket implicerar (\ref{APDX:asympW.gg}) och beviset är klart.
\end{proof}