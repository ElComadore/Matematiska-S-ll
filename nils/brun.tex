% Skrivet av Nils

Bruns såll publicerades 1915 av den norske matematikern Viggo Brun (1881-1978).
Det som utmärker detta såll är det uppskattar $\S{A}{P}{z}$ både under- och ovanifrån.

\todo{Hitta källa.}

\subsection{Beskrivning av sållet}
Brun visade att för varje funktion $g$ som uppfyller $g(1)=1$,
så gäller det att
\begin{equation*}
    \S{A}{P}{z} 
    = \sum_{d \mid P(z)} \mu(d) g(d) \card\A_d 
    - f_g,
\end{equation*} % f_g = \sum_{d \mid P(z)} \sum_{\substack{p \mid P(z) \\ p<q(d)}} \mu(d) (g(d)-g(pd)) S(\A_{pd},\P^{(pd)},p).
där $f_g$ är en funktion av $\A,\P$ och $z$, definierad med hjälp av $g$. 
Därefter konstruerade Brun två funktioner $g_U$ och $g_L$ så att $f_{g_U}\geq 0$ och $f_{g_L}\leq 0$ alltid gäller. 
Således kan en undre och övre begränsning till $\S{A}{P}{z}$ erhållas, nämligen
\begin{equation}\label{brun.eq.gugl}
    \sum_{d \mid P(z)} \mu(d) g_L(d) \card\A_d 
    \leq \S{A}{P}{z} 
    \leq \sum_{d \mid P(z)} \mu(d) g_U(d) \card\A_d.
\end{equation}
Detta var Bruns stora idé.
Genom att sedan skriva $\card\A_d=\frac{\omega(d)}{d}X + R_d$ kunde han hitta en approximation och felterm för begränsningarna i (\ref{brun.eq.gugl}) och på så vis formulera Bruns såll.

\begin{theorem}[Bruns såll] \label{brun.thm.brun}
Låt $b$ vara ett positivt heltal och $\lambda$,
ett reellt tal som uppfyller $0<\lambda e^{1+\lambda}<1$.
Antag att $\left|R_p\right|\leq\omega(p)\leq pA_1$ för varje $p\in\P$ och någon konstant $A_1<1$,
samt att det finns konstanter $A_2\geq 1$ och $\kappa>0$ som uppfyller
\begin{align*}
    \sum_{w\leq p<z} \frac{\omega(p)\log p}{p} \leq \kappa\log(z/w) + A_2,\quad \text{för}\ 2\leq w\leq z.
\end{align*}

Då begränsas $\S{A}{P}{z}$ ovanifrån av
\begin{align}
    & XW(z)\left(1 + \lambda c_1\exp\left(c_2\frac{2b+3}{\lambda\log z}\right)\right) + O\left(z^{c_3}\right), \label{brun.eq.upper}
    \intertext{och underifrån av}
    & XW(z)\left(1 - c_1\exp\left(c_2\frac{2b+2}{\lambda\log z}\right)\right) + O\left(z^{c_3 - 1}\right). \label{brun.eq.lower}
\end{align}

%\bigskip\noindent
Konstanterna $c_1$, $c_2$ och $c_3$ definieras som
\begin{equation*}
    c_1 := \frac{ 2\lambda^{2b}e^{2\lambda} }{ 1 - \lambda^2e^{2+2\lambda} }, \quad
    c_2 := \frac{A_2}{2}\left(1+\frac{\kappa+\frac{A_2}{\log 2}}{1+A_1}\right)\ \text{och} \quad
    c_3 := 2b + \frac{2.01}{e^{2\lambda/\kappa} - 1}.
\end{equation*}
\end{theorem}

\todo{Förtydligande av sållet. Val av konstanter, mängder etc.}

\todo{Förklara vilka problem som kan lösas m.h.a sållet. Kanske i styckets inledning.}


\subsection{En tillämpning av Bruns såll}
Låt oss uttrycka en modifierad version av primtalstvillingsförmodan på följande vis:\\
\textit{Det finns oändligt många par av heltal $(n,n+2)$ där båda talen har högst $r$ primtalsfaktorer.}\\
För valet $r=1$ erhålls exakt primtalstvillingsfömodan, vilken står obevisad.
Däremot är det möjligt att bevisa påståendet för $r=7$, genom att ta hjälp av Bruns såll.
Definiera mängden
\begin{equation*}
    \mathcal{T} := \{\textit{$n\leq x$: $n$ och $n+2$ har som mest $7$ primtalsfaktorer}\}.
\end{equation*}
Målet är att hitta en undre begränsning till $\card{\mathcal{T}}$ som divergerar då $x\to\infty$.
Låt $\P$ vara mängden av alla primtal och låt
\begin{equation*}
    \A := \{n(n+2): n\leq x\}.
\end{equation*}
Vi vill uppskatta storleken på $\A$ efter bortsållning av tal delbara med primtal $p<z$ för något $z$.
För att kunna använda (\ref{brun.thm.brun}) behöver vi se till så att alla antaganden är uppfyllda, vilket vi ska ägna oss åt nu.


Vi skiljer på två typer av restklasser; noll och nollskild modulo $p$, och har därmed att
\begin{equation*}
    \omega(2)=1\ \text{ och }\ \omega(p)=2,\ \text{för}\ p>2.
\end{equation*}
Detta ger direkt att $\frac{\omega(p)}{p}\leq\frac{2}{3}$ med likhet då $p=3$, låt därför $A_1=2/3$.
För att hitta ett passande $\kappa$, använder vi oss av
\begin{equation}\label{brun.eq.sum_logp_over_p}
    \sum_{p\leq n}\frac{\log p}{p} = \log n + O(1).
\end{equation}
\todo{Referera till boken eller appendix} \\ \noindent
Detta implicerar
\begin{equation*}
    \sum_{w\leq p<z} \frac{\omega(p)\log p}{p} \leq 2\log\left(z/w\right) + O(1),
\end{equation*}
varpå vi ser att $\kappa=2$ uppfyller kriterierna och att $A_2$ existerar,
något explicit värde på $A_2$ är inte nödvändigt här.
%Slutligen, är storleken på $\A$ är $x$ så låt $X=x$ och vi sätter $b=1$, av anledning som kommer att visa sig senare.
Betrakta nu huvudtermen i ($\ref{brun.eq.lower}$). Taylorutveckling av exponenten ger
\begin{align*}
    \exp\left(c_2\frac{2b+2}{\lambda\log z}\right) = 1 + c_2\frac{2b+2}{\lambda}\log(z)^{-1} + \left(c_2\frac{2b+2}{\lambda}\right)^2\log(z)^{-2} + ...
\end{align*}
För fixerade $b,\lambda$ och $c_2$ får vi att exponenten begränsas av $1+O(log(z)^{-1})$, då $z\to\infty$.


\todo{Några steg till här}

Om det är så att $n$ eller $n+2$ har mer än 7 primtalsfaktorer,
så måste $p\mid n(n+2)$ för något primtal $p\leq n^{1/u}$, där $u<8$.
Vi kan därför uttrycka $\card\mathcal{T}$ som $\S{A}{P}{z}$ genom att sätta $z := x^{1/u}$.
Detta ger oss att
\begin{equation*}
    \card\mathcal{T} \gg \frac{x}{(\log x)^2}.
\end{equation*}






% Sådant som inte används just nu
\begin{comment}

två funktioner $g_U$ och $g_L$ som uppfyller 
\begin{alignat*}{3}
    g_U(1)&=1,\quad \text{och}&\quad \mu(d)(g_U(d)-g_U(pd)) &\geq 0, \\
    g_L(1)&=1,\quad \text{och}&\quad \mu(d)(g_L(d)-g_L(pd)) &\leq 0,
\end{alignat*}






\end{comment}