% Skrivet av Nils

När den norska matematikern Viggo Brun (1881-1978) år 1915 publicerade artikeln \textit{Über das Goldbachsche Gesetz und die Anzahl der Primzahlpaare}, lades grunden till modern sållteori \cite{ViggoBrun}. 
I artikeln presenterade Brun ett såll baserat på Eratosthenes idéer, som numera kallas Bruns såll och fyra år senare använde han det för att bevisa att summan av reciproker $\frac{1}{p_1}+\frac{1}{p_2}+...$ av primtalstvillingar $p_1,p_2,...$ konvergerar \cite{ViggoBrun}.
I detta avsnitt ges en kortfattad beskrivning av Bruns såll och ett exempel taget ur \cite{cojocarumurty} på hur sållet kan tillämpas för att ge ett resultat som angränsar till primtalstvillingsförmodan.


\subsection{Beskrivning av sållet}
Brun visade att för varje funktion $g$ som uppfyller $g(1)=1$,
så gäller det att
\begin{equation*}
    \S{A}{P}{z} 
    = \sum_{d \mid P(z)} \mu(d) g(d) \card\A_d 
    - f_g,
\end{equation*} % f_g = \sum_{d \mid P(z)} \sum_{\substack{p \mid P(z) \\ p<q(d)}} \mu(d) (g(d)-g(pd)) S(\A_{pd},\P^{(pd)},p).
där $f_g$ är en funktion av $\A,\P$ och $z$, definierad med hjälp av $g$. 
Därefter konstruerade Brun två funktioner $g_U$ och $g_L$ så att $f_{g_U}\geq 0$ och $f_{g_L}\leq 0$ alltid gäller. 
Således kan en undre och övre begränsning till $\S{A}{P}{z}$ erhållas, nämligen
\begin{equation}\label{brun.eq.gugl}
    \sum_{d \mid P(z)} \mu(d) g_L(d) \card\A_d 
    \leq \S{A}{P}{z} 
    \leq \sum_{d \mid P(z)} \mu(d) g_U(d) \card\A_d.
\end{equation}
Enligt \cite{cojocarumurty} var detta Bruns innoverande idé.
Genom att sedan skriva $\card\A_d=\frac{\omega(d)}{d}X + R_d$ kunde han, efter mycket möda, hitta en approximation och en felterm för begränsningarna i (\ref{brun.eq.gugl}) och på så vis formulera Bruns såll. Följande sats är hämtad ur \cite[Kap. 6.2]{cojocarumurty}.


\begin{theorem}[Bruns såll] \label{brun.thm.brun}
Med notation från avsnitt \ref{sallproblemet}, låt $b$ vara ett positivt heltal och $\lambda$,
ett reellt tal som uppfyller $0<\lambda e^{1+\lambda}<1$.
Antag att $\left|R_d\right|\leq\omega(d)\leq dA_1$ för varje kvadratfritt tal $d$ med faktorer ur $\P$ och någon konstant $A_1<1$,
samt att det finns konstanter $A_2\geq 1$ och $\kappa>0$ som uppfyller
\begin{align*}
    \sum_{w\leq p<z} \frac{\omega(p)\log p}{p} \leq \kappa\log(z/w) + A_2,\quad \text{för}\ 2\leq w\leq z.
\end{align*}
Då begränsas $\S{A}{P}{z}$ ovanifrån av
\begin{align}
    & XW(z)\left(1 + \lambda c_1\exp\left(c_2\frac{2b+3}{\lambda\log z}\right)\right) + O\left(z^{c_3}\right), \label{brun.eq.upper}
    \intertext{och underifrån av}
    & XW(z)\left(1 - c_1\exp\left(c_2\frac{2b+2}{\lambda\log z}\right)\right) + O\left(z^{c_3 - 1}\right). \label{brun.eq.lower}
\end{align}
Konstanterna $c_1$, $c_2$ och $c_3$ definieras som
\begin{equation*}
    c_1 := \frac{ 2\lambda^{2b}e^{2\lambda} }{ 1 - \lambda^2e^{2+2\lambda} }, \quad
    c_2 := \frac{A_2}{2}\biggl(1+\frac{\kappa+\frac{A_2}{\log 2}}{1+A_1}\biggr)\ \text{och} \quad
    c_3 := 2b + \frac{2.01}{e^{2\lambda/\kappa} - 1}.
\end{equation*}
\end{theorem}

\todo{Förtydligande av sållet. Val av konstanter, mängder etc.}


\subsection{En tillämpning av Bruns såll}
Låt oss uttrycka en modifierad version av primtalstvillingsförmodan på följande vis:\\
\textit{Det finns oändligt många par av heltal $(n,n+2)$ där båda talen har högst $r$ primtalsfaktorer.}\\
För valet $r=1$ erhålls exakt primtalstvillingsfömodan, vilken står obevisad.
Däremot går det att med hjälp av Bruns såll bevisa påståendet för $r=7$, vi följer samma tillvägagångssätt som i \cite[Kap. 6.2]{cojocarumurty}.
Definiera mängden
\begin{equation*}
    \mathcal{T} := \{\textit{$n\leq x$: $n$ och $n+2$ har som mest $7$ primtalsfaktorer}\}.
\end{equation*}
Målet är att hitta en undre begränsning till $\card{\mathcal{T}}$ som divergerar då $x\to\infty$.
Låt $\P$ vara mängden av alla primtal och låt
\begin{equation*}
    \A := \{n(n+2): n\leq x\}.
\end{equation*}
Med hjälp av (\ref{brun.eq.lower}), vill vi nu uppskatta storleken på $\A$ efter bortsållning av tal delbara med primtal $p<z$ för något $z$.
Vi skiljer på två typer av restklasser; noll och nollskild modulo $p$, och har därmed att
\begin{equation*}
    \omega(2)=1\ \text{ och }\ \omega(p)=2,\ \text{för}\ p>2.
\end{equation*}
Detta ger direkt att $\frac{\omega(p)}{p}\leq\frac{2}{3}$ med likhet då $p=3$, låt därför $A_1=2/3$.
För att hitta ett passande $\kappa$, använder vi oss av
\begin{equation}\label{brun.eq.sum_logp_over_p}
    \sum_{p\leq n}\frac{\log p}{p} = \log n + O(1).
\end{equation}
\todo{Bör nog presenteras tidigare, eller i appendix} \\ \noindent
För $\omega(p)\leq2$ implicerar detta att
\begin{equation*}
    \sum_{w\leq p<z} \frac{\omega(p)\log p}{p} \leq 2\log\left(z/w\right) + O(1),
\end{equation*}
varpå vi ser att $\kappa=2$ uppfyller kriterierna och att $A_2$ existerar.
Något explicit värde på $A_2$ är inte nödvändigt, för betrakta exponenten i ($\ref{brun.eq.lower}$). Taylorutveckling av ger
\begin{align*}
    \exp\left(c_2\frac{2b+2}{\lambda\log z}\right) = 1 + c_2\frac{2b+2}{\lambda}(\log z)^{-1} + \left(c_2\frac{2b+2}{\lambda}\right)^2(\log z)^{-2} + ...
\end{align*}
Fixera $b,\lambda$ och $A_2$ så att även $c_2$ är fixerat, då följer det att exponenten begränsas av $1+O((\log z)^{-1})$, då $z\to\infty$.
Vidare har vi att 
\begin{align*}
    W(z) = \frac{1}{2}\exp\Bigl(\sum_{2<p<z} \log(1-2/p) \Bigr) 
    \gg \frac{1}{2}\exp\Bigl(\sum_{2<p<z} \frac{-2}{p} \Bigr).
\end{align*}
Partiell summation av (\ref{brun.eq.sum_logp_over_p}) ger att $\sum_{p\leq n}\frac{1}{p} = \log\log z + O(1)$, så
\begin{align*}
    W(z) \gg \exp\Bigl(-2\log\log z\Bigr)
    = (\log z)^{-2}.
\end{align*}
Sats \ref{brun.thm.brun} ger därmed att
\begin{equation*}
    \S{A}{P}{z} \gg \frac{x}{(\log z)^2}\bigl(1 - c_1\bigr) + O\left(z^{c_3 - 1}\right).
\end{equation*}
Nu vill vi välja $b$ och $\lambda$, så att $c_1<1$ och $c_3$ är litet. 
Detta uppfylls av $b=1$ samt $\lambda=0.101$ som medför att $c_3-1<8$.

\todo{Krävs nog lite förtydligande här.}

I fallet att $n$ eller $n+2$ har mer än 7 primtalsfaktorer,
så måste $p\mid n(n+2)$ för något primtal $p\leq n^{1/u}$ och $u<8$.
Vi kan därför uttrycka $\card\mathcal{T}$ som $\S{A}{P}{z}$ genom att sätta $z := x^{1/u}$, vilket ger
\begin{equation*}
    \card\mathcal{T} \gg \frac{x}{(\log x)^2}.
\end{equation*}






% Sådant som inte används just nu
\begin{comment}

två funktioner $g_U$ och $g_L$ som uppfyller 
\begin{alignat*}{3}
    g_U(1)&=1,\quad \text{och}&\quad \mu(d)(g_U(d)-g_U(pd)) &\geq 0, \\
    g_L(1)&=1,\quad \text{och}&\quad \mu(d)(g_L(d)-g_L(pd)) &\leq 0,
\end{alignat*}






\end{comment}