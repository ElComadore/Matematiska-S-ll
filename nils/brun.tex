% Skrivet av Nils

% Inledning
År 1915 presenterade norrmanen Viggo Brun (1885-1978) ett såll som senare fått namnet Bruns såll. Brun hade inspirerats av publikationer av Jacques Hadamard som handlade om de framsteg som Jean Merlin gjort i att utöka Eratosthenes teori. Sållet blev inte uppmärksammat direkt, istället tog det ca 30 år innan andra matematiker började intressera sig för det. En anledning till dröjsmålet kan vara att Bruns skrivsätt och val av notation gjorde materialet onödigt svårläst. Sedan den ursprungliga publikationen har det gjorts flera insatser för att förkorta och förenkla läsningen men trots detta är Bruns såll och dess bevis krävande för läsaren, med detta i åtanke har vi valt att korta ned beviset avsevärt i förhoppningen om att stycket ska upplevas som överkomligt.


\bigskip\noindent
Vi börjar med att välja två funktioner $f$ och $g$ som uppfyller $f(n) = \sum_{d\mid n} \mu(d) g(d)$. Låt $\P^{(\delta)}$ beteckna $\{ p\in\P;p \nmid \delta\}$,  genom Möbius inverteringsformel finner vi då att
\begin{align*}
    \sum_{d\mid P(z)} \mu(d) g(d) \card\A_d 
    = \sum_{\delta\mid P(z)} f(\delta) \sum_{d\mid \frac{P(z)}{\delta}} \mu(d) \card\A_d 
    = \S{A}{P}{z} + \sum_{\substack{\delta\mid P(z) \\ \delta>1}} f(\delta) \S{A_\delta}{P^{(\delta)}}{z},
\end{align*}
Vi är intresserade av att uppskatta den sista summan. Låt $q(\delta)$ beteckna det minsta primtalet som delar $\delta$. Detta ger oss, efter flera omskrivningar:


\begin{theorem} \label{brun.thm.1}
Med ovan nämnda antaganden och att $g(1)=1$, gäller 
\begin{equation*}
    \S{A}{P}{z} 
    = \sum_{d \mid P(z)} \mu(d) g(d) \card\A_d 
    - \sum_{d \mid P(z)} \sum_{\substack{p \mid P(z) \\ p<q(d)}} \mu(d) (g(d)-g(pd))
    S(\A_{pd},\P^{(pd)},z).      % Undantag i användning av \S-kommandot
\end{equation*}
\end{theorem}

Genom vissa restriktioner på funktionen $g$ går det att tvinga dubbelsumman ovan till att alltid vara $\geq 0$ eller alltid $\leq 0$. Vi har följande:

\begin{theorem} \label{brun.thm.2}
Behåll notation och antaganden från ovan. Låt $g_U$ och $g_L$ vara två funktioner där $g_U(1)=g_L(1)=1$ med $\mu(d)(g_U(d)-g_U(pd)) \geq 0$ och $\mu(d)(g_L(d)-g_L(pd)) \leq 0$, för alla $d \mid P(z), p < q(d)$. Då gäller
\begin{equation}\label{brun.eq.bound}
    \sum_{d \mid P(z)} \mu(d) g_L(d) \card\A_d 
    \leq \S{A}{P}{z} 
    \leq \sum_{d \mid P(z)} \mu(d) g_U(d) \card\A_d.
\end{equation}
\end{theorem}


Detta var Bruns huvudidé, genom att hitta funktionerna $g_U$ och $g_L$ är det alltså möjligt att begränsa $\S{A}{P}{z}$ både ovan och underifrån. Vi ska nu ägna oss åt att välja passande $g_U$ och $g_L$ som uppfyller de önskade kriterierna.

Låt $b$ vara ett fixt positivt heltal och välj $(z_n)$ där $0 \leq n \leq r$, för något $r$ så att
\begin{equation*}
    2 = z_r < z_{r-1} < \cdots < z_1 < z_0 = z.
\end{equation*}
Definiera produkten $P_{z_n, z} := \prod_{z_n \leq p < z} p$ och låt
\begin{equation*}
    g_U(d) :=
    \begin{cases}
        1, & \text{om}\ \nu(\gcd{d, P_{z_n, z}}) \leq 2b+2n-2,\ \forall n \leq r, \\
        0, & \text{annars,}
    \end{cases}
\end{equation*}
och
\begin{equation} \label{brun.eq.gugl}
    g_L(d) :=
    \begin{cases}
        1, & \text{om}\ \nu(\gcd{d, P_{z_n, z}}) \leq 2b+2n-3,\ \forall n \leq r, \\
        0, & \text{annars.}
    \end{cases}
\end{equation}
Dessa två funktioner $g_U$ och $g_L$ uppfyller antagandena i sats \ref{brun.thm.2}. Med notation från (\ref{deltaX}), skriver vi
\begin{equation*} \label{brun.eq.omega}
        \card{\A_d} = \delta_d X + R_d = \frac{\omega(d)}{d} X + R_d.
\end{equation*}
Då är $\omega(d)$ en multiplikativ funktion. Vi kan nu skriva den undre begränsningen i (\ref{brun.eq.bound}) som 
\begin{equation*}
    \sum_{d\mid P(z)}\mu(d)g_L(d)\frac{\omega(d)}{d} +
    \sum_{d \mid P(z)} \mu(d) g_L(d) R_d
\end{equation*}
och liknande för den övre begränsningen. Med detta i åtanke formulerar vi:


\begin{lemma} \label{brun.lem.1}
Låt $g_U$ och $g_L$ vara som i (\ref{brun.eq.gugl}), då gäller
\begin{align*}
    \sum_{d\mid P(z)}\mu(d)g_U(d)\frac{\omega(d)}{d} =
    W(z)\left(1 + \sum_{p<z}\frac{\omega(p)W(p)}{pW(z)} \times \sum_{t\mid P_{p^+,z}}\frac{g_U(t)(1-g_U(pt)}{t}\omega(t)\right),
\end{align*}
och
\begin{align*} \label{brun.eq.sum}
    \sum_{d\mid P(z)}\mu(d)g_L(d)\frac{\omega(d)}{d} =
    W(z)\left(1 + \sum_{p<z}\frac{\omega(p)W(p)}{pW(z)} \times \sum_{t\mid P_{p^+,z}}\frac{g_L(t)(1-g_L(pt)}{t}\omega(t)\right),
\end{align*}
där $p^+$ betecknar det minsta primtal i $\P$ större än $p$ och
\begin{equation*}
    W(z) := \prod_{p\mid P(z)}\left(1 - \frac{\omega(p)}{p} \right).
\end{equation*}
\end{lemma}
%Högerleden till ekvationerna i lemmat ovan går att uppskatta.

\comment{
\begin{lemma}
Med notation definierad tidigare i kapitlet gäller det att
\begin{equation*}
    \sum_{d\mid P(z)}\mu(d)g_U(d)\frac{\omega(d)}{d} \leq
    W(z)\left(1+ \frac{2}{} \right)
\end{equation*}

\end{lemma}
}


\begin{theorem}[Bruns såll] \label{brun.thm.brun}
Behåll notation från tidigare i kapitlet. Låt $b$ vara ett positivt heltal och $\lambda$, ett reellt tal som uppfyller $0<\lambda e^{1+\lambda}<1$.
Gör följande tre antaganden:

\bigskip\noindent
1. $\left|R_d\right| \leq \omega(d)$ där $d\mid \prod_{p\in\P}p$. \\
2. Det finns en konstant $A_1\geq1$ så att $\omega(p)/p \leq 1-1/A_1$, för alla $p\in\P$\\
3. Det finns konstanter $\kappa>0$ och $A_2\geq 1$ så att
\begin{equation*}
    \sum_{w\leq p<z} \frac{\omega(p)\log p}{p} \leq \kappa\log(z/w) + A_2,\ \text{ för }\ 2\leq w\leq z.
\end{equation*}

\bigskip\noindent
Då begränsas $\S{A}{P}{z}$ ovanifrån av
\begin{align*}
    XW(z)\left(
    1 + \lambda\frac{ 2\lambda^{2b}e^{2\lambda} }{ 1 - \lambda^2e^{2+2\lambda} }
    \exp\left( c_1\frac{2b+3}{\lambda\log z}
    \right)\right)
    + O\left(z^{2b+\frac{2.01}{e^{2\lambda/\kappa} - 1}}\right)
\end{align*}
och underifrån av
\begin{align*}
    XW(z)\left(
    1 - \frac{ 2\lambda^{2b}e^{2\lambda} }{ 1 - \lambda^2e^{2+2\lambda} }
    \exp\left( c_1\frac{2b+2}{\lambda\log z}
    \right)\right)
    + O\left(z^{2b+\frac{2.01}{e^{2\lambda/\kappa} - 1} - 1}\right),
\end{align*}
där 
\begin{equation*}
    c_1 := \frac{A_2}{2}\left(1+A_1\left(\kappa+\frac{A_2}{\log 2}\right)\right).
\end{equation*}




\end{theorem}




