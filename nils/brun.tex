% Skrivet av Nils

\subsection{Sållet}
Bruns såll publicerades 1915 av den norska matematikern Viggo Brun (1881-1978). 

\todo{Hitta källa}

Den idé som utmärker Bruns såll är konstruktionen av två funktioner, som via Eratosthenes såll ger en övre och undre begränsning till $\S{A}{P}{z}$. 
Nämligen
\begin{equation}\label{brun.eq.gugl}
    \sum_{d \mid P(z)} \mu(d) g_L(d) \card\A_d 
    \leq \S{A}{P}{z} 
    \leq \sum_{d \mid P(z)} \mu(d) g_U(d) \card\A_d.
\end{equation}

Brun hittade $g_U$ och $g_L$ så att ovanstående är uppfyllt, dessa funktioner definieras med hjälp av $\P(z)$ och ett positivt heltal $b$ som kan väljas fritt, vilket ger sållet flexibilitet. Genom att skriva $\card\A_d=\frac{\omega(d)}{d}X + R_d$ kunde han hitta en approximation och felterm för begränsningarna i (\ref{brun.eq.gugl}) och på så vis formulera Bruns såll.

\begin{theorem}[Bruns såll] \label{brun.thm.brun}
Låt $b$ vara ett positivt heltal och $\lambda$, ett reellt tal som uppfyller $0<\lambda e^{1+\lambda}<1$. Antag att det finns en konstant $A_1<1$ så att $0\leq\frac{\omega(p)}{p}\leq A_1$, för varje $p\in\P$. Antag även att $\left|R_p\right| \leq \omega(p)$, för varje $p\in\P$.

\bigskip%\noindent
Då begränsas $\S{A}{P}{z}$ ovanifrån av
\begin{align}
    & XW(z)\left(1 + \lambda c_1\exp\left(c_2\frac{2b+3}{\lambda\log z}\right)\right) + O\left(z^{c_3}\right), \label{brun.eq.upper}
    \intertext{och underifrån av}
    & XW(z)\left(1 - c_1\exp\left(c_2\frac{2b+2}{\lambda\log z}\right)\right) + O\left(z^{c_3 - 1}\right). \label{brun.eq.lower}
\end{align}

%\bigskip\noindent
Konstanterna $c_1$, $c_2$ och $c_3$ definieras som
\begin{equation*}
    c_1 := \frac{ 2\lambda^{2b}e^{2\lambda} }{ 1 - \lambda^2e^{2+2\lambda} }, \quad
    c_2 := \frac{A_2}{2}\left(1+\frac{\kappa+\frac{A_2}{\log 2}}{1+A_1}\right)\ \text{och} \quad
    c_3 := 2b + \frac{2.01}{e^{2\lambda/\kappa} - 1},
\end{equation*}
där $\kappa>0$ och $A_2\geq1$ är två konstanter som uppfyller
\begin{equation*}
    \sum_{w\leq p<z} \frac{\omega(p)\log p}{p} \leq \kappa\log(z/w) + A_2,\ \text{ för }\ 2\leq w\leq z.
\end{equation*}
\end{theorem}

\todo{Förtydligande av sållet. Val av konstanter, mängder etc}

\todo{Förklara varför sållet är spännande}

\todo{Förklara vilka problem vi kan lösa m.h.a sållet}


\subsection{En tillämpning av Bruns såll EJ OMSKRIVET}
Låt oss uttrycka en mer generell version av primtalstvillingsförmodan på följande vis:\\
\textit{Det finns oändligt många par av heltal $(n, n+2)$ där båda talen har som mest $r$ primtalsfaktorer.}\\
Sätter vi $r=1$ får vi exakt primtalstvillingsfömodan som ej är bevisad, däremot kan vi med hjälp av Bruns såll bevisa påståendet för $r=7$. Definiera
\begin{equation*}
    \mathcal{T} := \{\textit{$n\leq x$: $n$ och $n+2$ har som mest $7$ primtalsfaktorer}\}.
\end{equation*}
Vi vill hitta en undre begränsning till $\card{\mathcal{T}}$ som går mot $\infty$ då $x\to\infty$. Låt
\begin{equation*}
    \A := \{n(n+2): n\leq x\}
\end{equation*}
Från denna mängden vill vi sålla bort tal som delas av primtal $p<z$ för något $z$. Vi skiljer alltså på två typer av residualklasser och har därmed
\begin{equation*}
    \omega(2)=1\ \text{ och }\ \omega(p)=2,\ \text{för}\ p>2.
\end{equation*}
Vidare väljer vi
\begin{equation*}
    X=x,\ 
    A_0=\kappa=2,\ 
    A_1=3\
    \text{ och }\
    b=1.
\end{equation*}

\todo{Visa några steg i uträkningen}

Om det är så att $n$ eller $n+2$ har mer än 7 primtalsfaktorer så måste $p\mid n(n+2)$ för något primtal $p\leq n^{1/u}$, där $u<8$. Vi kan därför uttrycka $\card\mathcal{T}$ som $\S{A}{P}{z}$ genom att sätta $z := x^{1/u}$. Detta ger oss att
\begin{equation*}
    \card\mathcal{T} \gg \frac{x}{(\log x)^2}.
\end{equation*}






% Sådant som inte används just nu
\begin{comment}

två funktioner $g_U$ och $g_L$ som uppfyller 
\begin{alignat*}{3}
    g_U(1)&=1,\quad \text{och}&\quad \mu(d)(g_U(d)-g_U(pd)) &\geq 0, \\
    g_L(1)&=1,\quad \text{och}&\quad \mu(d)(g_L(d)-g_L(pd)) &\leq 0,
\end{alignat*}






\end{comment}