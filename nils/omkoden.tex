\newcommand{\code}[1]{\inputminted[frame=lines,fontsize=\footnotesize,linenos]{python}{code/#1.py}}

Nedan följer väsentliga delar av den förbättrade versionen av programmet som omnämns i \ref{partB.implementering}.
Om läsaren vill testa programmet all kod förslagsvis läggas i samma fil och för att koden ska kunna köras behöver Python-bibliotek \textit{Bitarray} vara installerat på datorn.

Vi börjar import av externa funktioner, förslagsvis görs detta allra först i koden.
\code{imports}

Härnäst följer algoritmerna vars funktion beskrivs i \ref{partB.algoritmteori}, 
på flera rader finns kommentarer som förklarar radens syfte.
\code{SimpleSiev} %\label{code.SimpleSiev}
\code{SimpleSegSiev} %\label{code.SimpleSiev}
\code{SubSegSiev} %\label{code.SimpleSiev}
\code{NewSegSiev} %\label{code.SimpleSiev}
\code{DiophAppr} %\label{code.SimpleSiev}

Här presenteras den funktion som använts i \ref{partB.applications}.
Denna tar in en redan sållad lista med primtal och sållar bort alla primtal $p$ där $p+2$ inte är prima.
De tal i listan som är kvar efter detta är primtalstvillingarna i intervallet.
\code{RemoveNonTwins}\label{code.twins}

Till sist ger vi även en modifierad version av \textsc{NewSegSiev} och \textsc{SubSegSiev} som vi kallar
\textsc{NewSegSievTwins} respektive \textsc{SubSegSievTwins}. 
Som namnet antyder kan dessa användas för att från grunden sålla fram alla primtalstvillingar i det angivna intervallet.
\code{SubSegSievTwins}
\code{NewSegSievTwins}