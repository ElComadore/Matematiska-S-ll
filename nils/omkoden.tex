% Skrivet av Nils

\newcommand{\code}[1]{\inputminted[frame=lines,fontsize=\footnotesize,linenos]{python}{code/#1.py}}

\subsection{Vår förbättrade version av programmet}
Nedan följer väsentliga delar av den förbättrade versionen av programmet som omnämns i \ref{partB.implementering}.
Om läsaren vill testa programmet kan all nedanstående kod förslagsvis läggas i samma fil. För att koden ska kunna köras behöver Python-bibliotek \textit{Bitarray} vara installerat.


Vi börjar med import av några externa funktioner, detta ska göras först i koden för att de senare funktionerna ska fungera.
Biblioteket \texttt{bitarray} ger oss möjlighet att arbeta direkt med bitsträngar och funktionen \texttt{zeros(N)} konstruerar en sådan bitsträng bestående av \texttt{N} stycken nollor.
Funktionerna \texttt{sqrt} och \texttt{floor} ger kvadratroten respektive golvfunktionen av ett tal 
och \texttt{isqrt} returnernar heltalsdelen av kvadratroten.
\code{imports}


Härnäst följer algoritmerna vars funktion beskrivs i \ref{partB.algoritmteori}, 
på flera rader finns det extra kommentarer (på engelska) som förklarar radens syfte.
Alla funktioner, med undantag för \textsc{DiophAppr}, returnerar en bitsträng. Strängen representerar något sållat heltalsintervall $[a,b]$ 
där biten med index $i$ är en nolla om talet $i+a$ har sållats bort och en etta annars.


\subsubsection*{\textsc{SimpleSiev}}
Detta är det vanliga Eratosthenes såll. Funktionen tar in ett heltal $N$ och returnerar en lista med alla primtal i intervallet $[0,N]$.
Exempelvis ger \texttt{SimpleSiev(7)} svaret \texttt{00110101} vilket representerar alla primtal från 0 till 7. 
\code{SimpleSiev} 


\subsubsection*{\textsc{SimpleSegSiev}}
Denna funktion tar in tre argument; $n$, $\Delta$ och $M$. Därefter returnerar den en lista vilken representerar alla tal i intervallet $[n, n+\Delta]$ som antingen är prima eller saknar delare mindre än, eller lika med $M$. Använder vi notation från avsnitt \ref{sallproblemet}, representerar denna lista mängden $\S{A}{P}{z}$ där $A=[n, n+\Delta]$, $z=M$ och $\P$ är mängden av alla primtal. Sållningen görs med primtal genererade av \textsc{SimpleSiev}.
\code{SimpleSegSiev} 


\subsubsection*{\textsc{SubSegSiev}}
Denna funktion är exakt som \textsc{SimpleSegSiev} gällande indata och utdata. 
Skillnaden mot förstnämnda är att \textsc{SubSegSiev} kan sålla i ett större intervall då den delar upp intervallet i mindre segment som sedan sållas en åt gången. Sållningen av de mindre intervallen utförs av \textsc{SimpleSegSiev}.
\code{SubSegSiev} 


\subsubsection*{\textsc{NewSegSiev}}
Detta är algoritmens huvudfunktion. Den tar in tre argument; $n$, $\Delta$ och $K$, och returnerar alla primtal i intervallet $[n-\Delta,n+\Delta]$. Det sista argumentet $K$ ska vara ett flyttal större eller lika med $2.5$ och påverkar hur funktionen går tillväga för att sålla. Funktionen sållningen sker i två delar. Först sållas intervallet för multiplar av primtal mindre än eller lika med $K\Delta$. Detta utförs av \textsc{SubSegSiev} och sker på rad 16. Därefter sållas intervallet för resterande tal, vilket är multiplar av tal mellan $K\Delta$ och $\sqrt{n+\Delta}$. Detta görs med hjälp av \textsc{DiophAppr} i while-loopen som börjar på rad 17 och slutar på rad 31.
\code{NewSegSiev} 

\subsubsection*{\textsc{DiophAppr}}
Denna funktion tar in två tal; $\alpha$ och $Q$, och beräknar en rationell approximation $a/q$ av $\alpha$ med hjälp av kedjebråk.
Heltalen $a$ och $q$ uppfyller $\gcd{a,q}=1$ samt att $q\leq Q$ och $\abs{a/q-\alpha}\leq 1/qQ$.
Funktionen returnerar talparet $(q, a^{-1})$ där $a^{-1}$ är den multiplikativa inversen till $a$ modulo $q$.
\code{DiophAppr} 


\subsection{Några extra funktioner}
\subsubsection*{\textsc{RemoveNonTwins}}
Här presenteras den funktion som använts i \ref{partB.applications}.
Denna tar in en redan sållad lista med primtal och sållar bort alla primtal $p$ där $p+2$ inte är prima.
De tal i listan som är kvar efter detta är primtalstvillingarna i intervallet.
Om det sista eller näst sista talet i listan är prima så är det omöjligt för funktionen att avgöra om talet är ett tvillingprimtal eller ej, funktionen sållar inte bort talet men utfärdar en varning om att detta har inträffat.
\code{RemoveNonTwins}\label{code.twins}

Till sist ger vi även en modifierad version av \textsc{NewSegSiev} och \textsc{SubSegSiev} som vi har valt att ge namnen
\textsc{NewSegSievTwins} respektive \textsc{SubSegSievTwins}.
Som namnet antyder kan \textsc{NewSegSiev} användas för att från grunden sålla fram alla primtalstvillingar i det angivna intervallet och funktionen brukas på samma sätt som \textsc{NewSegSiev}.
Funktionen \textsc{SubSegSievTwins} är nödvändig för att \textsc{NewSegSievTwins} ska fungera.

\subsubsection*{\textsc{NewSegSievTwins}}
\code{NewSegSievTwins}
\subsubsection*{\textsc{NewSegSievTwins}}
\code{SubSegSievTwins}