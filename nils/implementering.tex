% Skrivet av Nils

\todo{Förklara hur vi har översatt algoritmener till Python-kod och vad vi har gjort för att snabba upp programmet}

Algoritmerna i \cite{HaraldSieve} presenteras i form av pseudokod som för att kunna användas, måste översättas till något programmeringsspråk
Här valdes Python på grund av gruppens tidigare erfarenhet med språket. 
Det var möjligt att översätta pseudokoden mer eller mindre ordagrant, vilket gjordes och resulterade i en första version av programmet.
Därefter kunde flera förbättringar av koden göras för att korta ned dess körningstid. 
Vissa av förbättringar var möjliga då pseudokoden i \cite{HaraldSieve} är skriven i syfte att tydligt illustrera algoritmerna,
och är således inte ämnad till att vara färdig, optimerad kod.
Andra förbättringar var språkspecifika och åstadkoms genom att jämföra beräkningstiden hos olika funktioner och metoder i Python, för att sedan implementera de som visade sig vara snabba.

\todo{Berätta om väsentliga förbättringar som har gjorts}

\todo{Infoga graf där körningstid av gammal och ny kod jämförs}

\todo{Diskutera vilka ytterligare förbättringar som kan göras}