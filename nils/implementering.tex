% Skrivet av Nils

Algoritmerna i \cite{HaraldSieve} presenteras i form av pseudokod som för att kunna användas, måste översättas till något programmeringsspråk,
här valdes Python på grund av gruppens tidigare erfarenhet med språket. Det var möjligt att översätta pseudokoden mer eller mindre ordagrant,
vilket gjordes och resulterade i en första version av programmet.
Därefter kunde flera förbättringar göras i koden för att korta ned dess körningstid. 
Vissa av förbättringar kunde göras då pseudokoden i \cite{HaraldSieve} är skriven i syfte att tydligt illustrera algoritmerna,
och är således inte ämnad till att vara färdig optimerad kod.
Andra förbättringar kunde göras genom testa och jämföra snabbhet för olika Pythonfunktioner.

Förbättringar som är värda att nämna är:
\begin{itemize}
  \item 
  \item 
\end{itemize}

%\todo{Prata om hur vi skrev algoritmerna i Python och vilka val vi gjorde för att effektivisera koden}


%Vissa av förbättringarna var specifika för hur Python fungerar, 

%medan andra förbättringar kunde göras på grund av att pseudokoden i \cite{HaraldSieve} skrivits främst för att vara förklarande och inte nödvändigtvis effektiv. 