% Skrivet av Nils

I numeriska sammanhang är Eratosthenes såll ett kraftfullt verktyg.
Sållet ger oss en metod för att sålla bland eller faktorisera alla tal upp till något $N$,
där förhållandet mellan $N$ och antalat beräkningar som krävs är nära linjärt \cite[s.333]{HaraldSieve}.
I detta avsnitt utforskas de idéer och algoritmer baserade på Eratosthenes såll som presenteras i \textit{An Improved Sieve of Eratosthenes} av Harald Helfgott \cite{HaraldSieve}, samt en datorimplementation av detta skriven i programmeringsspråket Python. 
Avsnittet består av tre delar.
I den första delen beskrivs algoritmernas funktion och deras underliggande matematiska principer.
Del två är en metoddel där vi beskriver vilka beslut som gått in i att skriva ett snabbt och välfungerande program baserat på dessa algoritmer.
Slutligen visar vi hur programmet kan användas för att ge resultat om fördelningen av primtalstvillingar och frekvensen av primtalsgap.