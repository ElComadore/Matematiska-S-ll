Syftet med denna rapport är att ge läsaren en inblick i det matematiska delområdet sållteori genom att redogöra för dess grundläggande idéer och tillämpningar, samt att presentera en datorimplementation av algoritmen Eratosthenes såll.
%
Vi presenterar Eratosthenes generaliserade såll, samt Bruns och Selbergs såll en åt gången med följande tillvägagångssätt:
Först ges en kortfattad historisk kontext till sållet ifråga, följt av en översiktlig härledning och slutligen, ett exempel på hur sållet kan tillämpas för att ge resultat om bland annat primtalstvillingar och primtal i aritmetiska serier.
%
Efter att de tre sållen har presenterats diskuteras och jämförs orsaken till deras feltermer.
Detta med avsikt att ge läsaren en inblick i vilka möjligheter och begränsningar som finns i sållen som verktyg.
%





%Detta gör vi genom att först ge en mycket kortfattad historisk bakgrund till sållet ifråga

%Detta gör vi genom att kortfattat ge historisk kontext till  för att sedan koncentrera oss på att i stora drag härleda sållen.
%Efter detta visar vi hur sållen kan tillämpas för att ge resultat om bland annat primtalstvillingar och primtal i aritmetiska serier.
%
