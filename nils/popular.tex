% Skrivet av Nils

Primtal är förrädiskt oförutsägbara.
De dyker upp när du minst anar det och verkar nästan bete sig slumpmässigt.
Trots detta är de helt deterministiska i sin natur och gränsen för vad som är och inte är ett primtal är mycket tydlig.
Det är kanske just av denna anledning som primtalen har fascinerat matematiker i tusentals år och fortsätter att göra så än idag.

Hur många primtal finns det? Svaret är att det finns oändligt många.
Om vi istället frågar oss hur många primtal det finns som är mindre än en miljon, så är svaret inte lika lätt.
Förutom talet 2 så är primtal aldrig jämna så vi kan åtminstone utesluta vartannat tal och säga att svaret är mindre än en halv miljon.
Hur går vi vidare härifrån?
Ett naturligt andra steg vore att föra samma resonemang för talet 3;
förutom just 3 så är primtal aldrig delbara med 3 så vi borde kunna dra bort ytterligare en tredjedel från svaret.
Detta är dessvärre inte riktigt sant.
Tal som både är jämna och delbara med 3 har ju uteslutits två gånger.
Det har skett en dubbelräkning och det visar sig att denna typ av fel blir allt vanligare ju fler steg som tas.

Denna rapport utforskar några metoder som kan användas för att hitta svaret på frågor som ovanstående.
Oftast är det inte möjligt att få ett exakt svar och istället måste vi nöja oss med en uppskattning,
vilket givetvis leder till en följdfråga om hur bra uppskattningen är.

